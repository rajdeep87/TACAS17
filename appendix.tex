\appendix
\section{Omitted proofs}

\pscmt{
Rough and sketchy:
%
$espan(V)=$all meet irreducibles $\in A$ that contain at most one variable that is not in $V$.
%
$select(S,PV)$ is the set of statements that contain variables in $PV$ where $PV$ is the set of variables in the previous deduction.
%
$LV(s)$ are the live variables of a statement $s$, i.e.\ the intersection of $PV$ with the variables in $s$.
\begin{theorem}
$$
\gfp\lambda a. \bigsqcap_{s\in S} ded(s,a,A) = 
\gfp\lambda a. \bigsqcap_{s\in select(S,PV)} ded(s,a,espan(LV(s)))
$$
\end{theorem}
Proof sketch:
When computing $ded$, we obtain a deduction that affects variables $V$.
This has two consequences: (1) we have to compute the closure; (2) we have to propagate through all statements $s$ affected by the changes to $V$. 
We do only (2): $espan$ augments the set of affected variables by one transitivity step that we would have to compute in the closure. $ded$ on this subdomain hence performs the deduction w.r.t. $a$ and $s$ plus one transitivity step of the closure. Thus, by induction, we eventually compute the entire closure just by doing (2).

What are the consequences regarding complexity?
Assuming we require $m$ iterations to reach the fixed point with eager closure; we have to perform $|A|^c|S|m$ (idealised) computation steps to do that (where $c=3$ for octagons). With the lazy closure we have to do $|L|^c|S|mk$ computations taking into account that we need $k$ times more iterations to converge. Hence, the approach pays off if $k<<\frac{|A|^c}{|L|^c}$, which seems likely for $c=3$.

}

\section{Detailed experimental results}
