\section{Related Work}
The work of~\cite{sas13} present an instantiation of ACDCL as a decision 
procedure for floating-point arithmetic that uses interval constraint 
propagation.  The use of ACDCL as program analysis is proposed
in~\cite{sas12,tacas12}.  The analysis operate over the abstract domain of 
floating-point and machine-integer.  Thus, it is restricted to bounds 
checking for numerical programs.  A similar technique that lifts DPLL(T) 
to programs is Satisfiability Modulo Path Programs (SMPP). SMPP enumerates 
path programs using a SAT formula which are then verified using abstract 
interpretation.  The lifting of CDCL to first-order theory is proposed 
in~\cite{dpll,cp09} and natural domain SMT framework~\cite{ndsmt}.  
\Omit{
operates on a fixed first-order partial assignment lattice structure, 
where first-order variables are mapped to domain values, similar to constants lattice 
in program analysis.  
}
However, unlike previous works that operate on a fixed first-order 
lattice structure, ACDCL can be instantiated with different 
abstract domains.  This involves model search and learning in 
abstract lattices.     
\Omit {
The abstract lattice in natural domain SMT does 
not have complementable meet irreducibles, and therefore does not 
support generalized clause learning~\cite{sas}.  On the other hand, 
most abstract lattice have complementation property, thus enabling 
ACDCL to perform generalized clause learning.
}
\pscmt{\cite{DBLP:journals/fmsd/BrainDGHK14} and \cite{DBLP:conf/esop/BrainDKS14} are also ACDL instantiations (the latter is even a relational one, but without backtracking)}
