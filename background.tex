\paragraph{Related Work}
The work of~\cite{sas13,DBLP:journals/fmsd/BrainDGHK14} present an
instantiation of ACDCL as a decision procedure for floating-point
arithmetic that uses interval constraint propagation.  
%
The use of
ACDCL as program analysis is proposed in~\cite{tacas12}.  The analysis
operates over the abstract domain of floating-point numbers and
machine integers.  Thus, it is restricted to bounds checking for
numerical programs.  
%
A similar technique that lifts DPLL(T) to
programs is Satisfiability Modulo Path Programs (SMPP) \cite{SMPP}. SMPP
enumerates program paths using a SAT formula, which are then verified
using abstract interpretation.  
%
The lifting of CDCL to first-order theory is proposed
in~\cite{dpll,cp09} and natural domain SMT framework~\cite{ndsmt}.
\Omit{ operates on a fixed first-order partial assignment lattice
  structure, where first-order variables are mapped to domain values,
  similar to constants lattice in program analysis.  } However, unlike
previous works that operate on a fixed first-order lattice structure,
ACDCL can be instantiated with different abstract domains.  This
involves model search and learning in abstract lattices.  \Omit { The
  abstract lattice in natural domain SMT does not have complementable
  meet irreducibles, and therefore does not support generalized clause
  learning~\cite{sas}.  On the other hand, most abstract lattice have
  complementation property, thus enabling ACDCL to perform generalized
  clause learning.  } 
%
\cite{DBLP:journals/fmsd/BrainDGHK14} presents an instantiation of ACDCL for heap-manipulating programs, but uses a simplistic conflict analysis.
%
\cite{DBLP:conf/vmcai/PelleauMTB13} describes a constraint solving algorithm using relational numerical domains, but performs model search only.
%
\cite{DBLP:conf/esop/MineBR16} proposes an algorithm inspired by constraint solvers for inferring disjunctive invariants using the interval domain.
% is presented in . They
%use a relational domain, but focus on the model search, whereas the
%conflict analysis is simplistic and restarts the algorithm with the
%learnt clause instead of backtracking.
%}

