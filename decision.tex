\section{Decision}
A decision $\mathcal{D}$ is a meet irreducible that refines the 
current partial assignment.  A decision can be of the form 
$\langle x \preceq c \rangle$ or $\langle x \succeq c \rangle$, 
where $c$ is the bound and $x$ is a program variable.  A 
decision $\mathcal{D}$ must always be consistent with respect 
to the trail $\mathcal{T}$, that is, it must not contradict with 
the elements in the trail.  Each variable in propositional CDCL 
is assigned at most once at a particular decision level.  However, 
a variable can be assigned several times in our lifting of CDCL to 
numerical domains, each time with increasingly precise bounds.  

Let $\mathcal{V}$ be a set of all singletons and non-singletons 
variables.  A singleton variable is one that has been assigned a 
singleton value, for example, $\langle x:[1,1] \rangle$.  Otherwise, 
the variable is non-singleton.  Given a set of variables $\mathcal{V}$, 
a decision phase heuristically chooses a non-singleton branching 
variable $v \in \mathcal{V}$, a bound $c$, and the polarity ($\succeq$ or 
$\preceq$).  

We implemented several decision heuristics namely, {\em Berkmin}~\cite{}, 
{\em longest-range}, {\em ordered}, {\em random}.  The {\em ordered} 
decision heuristics creates an ordering among non-singleton variables, 
making decision on conditional variables (variables that appear in 
conditional branches) first before choosing numerical variables.  Thus, 
this heuristics is path-sensitive which gives significant speed-up for 
control-intensive benchmarks. 
     
