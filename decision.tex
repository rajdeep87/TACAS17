\subsection{Decision}
A decision $\mathcal{D}$ is a meet irreducible that refines the 
current abstract valuation.  For example, a decision in interval 
domain can be of the form $\langle x R c \rangle$ where 
$R \in \{\leq, \geq\}$, $c$ is the bound.  A decision in 
relational domain such as octagon is of the form $ax - by \leq c$, 
where $x$ and $y$ are variables, $a,b \in \{-1,0,1\}$ are co-efficients, 
and $c \in \mathbb{R}\cup\{\infty\}$ is the bound of the inequality.  
That is, a decision can specify relation between variables, for example 
$\{x \leq y)$.  A decision must always be consistent with respect to 
the trail $\mathcal{T}$, that is, it must not contradict with the elements 
in the trail.  A new decision always increases the decision level by one. 

A singleton meet irreducible $s$ in an abstract domain $A$ is an element 
whose lower bound and upper bound are the same.  For example, 
$\langle x:[1,1] \rangle$ is a singleton meet irreducible in interval domain.  
For an octagon domain, $\langle 1 \geq x-y \leq 1 \rangle$ is a singleton meet 
irreducible for the template $(x-y)$, though the concretisation of individual 
template variables ($x$, $y$) may not be singletons.   

Given a set of variables $\mathcal{V}$ and a current abstract value $v$, 
a decision phase heuristically returns a non-singleton meet irreducible 
by choosing a set of branching variables $\{B\} \subseteq \mathcal{V}$, 
a bound $c$, and a polarity ($\leq$ or $\geq$).  For interval domain, $|B|=1$.  
A decision changes the sate of the solver by adding a new meet irreducible 
$m$ and labelling information $s=decision$ to the trail, which is described below. 
\[decide: \quad (\mathcal{E},S) \rightarrow (\mathcal{E}(m,s),S) \]

%Let $\mathcal{V}$ be a set of all singletons and non-singletons.  
%Otherwise, the variable is non-singleton 

ACDCL supports several decision heuristics namely, {\em ordered}, 
{\em longest-range}, {\em random}, {\em relational} and 
{\em Berkmin}~\cite{eugoldberg07} decision heuristics.  
The {\em ordered} decision heuristics creates an ordering among non-singleton 
meet irreducibles, thereby making decisions on meet irreducibles that involve 
conditional variables (variables that appear in conditional branches) first 
before choosing meet irreducibles with numerical variables.  
The ordering is path-sensitive as it gives an effect of trace partitioning.  

The {\em longest-range} heuristics simply keeps track of the
interval $[l,u]$ range ($r=u-l$) of a meet irreducible and chooses 
a meet irreducible with the longest range value.  This ensures a 
fairness policy in selecting a variable since it guarantees that 
the intervals of meet irreducibles are uniformly restricted.  
%
The {\em random} decision heuristics arbitrarily picks a meet irreducible  
for making decision. 
%
The {\em relational} decision heuristics is only relevant for relational 
abstract domains.  
%
Whereas, a {\em berkmin} decision heuristics is inspired 
from decision heuristic used in Berkmin~\cite{eugoldberg07} SAT solver.  
The berkmin heuristic is currently implemented for interval constraints only.  
The heuristic keeps track of the activity of an interval meet irreducible 
that participate in conflict clauses. 
\Omit {
as well as variables that actively contribute to conflicts but do not explicitly 
appear in conflict clauses.  The set of conflict clauses is 
organized chronologically with the top clause 
as the one deduced in the last.  A branching variable is chosen among the 
free variables whose literals are in the top unsatisfied conflict clause.  
A similar decision heuristics is also implemented in Chaff~\cite{chaff} SAT 
solver, that computes the activity of a variable as the number of occurrences 
of that variable in conflict clauses only. 
}
%
A bound of a meet irreducible is heuristically chosen to be an approximation of the 
arithmetic average of the current bounds.  However, the polarity ($\leq$ or $\geq$) 
of a meet irreducible is chosen randomly.  
