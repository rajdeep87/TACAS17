\subsection{Decision}
A decision $\mathcal{D}$ is a meet irreducible that refines the 
current abstract valuation.  A decision can be of the form 
$\langle x R c \rangle$ or $\langle x R y \rangle$, where 
$R \in \{\preceq, \succeq\}$, $c$ is the bound and 
$\{x, y\}$ are program variables.  A decision must always 
be consistent with respect to the trail $\mathcal{T}$, that 
is, it must not contradict with the elements in the trail.  

Let $\mathcal{V}$ be a set of all singletons and non-singletons 
variables.  A singleton variable is one that has been assigned a 
singleton value, for example, $\langle x:[1,1] \rangle$.  Otherwise, 
the variable is non-singleton.  Given a set of variables $\mathcal{V}$, 
a decision phase heuristically chooses a non-singleton branching 
variable $v \in \mathcal{V}$, a bound $c$, and the polarity ($\succeq$ or 
$\preceq$).  A decision changes the sate of the solver by adding new element 
$m$ and labelling information $s=decision$ to the trail, which is described below. 
\[decide: \quad (\mathcal{E},S) \rightarrow (\mathcal{E}(m,s),S) \]

ACDCL supports several decision heuristics namely, {\em ordered} 
{\em longest-range}, {\em random}, {\em relational} and 
{\em Berkmin}~\cite{eugoldberg07} decision heuristics.  
The {\em ordered} decision heuristics creates an ordering among non-singleton 
variables, thereby making decisions on conditional variables (variables that 
appear in conditional branches) first before choosing numerical variables.  
This variable ordering is path-sensitive as it gives an effect of trace
partitioning.  A {\em longest-range} heuristics simply keeps track of the
interval $[l,u]$ range ($r=u-l$) of a variable and chooses a variable with 
the longest range value.  This ensures a fairness policy in selecting a 
variable since it guarantees that the intervals of variables are uniformly 
restricted.  A {\em random} decision heuristics arbitrarily picks a variable 
for making decision.  A {\em relational} decision heuristics is of the form 
$x R y$ for $R \in \{preceq, succceq\}$ and is only relevant for relational 
abstract domains.  Whereas, a {\em berkmin} decision heuristics is inspired 
from decision heuristics used in Berkmin~\cite{eugoldberg07} SAT solver, which 
keeps track of the activity of a variable that participate in conflict clauses 
as well as variables that actively contribute to conflicts but do not explicitly 
appear in conflict clauses.  The set of conflict clauses is organized 
chronologically with the top clause as the one deduced in the last.  A 
branching variable is chosen among the free variables whose literals are 
in the top unsatisfied conflict clause.  A similar decision heuristics is 
also implemented in Chaff~\cite{chaff} SAT solver, that computes the activity 
of a variable as the number of occurrences of that variable in conflict 
clauses only.  

A bound of a variable is heuristically chosen to be an approximation of the 
arithmetic average of the current bounds.  However, the polarity ($\preceq$ or
$\succeq$) of a variable is chosen randomly.  
