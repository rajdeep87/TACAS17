\section{Abstract Model Search}
%
\Omit {
\begin{figure}[t]
\scriptsize
\begin{tabular}{l|l|l}
\hline
C program & SSA & Octagon \\
\hline
\begin{lstlisting}[mathescape=true,language=C]
int main() {
 unsigned x, y;
 __CPROVER_assume(x==y);
 x++;
 assert(x==y+1);
}
\end{lstlisting}
&
\begin{minipage}{4.40cm}
$\begin{array}{l@{\,\,}c@{\,\,}l}
SSA &\iff& ((g0 == TRUE) \land \\
    &    & (cond == (x == y)) \land \\
    &    & (g1 == (cond \&\& guard0)) \land \\
    &    & (x' == 1u + x) \land \\
    &    & (x' == 1u + y || !1))
\end{array}$
\end{minipage}
&
\begin{minipage}{3.75cm}
$\begin{array}{l@{\,\,}c@{\,\,}l}
C &\iff& ((x' > 1) \land (-x'-y < -2) \land \\
  &    & (-x-x' < -2) \land (y-x' < 0) \land \\                                                                
  &    & (x-x' < 0) \land (y > 0) \land \\
  &    & (x > 0) \land (-x'-y < 0) \land \\
  &    & (x+y > 1) \land (y-x < 1) \land \\
  &    & (x'-y < 2) \land (x-y < 1) \land \\
  &    & (x+y > 0) \land (x+x' > 0) \land \\
  &    & (x'-x < 2))
\end{array}$
\end{minipage}
\\
\hline
\end{tabular}
\caption{C Program, corresponding SSA and Octagonal Inequalities}
\label{ssa}
\end{figure}
}
%
%    
\begin{algorithm2e}[t]
\DontPrintSemicolon
\SetKw{return}{return}
\SetKwData{sat}{sat}
\SetKwData{conflict}{conflict}
\SetKwData{unsat}{unsat}
\SetKwData{unknown}{unknown}
\SetKwData{true}{true}
\SetKwInOut{Input}{input}
\SetKwInOut{Output}{output}
\SetKwFor{Loop}{Loop}{}{}
\SetKw{KwNot}{not}
\begin{small}
\Input{A program in the form of a set of abstract transformers $\abstransset$,
a propagation trail $\trail$, and a reason trail $\reasons$.}
\Output{\sat or \conflict or \unknown}
$\worklist \leftarrow \initworklist_{\propheur}(\abstransset)$ \;
\While{$!\mathit{worklist.empty}()$} 
{
  $\abstransel{\subdomain} \leftarrow \mathit{worklist.pop}()$ \; 
  \rmcmt{how do we get the $\subdomain$ at this point ?}
  $\newdeductions \leftarrow \onlynew(\abstransel{\subdomain}(\abs(\trail)))$\;
  \uIf{$\newdeductions = \bot$} {
    $\reasons[\bot] \leftarrow \abstransel{\subdomain}$ \rmcmt{shouldn't this be
    just $\abstransset$} \;
    $\mathit{worklist.clear}()$ \;
    \return \conflict \;
  }
  \uElse
  {
%    $\newdeductions=\decomp(\absval)\setminus\decomp(\abs(\trail))$\;
    $\trail \leftarrow \trail \cdot \decomp(\newdeductions)$ \; 
    $\reasons[|\trail|] \leftarrow \abstransel{\subdomain}$ \;
    $\updateworklist_{\propheur}(\newdeductions, \worklist)$ \; 
  }
}
\lIf{$\abs(\trail)$ is $\gamma$-complete} {
  \return \sat;
}
 \return \unknown \;

\end{small}
\caption{Abstract Model Search $\mathit{deduce}_{\propheur}(\abstransset,\trail,\reasons)$ \label{Alg:ms}}
\end{algorithm2e}
%  
A model search procedure in a SAT solver involves two steps -- {\em deductions} 
using the unit rule refines current partial assignments and 
{\em decisions} to heuristically guess a value for an unassigned 
literal.  The unit rule overapproximates a model transformer and deduction 
computes a greatest fixed point over the partial assignments
domain~\cite{dhk2013-popl}.  We present an abstract model search procedure 
that computes a greatest fixed point over meet irreducible deduction 
transformer in $S$ \pscmt{???}.  
%

%-------------------------------------------------------------------------------
\subsection{Abstract Deduction Transformers} \label{sec:abst}
%-------------------------------------------------------------------------------

To make our algorithm efficient, we have to focus abstract
transformers on performing only the minimally necessary
work. 
%
We thus define a specialised variant of the abstract transformer to compute
deductions w.r.t.\ a given subdomain $\subdomain\subseteq \domain$,
which we call \emph{abstract deduction transformer}.
%
A subdomain contains a chosen subset of the elements in $\domain$ including $\bot$ and $\top$
\begin{equation}\label{eq:at2}
\abstrans{\domain}{\constraint}^\subdomain(\absval)=\absval\meet_\domain \alpha_\subdomain(\{\val\mid \val\in\gamma_\domain(\absval), \val\models \constraint\})
\end{equation}
For $\subdomain=\domain$, the abstract deduction transformer behaves in the same way as the abstract transformer defined in Eq.~\ref{eq:abstrans} in Section~\ref{sec:domains}.

%We construct the subdomain from a set of variables $\vars$ such that $\subdomain_\domain(\vars)$ is the set of meet irreducibles $\in \domain$ that contain at last one variable in $\vars$ and most one variable that is not in $\vars$. \pscmt{split into two parts}
%
%For example, $\subdomain=\domain[\{y\}]$ being the octagons domain over variables $\{x,y,z\}$ contains all octagonal meet irreducibles involving $\{y\}$, and the pairs $\{x,y\}$ and $\{y,z\}$, but not the meet irreducibles\pscmt{wrong} over $\{x\}$, $\{z\}$ and~$\{x,z\}$.

The first advantage of this definition is that with the help of $\subdomain$
we can control how we perform propagation using our abstract transformers.
%
For example, for $\absval=(0\leq y \leq 1 \wedge 5\leq z)$, we have
$\abstrans{\intervals[\{x,y,z\}]}{x=y+z}^{\intervals[\{x\}]}(\absval)=\absval\meet(x\geq
6)$. For $x=y+z$, this performs a right-hand side to left-hand side propagation and
hence emulates a forward analysis, whereas
$\abstrans{\intervals[\{x,y,z\}]}{x=y+z}^{\intervals[\{y,z\}]}=\absval\meet\top$
emulates a backward analysis.

We call the propagation with arbitrary subdomains,
e.g. $\subdomain=\domain$, \emph{multi-way propagation}, which is able
to simultaneously perform forward and backward propagation.
%
Note that the restriction to a subdomain makes the transformer less
precise. Therefore, we have the property
$\abstrans{\domain}{\constraint}^\domain(\absval)\sqsubseteq
\abstrans{\domain}{\constraint}^\subdomain(\absval)$.
However, the choice of a subdomain makes the deductions more efficient.

The second advantage of Eq.~(\ref{eq:at2}) is that we can compute the
\emph{closure} operation for relational domains in a lazy manner. To
this end, we construct a $\subdomain=\makesubdomain_\domain(\vars)$
which is sufficient to perform one step of the closure operation each
time the abstract transformer is applied.
%
For example, let us consider $\domain=\octagons[\{x,y,z\}]$ and
$\vars=\{y\}$. An octagonal inequality relates at 
most two variables. Thus it is sufficient to consider the subdomain
$\makesubdomain_\domain(\{y\})=\octagons[\{y\}]\cup\octagons[\{x,y\}]\cup\octagons[\{y,z\}]$,
which will compute the one-step transitive relations of~$y$ with each
of the other variables. 
%
\Omit{
%TODO
Hence, we do not compute the full closure in the full 
domain $\octagons[\{x,y,z\}]$, but we compute only a single step of the closure when the abstract deduction transformer is applied.
%
Only if $\abstrans^\subdomain$ deduces new information about $x$ or $z$
then the next step of the closure will be 
}
If this indeed affects $x$ and~$z$ then their
relations will be inferred in a later step of the propagation
algorithm, thus gradually computing the closure. 

%Note that this might
%increase the number of propagation steps, but it reduces the size of
%the subdomain used in the abstract deduction transformer.

%Moreover, we want to know what the reasons for a specific deduction are.
%\[ded(s,a,L)=\{R\rightarrow d\mid R=reasons(s,a,L,d), d \in decomp(\llbracket s \rrbracket^\sharp(a,L))\}\]
%where 
%$reasons(s,a,L,d)=\text{argmin}_{R\in\{R'\mid R'\subseteq decomp(a), \llbracket s\rrbracket^\sharp(R,L)=d\}} |R| $.

%-------------------------------------------------------------------------------
\subsection{Algorithm for Deduction Phase}
%-------------------------------------------------------------------------------
%

Algorithm~\ref{Alg:ms} presents the deduction phase $\deduce$ in 
abstract model search procedure.  The input to the algorithm is 
the set of abstract transformers, a propagation trail and a reason 
trail.  Additionally, the procedure $\deduce$ is parameterised with 
a propagation heuristic. 
%
The algorithm maintains a  {\em worklist}, which is a queue that contains 
abstract deduction transformers.  
The propagation heuristics provides two 
functions $\initworklist$ and $\updateworklist$.
The order of the elements in the worklist and the subdomains $\subdomain$
attached to each of the abstract deduction transformers
determine the propagation strategy (forward, backward, multi-way).
%
The {\em forward} propagation strategy initialises the worklist with
abstract deduction transformers that contains constants in the
right-hand side and the subdomain is constructed via $\makesubdomain$
from variables in left-hand side of the equality constraints
originating from the assignment statements in the program.
%
The {\em backward} iteration strategy intialises the worklist with the assertions and the subdomain is constructed from the right-hand side variables.
%
The {\em multi-way} iteration strategy initialises the worklist 
with set of all transformers and each subdomain contains the variables occurring in the respective transformers.

Depending on the propagation heuristics $\updateworklist$ adds
abstract deduction transformers to the worklist that contain variables
that appear in $\newdeductions$.  It also updates the subdomains of
existing abstract deduction transformers that are in the worklist.

The function
$\onlynew(\absval)=\bigsqcap(\decomp(\absval)\setminus\decomp(\abs(\trail)))$
filters out all meet irreducibles that are already on the trail.

\Omit{
For every abstract deduction transformer in the worklist, 
the set of new meet irreducibles inferred by the domain is added 
to the $\trail$.  Additionally, the reason trail $\reasons$ is 
updated with the corresponding abstract transformer as the reason for inferring 
those new meet irreducibles.  If a conflict is deduced, 
%that is, $\mathit{meet\_irrd} = \bot$ is inferred by the domain, 
the algorithm terminates with \textsf{conflict}.  However,
when a fixed-point is reached, the abstract value $v$ computed from 
the trail is checked for set of models.  This is determined by 
checking whether the set of assignments in $v$ is 
$\gamma$-complete~\cite{dhk2013-popl}. If $v$ is $\gamma$-complete, that is, the 
abstract valuation $v$ is abstractly satisfying, the abstract model search 
terminates with \textsf{sat}.  Otherwise, the algorithm returns \textsf{unknown} and the 
model search procedure makes a new decision.    
}
