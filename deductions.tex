\section{Propagation Strategy}
Let $L$ be the number of constraints generated by a domain $D$
for a program with $N$ variables. For $D$ = {\em Interval}, the 
maximum value of $L$ is $2*N$, whereas the maximum value of $L$ is 
$2*(N^2)$ for $D$ = {\em Octagons}. Note that the conjunction of all 
constraints in the set $L$ forms an octagon.

Overapproximating.


\begin{figure}[t]
\scriptsize
\begin{tabular}{l|l}
\hline
C program & SSA \\
\hline
\begin{lstlisting}[mathescape=true,language=C]
int main() {
 unsigned x, y;
 __CPROVER_assume(x==y);
 x++;
 assert(x==y+1);
}
\end{lstlisting}
&
\begin{lstlisting}[mathescape=true,language=C]
cond == (x == y)
guard1 == (cond19 && guard0)
x' == 1u + x
x' == 1u + y || !guard1
guard0 == TRUE
\end{lstlisting}
\\
\hline
\end{tabular}
\caption{C Program ad its corresponding SSA}
\label{swssa}
\end{figure}

\subsection{Forward Iteration Strategy}
For forward iteration on Figure~\ref{swssa}, we need atleast 
one decision to deduce BOTTOM and prove safety. The deduction 
happens from right to left in forward iteration strategy.

The initialization of live variable and subsequent update depends on the iteration strategy.

\subsection{Backward Iteration Strategy}

\subsection{Chaotic Iteration Strategy}  
The set of octagonal equilities corresponding to Figurei~\ref{sw-ssa}
is shown below. For chaotic, no decision is required to deduce BOTTOM.
\begin{figure}[t]
\scriptsize
\begin{tabular}{l}
\hline
Octagonal Inequalities \\
\hline
\begin{lstlisting}
x' > 1                                                                      
-x' - y < -2                                                               
-x - x' < -2                                                               
y - x' < 0                                                                 
x - x' < 0                                                                 
y > 0                                                                       
x > 0                                                                       
-x' - y < 0                                                                
x + y > 1                                                                 
y - x < 1                                                                 
x' - y < 2                                                                 
x - y < 1                                                                 
x + y > 0                                                                 
x + x' > 0                                                                 
x' - x < 2  
\end{lstlisting}
\\
\hline
\end{tabular}
\caption{Octagonal Inequalities}
\label{octagons}
\end{figure}


Let us consider an example. 
int id(int a)
{
  return a;
}

void main() {
 int x,y,z;
 \_Bool c;
 if (c)
   x = -1;
 else
   x = 2;
 int k = id(x);
 z = x * k;
 assert((z!=-2));
}

With forward iteration strategy using octagon domian, the program 
is proved without any decision. However, with the forward strategy
using interval domain, we just required one decision to prove the program. 
