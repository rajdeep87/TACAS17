\section{Definitions}
%
\textit{Literal:} A Literal is a meet irreducible which are minimum complementable
elements which specify that for certain program variable, there is a certain
bound. An example of literal is $x \leq 0$.

Let us define the notion of literals with respect to interval domain. 
An interval $([l:u])$ is defined by a lower bound $(l)$ and an upper bound $(u)$.
Let I and J be two intervals. The relation between I and J is defined as follows:\\
We say $I \leq J$ or {\em (I leq J)} if $(I.u \leq J.u \wedge I.l \geq J.l)$\\
We say {\em I disjoint J} if $(I.u < J.l \vee  I.l > J.u)$.

Let $x$, $x'$ be the intervals of a variable $p$ in current 
partial assignment and the current clause respectively. 
We define a literal and a clause to be satisfiable or unsatisfiable 
or contradicting with respect to an interval abstraction as follows. \\
\textit{Satisfiable literal:} 
We say $x'$ is satisfiable literal if $(x \leq x')$.

\textit{Unsatisfiable literal:}
We say $x'$ is satisfiable literal if $!(x \leq x') \wedge$ $!$(x disjoint $x'$).

\textit{Contradicting Literal:}
We say $x'$ is contradicting literal if $!(x \leq x') \wedge$ (x disjoint $x'$).

\textit{Clause:} A clause is a disjunction of one or more meet irreducibles. An 
example clause is $(x \geq 0 \vee y \geq 5 \vee y \leq 10 \vee y \leq 7)$.

\textit{Satisfiable clause:}
If at least one meet irreducible in a clause is satisfiable, then a clause is said to 
be satisfiable. For example, consider a clause $C=(x<4 \vee y>10)$. Let the 
current partial assignment be $x \in [0,3]$, then $C$ is a satisfiable clause. 

\textit{Conflicting clause:}
If all the meet irreducibles in a clause are contradicting, then the clause is said to
be conflicting. For example, consider a clause $C=(x<4 \vee y>10 \vee z<15)$. Let the 
current partial assignment be $x \in [5,13]$ and $y \in [-2,9]$ and $z \in
[17,32]$, then $C$ is a conflicting clause.

\textit{Unsatisfiable clause:}
If no meet irreducibles in a clause are satisfiable or some meet
irreducibles are not satisfiable and the rest are contradicting, we call a 
clause unsatisfiable. For example, consider a clause $C=(x<4 \vee y>10 \vee z<15)$. 
Let the current partial assignment be $x \in [3,10]$, $y \in [8,10]$ and $z \in [12,20]$, 
then $C$ is a unsatisfiable clause. 

\textit{Unit clause:}
If all meet irreducible but one is contradicting in a clause, we call the clause
to be unit. For example, consider a clause $C=(x<4 \vee y>10 \vee z<15)$. Let the 
current partial assignment be $x \in [5,13]$ and $y \in [-2,9]$ and $z \in
[10,12]$, then $C$ is a unit clause where the unit literal is $z$. 

\textit{Unit Rule:} A unit rule is the best abstract transformer 
(strongest post-condition or weakest pre-condition). 

\textit{Boolean constraint propagation (BCP):} BCP is the repeated 
application of unit rule. This corresponds to computing the greatest fixed point.

\textit{Characteristics of Conflict clause}
\begin{enumerate}
\item A conflict clause must include asserting cuts. An asserting cut is a cut
that contain exactly one node at the current decision level. Assertion cuts yields 
clauses that can be used to derive new information after backtracking.

\item A conflict clause must be UNIT after backtracking. 

%\item There can be multiple cuts and hence multiple UIPs. In other words, there
%can be multiple incomparable reasons for a conflict. But conflict analysis
%procedure choses one that is asserting. 

%\item The conflict clause should be made false by the current partial assignment
%and thus exclude an assignment leading to conflict. 
\end{enumerate}

\textit{Backjumping:}
The backjumping level is defined by the literal of the conflict clause assigned
at the level that is the closest to the conflict one. In other words, the
backjumping level is the level closest to the root (decision level 0)  where the
conflict clause is still unit. If a conflict clause is "globally"  unit, then
the backjumping level is the root of the search tree.
