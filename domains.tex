\section{Preliminaries}
This section explains the program model and the abstract domains that
we are going to use.
% \subsection{Programs:}
% %
% A {\em program} $\mathcal{P}$ is formally defined as follows.
% \[
% \begin{array}[t]{@{}lll}
% \mathcal{P} & {:}{:}{=} & \mathit{Procedure} \\
% \mathit{Procedure} & {:}{:}{=} & \mathit{Statement} \mid \mathit{Procedure} \\
% \mathit{Statement} & {:}{:}{=} & CStatement \mid \mathit{function(Var_1,\dots,Var_n)} \\
% \mathit{CStatement} & {:}{:}{=} & x {:}{=} exp \mid \mathit{ITE}(b, s_1,s_2) \mid \mathit{s_1;s_2} \mid loop\{s\} \\
% \end{array}
% \]
% Consider sets of expressions $Exp$ and Boolean expressions $BExp$
% over variables $Var$ of $\mathcal{P}$.  The variables in $Var$ 
% can take numeric values in $Val$.  A procedure is denoted as 
% $\mathit{function(Var_1,\dots,Var_n)}$.  A $\mathit{CStatement}$ 
% is an assignment, conditional, sequential concatenation or a loop. \\
% \textbf{Control-flow Graph:} A CFG is a triple $(Loc, E, lbl)$, 
% where $Loc$ is the set of locations with an unique start 
% location $(init)$ and error location $(err)$, $E$ is the set 
% of control-flow edges that are labelled with the set 
% $lbl \in Statement$.  For the purpose of illustration, we 
% assume that all the procedures are inlined.  

\begin{figure}
\[
\begin{array}{rcl}
g_0 &=& \true \\
\multicolumn{3}{l}{0 \leq v\leq N} \\
g_1 &=& g_0\wedge c\\
x_0 &=& v \\
x_1 &=& -v \\
x_2 &=& g_1?x_0:x_1 \\
g_2 &=& g_1 \vee g_0\wedge \neg c\\
z &=& x_2 * x_2 \\
\multicolumn{3}{l}{z<0 \wedge g_2}
\end{array}
\]
\caption{\label{fig:ssa}
SSA for the example in Figure~\ref{fig:example}
}
\end{figure}

%-------------------------------------------------------------------------------
\subsection{Program Representation}  
%
We consider loop-free programs (respectively bounded unwindings of
programs with loops or recursion) with safety properties given as
assertions in the code.
%
Such a program is represented as set $\mathcal{P}$ of
\emph{statements} $s$.  These are constraints as typically obtained by
translating the program into single static assignment (SSA) form via a
data flow analysis.
%
Figure~\ref{fig:ssa} shows the constraints for the example program in
Figure~\ref{fig:example}.
%
Assignments such as \texttt{x=v;} become equalities $x_1=v$ where the
left-hand side variable gets a fresh name (subscript).
%
Control flow is encoded into guard variables such as $g_1=g_0\wedge c$.
%
Data flow joins become conditional expressions, e.g.\ $x_3=g_1?x_1:x_2$.
%
Assertions $\mathcal{A}$ are constraints $a$ of the form $g_3
\Rightarrow z\geq 0$, meaning that if $g_3$ holds (i.e.\ the assertion
is reachable) then the assertion must hold.

%For example, figure~\ref{swssa} presents the program analysis constraints for 
%a C program in Static Single Assignment (SSA) form.  

% \subsection{Concrete Semantics}
% \pscmt{not sure that adds anything for TACAS}
% The {\em concrete} domain is a lattice 
% of concrete environments, $Env = Var \rightarrow Val$, and is 
% defined by $CDom = (Env, \sle_{C}, \join_{C}, \meet_{C})$.
% A transformer, $post_{stmt}$, of concrete domain defines 
% the effect of a statement $stmt$ on the concrete domain, 
% $post_{stmt} : \powerset(Env) \rightarrow \powerset(Env)$.  
% 
% A state in concrete domain is a tuple $\langle l, \sigma \rangle$, 
% where $l$ is a location and $\sigma \in Env$.  A trace is a sequence 
% of states $(l_0, \sigma_0), \ldots (l_n, \sigma_n)$ such that for all 
% $0 \leq i \leq n$, there exists a cfg egde $(l_{i}, l_{i+1})$ 
% if $\sigma_{i+1} \in post_{i}(\sigma_{i})$. 

%-------------------------------------------------------------------------------
\paragraph{Safety equation}  
%
Based on above program representation, the safety equation solved by the ACDCL solver is the following:
%
\begin{equation}\label{eq:safety}
\varphi:= \bigwedge_{s\in\mathcal{P}} s \wedge \neg \bigwedge_{a\in\mathcal{A}} a
\text{ is unsatisfiable if and only if the program is safe.}
\end{equation}

%-------------------------------------------------------------------------------
\subsection{Abstract Domain}
%-------------------------------------------------------------------------------
The results that we present on lifting ACDL to relational domains hold
for any relational domain.
% 
In our prototype implementation and for illustrative purposes in this
paper we use a product domain $\mathcal{B}^b\times\mathcal{R}$ where
$\mathcal{B}$ is the Boolean domain
$\langle\{\true,\false,\bot,\top\},\Rightarrow,\wedge,\vee\rangle$,
$b$ is the number of Boolean variables in the program, and $\mathcal{R}$
is a \emph{relational} domain over the numerical (bitvector) variables
in their respective data types (currently, signed and unsigned
integers are supported).
%
We instantiate $\mathcal{R}$ with the template polyhedra domain.

For notational convenience we will denote elements of
$\mathcal{B}^b$ by their concretisations to propositional formulae,
e.g.\ for the Boolean variable vector $(g_1,c)$ with abstract value
$(\false,\top)$ we will write $\neg g_1$.

%-------------------------------------------------------------------------------
\paragraph{Template Polyhedra Abstract Domain}
%
The template polyhedra domain~\cite{sriram} represents sets $X$ of
valuations of a vector of variables $\vec{x}$ such that
$\mat{C}\vec{x}\leq\vec{d}$ for a fixed coefficient matrix $\mat{C}$
and a constant vector $\vec{d}$. The domain of $\vec{d}$ is augmented
by a special element to denote the minimal element $\bot$ of the lattice.
%
The abstraction function is defined as $\alpha(X) = \min \{\vec{d}\mid
\mat{C}\vec{x}\leq\vec{d}, \vec{x}\in X\}$ where $\min$ is
applied component-wise. There are several optimisation-based
techniques for computing the domain
operations~\cite{sriram,GS07b,BJKS15}. In our implementation we use
the strategy iteration approach available in 2LS~\cite{BJKS15}.
%
For notational convenience we will use conjunctions of linear
inequalities, e.g.\ $x_1\geq 0 \wedge x_1-z\leq 30$ to denote elements
of the domain. $\true$ is $\top$ and $\false$ is $\bot$.

% \paragraph{Interval Abstract Domain} 
% The {\em Interval} abstract domain is a lattice 
% $ItvDom = (ItvElm, \sle_{I}, \join_{I}, \meet_{I})$, where
% $ItvElm: (Var \rightarrow Itv) \cup \{\bot\}$, and $Itv$ is 
% the set of intervals of type $[l,u]$ over numeric data 
% type with $l \leq u$. The least element is $\bot$ and the 
% greatest element is $\top$ which maps all variables to their
% minimum ($min$) and maximum ($max$) values.  An interval 
% $\langle x, [min, v] \rangle$ is written as $x \leq v$.  The 
% partial order $\sle_{I}$ over elements in the set $Itv$ is 
% given by $I_1 \sle_{I} I_2$ if $I_2$ contains $I_1$.
% A join $(\join_{I})$ of two intervals $\langle x_1 \rightarrow [l_1, u_1], 
% x_1 \rightarrow [l_2, u_2] \rangle$ is an interval 
% $\langle x_1 \rightarrow [min(l_1, l_2), max(u_1, u_2)]$.
% A meet $(\meet_{I})$ of two intervals $\langle x_1 \rightarrow [l_1, u_1], 
% x_1 \rightarrow [l_2, u_2] \rangle$ is an interval 
% $\langle x_1 \rightarrow [max(l_1, l_2), min(u_1, u_2)] \rangle$.
% The galois connection between the concrete environment and intervals is 
% given as follows.
% \[\alpha(x) = \{[inf(x), sup(x)] | x \in Var\} \qquad   \alpha(\emptyset) = \bot \]
% \[\gamma[p,q] = \{c \in \mathbb{Z} | p \leq c \leq q\} \qquad \gamma(\bot) = \emptyset \]
Template polyhedra subsume, for example, interval and octagon domains.

\begin{wrapfigure}{r}{4.5cm}
\vspace*{-3ex}
\scalebox{.65}{\import{figures/}{octagons.pspdftex}}
\caption{An example of Octagon}\label{octagon}
\vspace*{-3ex}
\end{wrapfigure} 

For a program with $N$ variables, the matrix $\mat{C}$ for intervals
generates at most $2N$ inequality constraints, one for the upper and lower
bounds of each variable.
%
For octagons, we have at most $2N^2$ inequalities, one for the upper
and lower bounds of each variable and sums and differences for each
pair of variables. Figure~\ref{octagon} shows an 
example octagon and its associated inequalities.   

The interval domain is a non-relational domain because a single
inequality only contains a single variable.
%
The octagon domain, however, is relational.  A fundamental difference
between relational and non-relational domains is that relational
domains require the computation of a \emph{closure} in order to obtain
a normal form that is usually required for efficient domain operations.
An example of a closure computation is shown below.
\[\mathit{closure}((x-y \leq 4) \wedge (y-z \leq 5))=((x-y \leq 4) \wedge (y-z \leq 5) \wedge (x-z \leq 9)) \]  
The closure computes all implied domain constraints.  For octagons,
the closure operation is very expensive (cubic complexity in the
number of program variables).
%
We will see that our algorithm performs closure computations lazily,
postponing them to the point where they are actually necessary.

% \paragraph{Octagon Abstract Domain} \pscmt{TODO: simplify}
% The {\em Octagon} abstract domain is a lattice 
% $OctDom = (OctElm, \sle_{O}, \join_{O}, \meet_{O})$, where
% $OctElm: (Var \times Var \rightarrow (\mathbb{R} \cup \{\infty\})) \cup \bot$. 
% The least element is $\bot$ that contains all unsatisfiable 
% set of inequalities and the greatest element is $\top$ which 
% maps the bounds of all octagonal inequalities to $\infty$. 
% The partial order $\sle_{O}$ over elements in the set $OctElm$ is 
% given by $O_1 \sle_{O} O_2$ iff the bounds of each inequalities in $0_1$ 
% is included (by $\leq$ order) in the bounds of corresponding inequalities 
% in $O_2$, that is, the octagons are ordered by the inclusion relations.

% The join $(\join_{O})$ of two octagons, $\langle (v_i \times v_j \rightarrow N_1),
% (v_i \times v_j \rightarrow N_2) \rangle$, is not necessarily an octagon 
% and is computed by taking piece-wise maximum of bounds of corresponding 
% octagonal inequalities, $\langle (v_i \times v_j \rightarrow max(N_1, N_2) \rangle$.
% However, the meet $(\meet_{O})$ of two octagons, $\langle (v_i \times v_j \rightarrow N_1),
% (v_i \times v_j \rightarrow N_2) \rangle$, is always an octagon and is 
% computed by taking piece-wise minimum of bounds of corresponding 
% octagonal inequalities, $\langle (v_i \times v_j \rightarrow min(N_1, N_2)$.  A closed 
% octagon is the smallest octagon following the partial order $\sle_{O}$, among all 
% the octagons that abstract the same concrete values.

% The octagon domain is a relational abstract domain that permits $2n^2$ 
% linear inequalities between $n$ program variables.  The octagonal 
% inequalities are of two types: binary or unary inequalities as shown below.
% \[Binary: \pm v_i \pm v_j \leq a, v_i \neq v_j \qquad Unary: v_i \leq b, \{a, b\} \in \mathbb{R} \cup \infty \]  
% An element in octagon domain, $OctElm$, is a conjunction of such 
% inequalities and is called an octagon.  
%
% For building program analyzers using octagon domains, the domain also 
% provides few operators like {\em widening ($\nabla$)} and {\em closure ($*$)}.  
% The widening operator is used to accelerate convergence for loops in the program
% and has a quadratic complexity in the number of variables.  If the bound of an 
% octagonal inequality increases every iteration, then $\nabla$ sets the bounds 
% to $\infty$.  However, the closure operator is often used to reduce the degree 
% of over-approximation resulting from the join operation. 

\pscmt{TODO from here}
 
The abstract transformer, $apost_{stmt}$, captures the effect of different program 
statements in the abstract domain. The transformer is precise for octagonal 
assignments $(x:=y+1)$ but imprecise for non-octagonal assignments $(x:=y+z)$, 
as shown below.
\[apost_{x:=y+1}(a) = b = \langle x-y \leq 1, y-x \leq 1 \rangle \qquad apost_{x:=y+z}(b) = \langle \top \rangle \]  

\begin{definition}{(Abstract Valuation)} An {\em abstract valuation} is a
mapping of variables to an element of abstract domain, for example 
a mapping of variable $x$ to an interval environment is given by 
$\langle x \mapsto [2,5] \rangle$ or a mapping of $\{x,y\}$ to octagon 
environment is given by $\langle x-y \mapsto 0, y-x \mapsto 0 \rangle$.  
An abstract valuation is {\em atomic} if each variable is mapped to a singleton 
value or to $\bot$.  
\end{definition}

\begin{definition}{(Meet Irreducibles)} A {\em meet irreducible} $(M)$ 
in a complete lattice structure $A$ is a minimum complementable element 
$M \in A$ that has the following property.
\[\forall M_1, M_2 \in A, M_1 \meet M2 = M \implies (M = M_1 \lor M = M_2), M \neq \top \]  
\end{definition}

A meet irreducible in partial assignments domain are the singleton 
assignments, for example $\{x \mapsto true \}$, which represents the set 
of all propositional assignments where the literal $x$ is mapped to true.  
A meet irreducible in interval domain is $\langle x \leq n \rangle$ or 
$\langle x \geq n \rangle$.  

%
\begin{definition}{(Complementable Meet Irreducibles)} A complementable meet
irreducible $\bar{M}$ of an abstract lattice $A$ is the complement of a meet 
irreducible $M \in A$ such that $\bar{M} \in A$ and the concretisation of $M$ 
is the set complement of $\bar{M}$.  If every meet irreducible in $A$ is
complementable, then $A$ is said to be have complementable meet irreducibles.  
\end{definition}

\Omit {
An important property of meet irreducibles in case of partial assignments 
domain is that they have precise complements.  
For example, the complement of $\{x \mapsto true \}$ is a 
singleton element,$\{x \mapsto false \}$, in the partial assignments domain. 
}

\subsection{Abstract Transformers}

In a typical abstract interpretation based Galois-connection setting
over an over-approximate domain, every concrete element has a unique
over-approximate representation in the abstract.  Likewise, every
concrete transformer is over-approximated by a unique abstract
transformer.  We now define an abstract deduction transformer.

\begin{definition}{(Abstract Deduction Transformer)} An abstract deduction
transformer, $\widehat{ded_{\varphi}}$ for a formula $\varphi$ over an abstract 
domain $A$ is a sound approximation of a concrete model transformer
$ded_{\varphi}$, given by $\widehat{ded_{\varphi}} : A \rightarrow A$, such that 
$\forall a \in A: \widehat{ded_{\varphi}}(a) \in \{\top, \bot, m\}$, where 
$m \in A$ is a meet irreducible.   
\end{definition}

Let us consider a formula $\varphi = (x:=y-1)$ to be analyzed over 
an interval abstract domain, $A = ItvDom$, and let $a = \langle y:[3, 5]
\rangle \in ItvDom$, then $\widehat{ded_{\varphi}}(a) = a \meet \langle x:[2, 4]
\rangle$.  An abstract deduction transformer is typically computed in the form 
of strongest post-condition or a weakest pre-condition of a formula in the 
abstract domain.  

A meet decomposition of the outcome of abstract deduction transformer 
is obtained by taking a meet of the element $\langle y:[3, 5] \rangle, 
\langle x:[2, 4] \rangle \in ItvDom$ which gives set of meet irreducibles, 
$\{ \langle y \succeq 3 \rangle, \langle y \preceq 5 \rangle, 
\langle x \succeq 2 \rangle, \langle x \preceq 4 \rangle \}$, that are 
precisely complementable.

\begin{definition}{(Meet Decomposition)} A meet decomposition of an abstract
element $a \in A$ is a set of meet irreducibles $M \subseteq A$ such that 
$\forall m_i \in M, \meet(m_i) = a$, where $max(i) = |M|$.
\end{definition}
 
For a program with $N$ variables, let $L$ be the total number of 
meet irreducibles returned by a domain $D$.  For $D$ = {\em ItvDom}, the 
maximum value of $L$ is $2*N$, whereas the maximum value of $L$ is 
$2*(N^2)$ for $D$ = {\em OctDom}. Note that an octagon is the conjunction 
of all octagonal inequalities in the set $L$.
%
%========================
\subsection{Literals and Clauses for Abstract CDCL}
%
\textit{Literal:} A Literal is a meet irreducible which specify that 
for certain program variable, there is a certain bound. An example of 
literal in interval domain is $x \in [0,5]$. An example of octagonal 
literal is given by $((x-y \leq 0) \wedge (y-x \leq 0))$.

Let I and J be two interval literals of the form $x \in [l, u]$ with $l$ 
as lower bound and $u$ as upper bound. We say $I \leq J$ or {\em (I leq J)} 
iff $(I.u \leq J.u \wedge I.l \geq J.l)$.  Whereas, {\em I disjoint J} iff 
$(I.u < J.l \vee  I.l > J.u)$. 

Similarly, let $O_1$ and $O_2$ be two octagonal literals of the form 
$x-y \leq c$.  Then, $O_1 \leq O_2$ {\em ($O_1$ leq $O_2$)} iff 
$O_1.c \leq O_2.c$.  Two octagons are disjoint ({\em $O_1$ disjoint $O_2$}) 
iff $O_1:= (x-y < c)$ and $O_2:= (x-y > c)$.  

\textit{Clause:} A clause is a disjunction of one or more meet irreducibles. 
An example clause is given by $(x \geq 0 \vee y \geq 5 \vee y+z \leq 10)$.

Let $x$, $x'$ be the literals in clause $C$ and current partial assignment 
$A$ respectively.  We call a literal $x \in C$ to be {\em satisfiable} with 
respect to the literal $x' \in A$ iff $(x \leq x')$.  We call $x$ to be  
{\em unsatisfiable} with respect to $x'$ iff $!(x \leq x') \wedge$ $!$(x disjoint $x'$).
A literal is said to be {\em contradicting} iff $!(x \leq x') \wedge$ (x disjoint $x'$).

Similarly, a clause may be catagorized into four classes -- satisfiable,
unsatisfiable, conflicting, unit. \\ 
\textit{Satisfiable clause:}
A clause $C$ is said to be {\em satisfiable} with respect to a current partial 
assignment $A$ if at least one literal in $C$ is satisfiable. For example, 
consider a clause $C=(x<4 \vee y>10)$ and let the current partial assignment 
be $A: x \in [0,3]$. Then $C$ is a satisfiable clause, where $x<4$ is a
satisfiable literal. 

\textit{Conflicting clause:}
If all literals in a clause are contradicting, then the clause is said to
be {\em conflicting}. For example, consider a clause $C=(x<4 \vee y>10 \vee z<15)$. 
Let the current partial assignment be $A: x \in [5,13]$ and $y \in [-2,9]$ and $z \in
[17,32]$, then $C$ is a conflicting clause where all literals in $C$ are
contradicting with respect to $A$. 

\textit{Unsatisfiable clause:}
A clause $C$ is said to be {\em unsatisfiable} with respect to a current partial 
assignment $A$ if there exists no satisfiable literal in $C$ or some literals in 
$C$ are unsatisfiable and the rest are contradicting. For example, consider a 
clause $C=(x<4 \vee y>10 \vee z<15)$.  Let the current partial assignment be 
$A: x \in [3,10]$, $y \in [8,10]$ and $z \in [12,20]$, then $C$ is a unsatisfiable 
clause, where $x<4$ and $z<15$ are unsatisfiable literal and $y>10$ is
contradicting literal. 

\textit{Unit clause:}
Clause $C$ is unit if all literals but one is contradicting in $C$. 
For example, consider a clause $C=(x<4 \vee y>10 \vee z<15)$. Let the 
current partial assignment be $A: x \in [5,13]$ and $y \in [-2,9]$ and $z \in
[10,12]$, then $C$ is a unit clause where the unit literal is $z$. 

\Omit {
\textit{Boolean constraint propagation (BCP):} BCP is the repeated 
application of unit rule. This corresponds to computing the greatest fixed point.

\textit{Characteristics of Conflict clause}
\begin{enumerate}
\item A conflict clause must include asserting cuts. An asserting cut is a cut
that contain exactly one node at the current decision level. Assertion cuts yields 
clauses that can be used to derive new information after backtracking.

\item A conflict clause must be UNIT after backtracking. 

%\item There can be multiple cuts and hence multiple UIPs. In other words, there
%can be multiple incomparable reasons for a conflict. But conflict analysis
%procedure choses one that is asserting. 

%\item The conflict clause should be made false by the current partial assignment
%and thus exclude an assignment leading to conflict. 
\end{enumerate}

\textit{Backjumping:}
The backjumping level is defined by the literal of the conflict clause assigned
at the level that is the closest to the conflict one. In other words, the
backjumping level is the level closest to the root (decision level 0)  where the
conflict clause is still unit. If a conflict clause is "globally"  unit, then
the backjumping level is the root of the search tree.
}

%========================

\subsection{Properties of Domain Elements for Learning}
%
\pscmt{TODO: Review by Peter}
An important property of a clause learning SAT solver 
is that each element of the partial assignments domain 
are decomposable into precisely complementable 
elements.  This property of the domain helps to guide 
the model search away from the conflicting region of the 
search space.  

Most numerical abstract domains, such as intervals, octagons, 
polyhedra lacks precise complements, but they can be 
represented as intersections of complementable half-spaces, 
each of which have precise complements.  For example, intervals 
lack complements, but the decomposition of intervals into 
meet-irreducibles have complements.  
\[x \mapsto [2,5] \mathrel{\mathop{\longrightarrow}^{\mathrm{decompose}}} \{\langle x
\geq 2 \rangle \meet \langle x \leq 5 \rangle \} \]
Similarly, the complementation of the octagon in Figure~\ref{octagon} 
can be written as disjunction of:
\[(x<-2) \lor (x>1) \lor (y<-1) \lor (y>2) \lor (x+y>2) \lor (x-y>1) \lor (y-x>3) \lor (-x-y>2)\]

Standard abstract interpretation does not require complementation property.  
Abstract Domain library, such as APRON C library~\cite{apron} is specialized 
for abstract interpretation.  We implement a template-based polyhedra domain 
that returns precise complements by decomposing it into half-spaces.    

\Omit {
The commonly used library for numerical abstract domains  
is the APRON C library~\cite{apron}.  This library is 
used for the static analysis of the numerical variables 
of a program by abstract interpretation. APRON provides a 
C API interface to various abstract domains and libraries 
such as {\em BOX}, {\em OCTAGON}, {\em Convex Polyhedra} and
{\em Linear Equalities} library.  The aim of such analysis is 
to compute invariants over numerical variables in the 
program~\cite{se2011}. 
}
\Omit {
To this end, we implement our own template-based polyhedra domain and interval 
domain which supports complementation operator.  
%For example, the octagon in Figure~\ref{octagon} can be written as a conjunction of:
%\[(x>=-2) \land (x<=1) \land (y>=-1) \land (y<=2) \land (x+y<=2)
%\land (x-y<=1) \land (y-x<=3) \land (-x-y<=2)\] 
\pscmt{That's no valid motivation. We never complement a whole octagon, but just a
  meet-irreducible. It's trivial to do that with APRON. The reason was
a different one: APRON does not support all C operators, e.g. the bitwise
operators.} 
} 

\subsection{Static Analysis Equations for Safety}
Static program analysis based on abstract interpretation~\cite{DBLP:conf/emsoft/Cousot07} 
perform safety analysis by computing fixed point to infer invariants 
over program variables.  However, bounded model checking (BMC) tries to search 
for a counterexample in a bounded execution trace, by symbolically executing a 
program up to a finite bound.  
Similar to BMC, ACDCL searches for a counterexample by solving the formula shown below.
For a set $N$ of program analysis constraints defined over a set 
of constraint variables $Var = \{X_i| i \in N\}$, representing a program
$\mathcal{P}$ and a concrete domain $\powerset(Env)$, the static analysis equation 
for safety of $\mathcal{P}$ is given as follows.  Here, $X_{init}, X_{Err}$ denote the initial valuation 
and the \rmcmt{final valuation} of constraint variables.
\[\varphi = X_{init} \subseteq Env \wedge \underset{p,q \in |\mathcal{P}|}
{\bigwedge} \{ X_p \subseteq post_{(p,q)}(X_q) \} \wedge X_{err} \supset \emptyset \] 
