\section{Motivating Example}
%
\begin{figure}[t]
\scriptsize
\begin{tabular}{l|l|l}
\hline
C program & CFG & Solver Statistics\\
\hline
\begin{lstlisting}[mathescape=true,language=C]
void main() {
 int x,result1=0,result2=0;
 assume(x>=1 && x<= N);
 result1 = 2*x; 
 int state = 1;
 int y=x;
 while(state!=0) {
  switch(state) {
   case 0: state = 1; break;
   case 1: state = 2; break;
   case 2: x--; y++;
   if(x == 0) state = 0;
   else state = 1; break;
  } 
 }
 result2 = y;
 assert(result1 == result2);
}
\end{lstlisting}
&
\begin{minipage}{4.40cm}
\centering
\vspace*{0.3cm}
\scalebox{.45}{\import{figures/}{example.pspdftex}}
%\caption{A control flow graph for P, P' and abstract conflict graph (ACFG) \label{fig:filter}}
\end{minipage}
&
\begin{lstlisting}[mathescape=true,language=C]
                 SAT solver  ACDCL  
restarts           1          0
conflicts  :       5          2
decisions  :       36         2 
propagations :     1365       240  
conflict literals  65         2   
\end{lstlisting}
\\
\hline
\end{tabular}
\caption{C Program, and statistics for decision procedure}
\label{example}
\end{figure}


\begin{figure}[htp]
\centering
\vspace*{0.3cm}
\scalebox{.45}{\import{figures/}{cfg-conflict.pspdftex}}
\caption{A control flow graph for P, P' and abstract conflict graph (ACFG) \label{fig:filter}}
\end{figure}

\Omit {
\lstdefinestyle{mystyle}{
    numbers=left,                    
    numbersep=5pt,                  
    tabsize=2,
    mathescape=true,
    language=C
}
\lstset{style=mystyle}
}
Let us consider a simple safe program $P$ as shown in Figure~\ref{fig:filter}A.
Astr{\'e}e fails to verify the safety using interval and trace-partition domain
due to control-flow join at location $n5$.  Astr{\'e}e requires external hints, provided by 
manually annotating the code with partition directives at $n1$, to prove safety.  
In general, the imprecision is either intended by the tool because such high precision analysis is 
normally not required for runtime error analysis or the imprecision is unavoidable 
due to the complexity of the application under analysis.

On the other hand, ACDCL uses decision and learning to generate the proof 
using simple interval domain. The analysis associates an interval with each location and variable.  
Assuming $bool: c = [0,1]$, ACDCL performs backward propagation 
starting from $n6$ by computing the weakest precondition for every statement 
in $P$ and immediately infers that (\(n4: x = \bot)\), thus proving safety as shown in 
red color in figure~\ref{fig:filter}A.  Now, let us consider program $P'$ in
Figure~\ref{fig:filter}B. Assuming $char: c$, ACDCL performs both forward and
backward propagations (shown in red color in Figure~\ref{fig:filter}B).  
The analysis then perform sequence of decisions starting with $c = [0,255]$ and reaches a 
conflict when the decision is (\(n1:c == [0,0]\)) which is connected to $(n6: \bot)$ 
and suffices to prove safety.  The deductions made during fixed point
iteration are represented by abstract conflict graph shown in Figure~\ref{fig:filter}C. 
Similar to conflict analysis phase in SAT solvers, ACDCL learns (\(n1: c = [1,255]\)), 
that is all error traces must satisfy (\(c \neq 0\)) at n1.  The analysis backtracks 
discarding all assumptions. Interval analysis is run with the learnt constraint and 
can prove safety.  Thus, decisions and clause learning are used to avoid case based reasoning. 
The advantage of ACDCL over propositional SAT solver is that the decision heuristics in 
ACDCL can exploit the program structure by making decisions on interesting program 
variables (for example variables in conditional branches or loop variables). Experimental 
evidence shows that this leads to significantly less number of decisions compared 
to the propositional solver. Compared to classical abstract interpretation 
based tools, ACDCL automatically performs program and property driven trace partitioning 
using decision and clause learning to generate proofs using simpler domain -- thus, it is more 
precise than classical abstract interpretation.  
 
