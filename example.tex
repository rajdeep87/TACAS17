\section{Motivating Example}

\begin{figure}[t]
\scriptsize\centering
\begin{tabular}{c|c|c}
\hline
C program & Control-Flow Graph & Abstract Conflict Graph \\
\hline
\begin{lstlisting}[mathescape=true,language=C]
void foo(int v, 
         _Bool c) 
{
  assume(v>=0 &&
         v<=N);
  int x; 
  if(c)
    x = v;
  else 
    x = -v;
  int z = x * x;
  assert(z>=0);
}
\end{lstlisting}
&
\begin{minipage}{3.7cm}
\centering
%\vspace*{0.3cm}
\scalebox{.52}{\import{figures/}{example.pspdftex}}
%\caption{A control flow graph for P, P' and abstract conflict graph (ACFG) \label{fig:filter}}
\end{minipage}
&
\begin{minipage}{5.6cm}
\centering
\vspace*{0.3cm}
\scalebox{.5}{\import{figures/}{acdl_run.pspdftex}}
%\caption{A control flow graph for P, P' and abstract conflict graph (ACFG) \label{fig:filter}}
\end{minipage}
\\
\hline
\end{tabular}
\caption{\label{fig:example}
C Program, Control-flow Graph (CFG) and Abstract Conflict Graph}
\end{figure}

Let us consider a simple program in Figure~\ref{fig:example}, which performs
multiplication of machine integers and checks that the result is always
positive.  To avoid overflow error, we assume that the input $v$ has an
upper bound~$N$.  The control-flow graph is given in the middle column of
Figure~\ref{fig:example}.

We analyse the program using three different analysers -- {\em bounded 
model checking (BMC)}, {\em abstract interpretation} and {\em ACDCL}.    
%
\Omit {
The SAT solver makes 36 decisions, 1365 Boolean constraint propagations 
and performs 5 conflict analysis with 65 conflict literals.  Moreover, 
the solver performs a restart, presumably to recover from bad decisions.  
}
%
Standard forward interval analysis is too imprecise to verify the
safety of the above program.  The imprecision is due to control-flow join at
node $n4$.  A state-of-the-art abstract interpretation tool,
Astr{\'e}e~\cite{se2011}, requires external hints, provided by manually
annotating the code with partition directives at $n1$, to prove safety.  The
partition directive tells the tool to analyse each program execution path
separately.  To determine safety, Astr{\'e}e requires a product of a trace
partitioning domain and an interval domain or a relational domain
such as octagons.
%
\Omit { In general, the imprecision is either intended by the tool because such
high precision analysis is normally not required for runtime error analysis
or the imprecision is unavoidable due to the complexity of the application
under analysis.
}

\begin{table}[!b]
\begin{center}
{
\begin{tabular}{l|r|r|r|r|r}
\hline
Solver & Decisions & propagations & conflict & conflict literals & restarts \\ \hline
SAT & 233 & 36436 & 162 & 2604 & 2 \\ \hline
ACDCL (Interval) & 1 & 17 & 1 & 1 & 0 \\ \hline
ACDCL (Octagons) & 0 & 7 & 0 & 0 & 0 \\ 
\hline
\end{tabular}
}
\end{center}
\caption{Solver statistics for Example given in Fig.~\ref{fig:example}}
\label{solver}
\end{table}

BMC converts the program into a bit-vector equation and passes that to a SAT
solver.  

The right-hand side of Figure~\ref{fig:example} illustrates an analysis
using ACDCL with the octagon domain (top) and the interval domain (bottom). 
The analysis associates an abstract element with each control location and
variable.  The octagons deduced by ACDCL are sufficient to prove safety
without any decision or clause learning.  However, unlike Ast{\'e}e, an
interval analysis is also sufficient to prove safety.  The intervals
generated by forward analysis in the initial deduction phase are $\langle
x:[-5,5], z:[-25,25] \rangle$.  Clearly, ACDCL cannot prove safety.

Hence, ACDCL makes a decision $c:=[1,1]$ to refine the analysis.  
ACDCL implements several decision heuristics.  In this case, the 
branching variable $c$ is chosen by an {\em ordered} decision 
heuristics that select variables appearing in conditional branches 
first before choosing other program variables. 
The decision constrains the interval of $x:=[0,5]$.  The deductions made 
during fixed point iteration are represented by abstract conflict 
graph in right-hand side of Figure~\ref{fig:example}.  Nodes in the abstract 
conflict graph denotes the corresponding updates to the program 
variables in the CFG.  Interval analysis concludes that $(\text{Error}:\bot)$, that 
is the decision $c:=[1,1]$ leads to a {\em conflict}.  Thus, 
the program is {\em safe} for $c:=[1,1]$.  

A clause learning SAT solver would learn the reason for conflict at
this point and then backtrack to a level such that the learnt clause
is \emph{unit}.  Similar to conflict analysis phase in SAT solvers,
ACDCL learns that (\(n0: c:=[0,0]\)), that is all error traces must
satisfy $(c \neq 1)$ at node $n0$.  The analysis discards all interval
constraints that lead to the conflict and backtrack to decision level
0.  ACDCL then performs interval analysis with the learnt constraint
$(c \neq 1)$ which also leads to a conflict, as shown in right-hand side of
Figure~\ref{ssa}.  The analysis cannot backtrack further and therefore
it terminates proving that the program is safe.  A DPLL-style solver
would perform proof by cases.  However, decision and clause learning
are used to avoid case-based reasoning and prevents enumeration
behaviour.
   
For $N=46000$, table~\ref{solver} shows the statistics from 
MiniSAT~\cite{minisat} solver for analysis using BMC, and 
analysis using ACDCL with interval and octagon domains.  
Compared to a SAT solver, there is a significant reduction 
in the number of decisions, propagations, learnt clauses and 
restarts.  Compared to abstract interpretation, ACDCL does 
not require external hints to proof the program, thus 
automatically performing program and property driven trace 
partitioning to generate proofs using decision and clause learning.  
