\section{Experimental Results}
We have implemented the instantiation of ACDCL on relational domains 
in 2LS verification tool~\cite{2ls}.  The abstract domains used for 
our experiments are intervals, template polyhedra, octagons and equalities.  
We verified a total of 70 benchmarks in ANSI-C, which are derived from 
(a) bit-vector regression category in SV-COMP'16, (c) bit-precise and 
cycle accurate software models of hardware circuits automatically 
generated using v2c~\cite{mtk2016}, (c) controller code generated 
from Simulink (d) several hand-crafted benchmarks for equivalence 
checking and bounded loop analysis.  All benchmarks containing 
bounded loops are completely unrolled before analysis.  We compare 
our tool with a state-of-the-art bounded model checker CBMC and a 
commercial static analysis tool, Astr{\'e}e.  CBMC uses MiniSAT 
solver in the backend.  Astr{\'e}e was configured with interval 
domain, trace-partition domain, and relational domains.  To enable precise 
analysis using Astr{\'e}e, all our benchmarks are manually instrumented with 
partition directives which provides external hint to the tool to guide 
the trace partitioning heuristics.  Usually, such high-precision is not 
needed for static analysis, since it makes the analysis very expensive.  
Without trace partitioning, the analysis using Astr{\'e}e shows high 
degree of imprecision. 

\paragraph {\em Observation of Analysis}

\paragraph {\em \textbf{Decision Heuristics}} We compare the performance of 
different decision heuristic on our benchmarks.  We observe that 
the ordered heuristic outperforms other heuristics on control-intensive 
benchmarks due to its ability to prioritize decisions on variables that 
appears in conditional branches.  For straight-line code and most of 
the bit-vector regression suite, random heuristics performs the best.  
Whereas, the activity based heuristics such as Berkmin heuristic which 
works well in propositional cases performs best for benchmarks that 
encountered the maximum number of conflicts to prove safety, thus allowing 
the heuristics to choose the decison variable from the set of learnt clauses.   

\paragraph {\em \textbf{Clause Learning}}      

  
