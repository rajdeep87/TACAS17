\section{Experimental Results}
We have implemented the instantiation of ACDCL on relational domains
in 2LS verification tool~\cite{2ls}.  We have used the abstract domains
provided by 2LS (intervals, octagons, polyhedra and equalities) for our
experiments. Although the performance of the abstract domain implementations 
in 2LS is not competitive with that of APRON~\cite{apron} library, but 
the domain implementation in 2LS can handle all C operators 
(including bitwise operations) out of the box and support precise
complementation of meet irreducibles, which are necessary for conflict-driven 
learning. Our tool and benchmarks are available online\footnote{http://www.cprover.org/acdcl/}.

We verified a total of 66 benchmarks in ANSI-C, which are derived from 
(a) bit-vector regression category in SV-COMP'16, (c) bit-precise and 
cycle accurate software models of hardware circuits automatically 
generated using v2c~\cite{mtk2016}, (c) controller code generated 
from Simulink (d) several hand-crafted benchmarks for equivalence 
checking and bounded loop analysis.  All bounded loops are completely 
unrolled before analysis.  We compare our tool with a state-of-the-art 
SAT-based bounded model checker CBMC~\cite{cbmc} and a commercial 
static analysis tool, Astr{\'e}e.  CBMC uses MiniSAT solver in the backend.  
Astr{\'e}e was configured with range of abstract domains which includes 
interval, bit-field, congruence, trace partitioning and relational 
domains (octagons, polyhedra, zones, equalities, filter).  Whereas, 
ACDCL was instantiated with a product domain of booleans and template 
polyhedra.   
%
\Omit {
To enable precise analysis using Astr{\'e}e, all our benchmarks are 
manually instrumented with partition directives which provides external 
hint to the tool to guide the trace partitioning heuristics.  Usually, 
such high-precision is not needed for static analysis, since it makes 
the analysis very expensive.  Without trace partitioning, the 
analysis using Astr{\'e}e shows high degree of imprecision. 
}
\paragraph {\textbf{Precision and Efficiency of Analysis}}
Figure~\ref{fig:results} presents a comparison of the analysis 
using CBMC and ACDCL.  Figure~\ref{fig:results}(a) clearly shows that 
the SAT solver made significantly more decisions compared to ACDCL 
for all the benchmarks.  Whereas, the extreme right points below the 
diagonal in Figure~\ref{fig:results}(b) shows the propagations in SAT solver  
is maximum for the benchmarks that exhibit relational behavior.  The 
cumulative statistics corresponding to each phase of the algorithm 
for all 66 benchmarks is presented in Table~\ref{result}.  
Experimental result shows a reduction of at least two orders of 
magnitude in the number of decisions, propagations and conflicts 
compared to analyis using CBMC.    
%
\begin{table}[t]
\begin{center}
{
\begin{tabular}{l|l|l|l|l|l}
\hline
Solver & Decisions & propagations & conflict & conflict literals & restarts \\ \hline
SAT & 27917 & 304031 & 3949 & 33646 & 64 \\ \hline
ACDCL (Product Domain) & 161 & 3130 & 10 & 5 & 0 \\ \hline  
\end{tabular}
}
\end{center}
\caption{Solver statistics}
\label{result}
\end{table}
%
Out of 66 benchmarks, CBMC with MiniSAT backend could only prove 16 benchmarks 
without any restarts.  The solver is restarted in 50 cases to avoid spending 
too much time in branches that do not easily lead to a satisfying assignment or 
stronger clause learning.  In contrast, the analysis using ACDCL solved all
66 benchmarks without any restarts.  This is attributed to intelligent \pscmt{hm...} decision 
heuristics which exploits the high-level structure of the program combined 
with the stronger deduction and clause learning mechanisms aided by the richer 
abstract domains.  On the other hand, Ast{\'e}e is often faster than ACDCL, 
but the analysis using Astr{\'e}e shows a high degree of imprecision.  
Astr{\'e}e reported 15 false alarms out of 41 safe benchmarks. \pscmt{even with trace partitioning?} 

\paragraph {\textbf{Propagation Strategy}}      
Figure~\ref{prop-dec}(a) \pscmt{in Fig 5 you distinguish between safe and unsafe, why not here too?} presents a comparison between {\em forward} propagation 
strategy and {\em multi-way} propagation strategy in ACDCL.  The choice of the
propagation strategy influences the total number of decisions and clause 
learning iterations.  Hence, the propagation strategy has a significant 
influence on the run time which can be visualised in Figure~\ref{prop-dec}(a).  
Unlike the forward propagation, the multi-way strategy may take several 
iterations \pscmt{forward does not need several? - reformulate} to reach the fixed-point, but it significantly reduces the 
total number of decisions and conflicts.  This is attributed to the higher 
precision of the meet irreducibles inferred by the multi-way strategy which 
subsequently aids the decision heuristics to make better decisions. \pscmt{maybe repeat that it is a simultaneous forward/backward iteration...}
\Omit {
We observe that the deductions made during chaotic analysis helps the decision 
heuristics to make better decisions compared to forward analysis.  This is 
attributed to the precision of deductions inferred by the chaotic analysis.         
} 

\paragraph {\textbf{Decision Heuristics}} Figure~\ref{prop-dec}(b) presents a 
visualization of different decision heuristics in ACDCL.  Random heuristic
performs the best for most safe benchmarks and all bit-vector category 
benchmarks.  Whereas, the ordered heuristic performs better for programs 
with conditional branches since it prioritises decisions on meet 
irreducibles that appear in conditional branches over meet irreducibles 
that involve numerical variables.  Whereas, the runtimes for berkmin 
heuristic is slightly greater than the random and ordered heuristic 
for our benchmarks. \pscmt{I don't see ordered in the graph. The differences between Berkmin and random are insignificant.}
\Omit{
Whereas activity-based heuristics such as Berkmin heuristic which 
works well in propositional cases performs best for benchmarks 
that encountered the maximum number of conflicts to prove safety, 
thus allowing the heuristics to choose the decison variable among the set of learnt clauses.   
}

\paragraph {\textbf{Learning}} Learning has a significant influence on
the runtime of ACDCL.  We compare the UIP-based learning technique with 
an analysis that performs chronoligical backtracking without learning. 
The effect of UIP computation allows ACDCL to backtrack non-chronologically 
and guide the model search with a learnt transformer.  However, the analysis 
without learning \pscmt{reformulate} exhibits case-enumeration behavior and thus immediately 
runs out of capacity for most benchmarks. \pscmt{how many, what are the bounds on time and memory?}

%================  
\section{Experiments}
\label{section:experiments}
\Omit {
We performed a number of experiments to demonstrate the utility and
applicability of {\algorithmName}.  All experiments were performed
on an Intel Xeon X5667 at 3\,GHz running Fedora 20 with 64-bit binaries.
Each individual run was limited to 13\,GB 
of memory and 900 seconds of
CPU time, % (with one exception noted below), 
enforced by the operating
system kernel.  We took the \emph{loops} meta-category (143 benchmarks) from the SV-COMP'15 benchmark
set.%
\footnote{\url{http://sv-comp.sosy-lab.org/2015/benchmarks.php}}
%

\subsection{\algorithmName\ Verifies More Programs Than the Algorithms it Simulates}

Table~\ref{tab:results} gives a comparison between \systemName\ running
\algorithmName\ (column 6) and \emph{the same system} running as an incremental
bounded model checker (IBMC) (column 2), incremental $k$-induction (i.e. without invariant inference, column 3)
and as an abstract interpreter (AI) (column 4).
%
 \algorithmName\ is more complete than each of the restricted modes. 
This is not self-evident since it could be much less efficient and,
thus, fail to solve the problems within the given time or memory limits.
%
$k$-induction can solve 60.8\% of the benchmarks, 13 more than IBMC. 
%
32\% of the benchmarks can be solved by abstract interpretation (bugs
are only exposed if they are reachable with 0 loop unwindings).  
%
\algorithmName\ solves 62.9\% of the benchmarks, 
proving 3 more properties than $k$-induction.

%%%%%%%%%%%%%%%%%%%%%%%%%% results table %%%%%%%%%%%%%%%%%%%%%%%%%%%%%%%%%%%%%%%

\begin{table}
\centering
\begin{tabular}{|l|ccc|c|c|c|c|}
\hline
& IBMC & $k$-induction & AI & portfolio & algorithmName &
                                                               CPAchecker & ESBMC \\
\hline
counterexamples & \bf 38 & \bf 38 &  17 & \bf 38 & \bf 38 & 36 &  35 \\
proofs          &    36 &  49 &  30 &  51 &  52 & 59 & \bf 91 \\
false proofs    &     0 &   0 &   0 &   0 &   0 &  2 &  12\\
false alarms    &     2 &   2 &   0 &   2 &   2 &  2 &   0\\
inconclusive    &     0 &   0 &  93 &   0 &   0 &  4 &   2\\
timeout         &    65 &  53 &   3 &  50 &  51 & 38 &   2 \\
memory out      &     2 &   1 &   0 &   2 &   0 &  2 &   1 \\
total runtime   & 17.1h & 13.8h & 0.89h & 13.3h & 13.2h & 10.9h & 0.54h \\
\hline
\end{tabular}~\\[1ex]
\caption{Comparison between algorithmName, the algorithms it subsumes,
the portfolio, and CPAchecker. The rows false alarms and false proofs 
indicate soundness bugs of the tool implementations.
\label{tab:results}}
\end{table}

\subsection{\algorithmName\ is at Least as Good as Their Na\"ive Portfolio}

To show
that \algorithmName\ is more than a mixture of three techniques and
that they strengthen each other, consider column 5 of
Table~\ref{tab:results}.  This gives the results of an ideal portfolio
in which the three restricted techniques are run in parallel on and
the portfolio terminates when the first returns a conclusive result.
Thus the CPU time taken is three times the time taken by the fastest
technique for each benchmark (in practice these could be run in
parallel, giving a lower \emph{wall clock} time).
%
In our setup, \algorithmName\ had a disadvantage as each component of
virtual portfolio had the same %time and 
memory restriction as
{\algorithmName}, thus effectively giving the portfolio three times as much memory.

Still, {\algorithmName} is slightly faster and more
accurate than the portfolio as can be seen in Table \ref{tab:results}.
%
The scatter plot in Figure~\ref{fig:results}a shows the results for
each benchmark:
%
one can observe that {\algorithmName} is up to one order of magnitude
slower on many unsafe benchmarks, which is obviously due to the
additional work of invariant inference that
{\algorithmName} has to perform in contrast to IBMC.
%
However, note that {\algorithmName} is faster than the 
portfolio on some safe and even one unsafe benchmarks.
This suggests that {\algorithmName} is more than the sum of its parts.

\subsection{\algorithmName\ is Comparable with State-of-the-Art Approaches}

We compared our implementation of \algorithmName\ with
CPAchecker%
\footnote{SVCOMP'15 version, http://cpachecker.sosy-lab.org/}%
%
and ESBMC%
\footnote{SVCOMP'15 version, http://www.esbmc.org/}%
, which uses $k$-induction.
%
The results are shown in the last three columns in Table \ref{tab:results}
and in the scatter plot in Figure~\ref{fig:results}b. Additional
results are given in \pponly{the extended version
  \cite{extended-version}}\rronly{Appendix \ref{sec:further}}.
%
In comparison to CPAchecker, the winner of SVCOMP'15,
our prototype of {\algorithmName} is overall a bit slower and proves 
fewer properties (due to more timeouts), but as
Figure~\ref{fig:results}b shows, it significantly outperforms
CPAchecker on most benchmarks.
%
ESBMC exposes fewer bugs, but proves many more properties and is 
significantly faster. However, it has 6 times
more soundness bugs than our implementation.%
\footnote{The two false alarms in our current implementation are due to 
limited support for dynamic memory allocation.}
%
These results show that our prototype implementation of
\algorithmName\ can keep up with state-of-the-art verification tools.

}
%%%%%%%%%%%%%%%%%%%%%%%%%% kiki vs vbs vs CPAchecker %%%%%%%%%%%%%%%%%%%%%%%%%%%
\begin{figure}[t]
\begin{tabular}{@{\hspace{-1.5em}}c@{\hspace{1em}}c}
\begin{tikzpicture}[scale=0.75]
	\begin{loglogaxis} [xmin=.1,xmax=1500, ymin=.1, ymax=1500, xlabel=kIkI (time in seconds),
			ylabel=portfolio (time in seconds), 
			legend style={at={(0.8,0.15)},
			anchor=north,legend columns=-1 },
			]
\addplot [mark size=1pt,only marks,scatter,point meta=explicit symbolic,
	scatter/classes={s={mark=square},u={mark=triangle*,blue}},] 
	table [meta=label] {scatter-kiki-vbs-sas.dat};
	\legend{Safe,Unsafe}
\addplot [domain=.1:1500] {x};
\addplot [red,sharp plot, domain=.1:1500] {900}
          node [below] at (axis cs:10,850) {timeout};
\addplot [red,sharp plot, domain=.1:1500] coordinates{(900,.1) (900,1500)}
          node [left,rotate=90] at (axis cs:700,10) {timeout}
 node [right,black] at (axis cs:10,3) {portfolio faster}
 node [right,black] at (axis cs:1,55) {kIkI faster};
%\addplot [red,sharp plot, update limits=false] coordinates{(900,.1) (900,1500)}
%	node [left] at {axis cs:700,200} {timeout};
\end{loglogaxis}
\end{tikzpicture}
 &
\begin{tikzpicture}[scale=0.75]
	\begin{loglogaxis} [xmin=.1,xmax=1500, ymin=.1, ymax=1500, xlabel=kIkI (time in seconds),
			ylabel=CPAchecker (time in seconds),
			legend style={at={(0.8,0.15)},
			anchor=north,legend columns=-1 },
			]
\addplot [mark size=1pt,only marks,scatter,point meta=explicit symbolic,
	scatter/classes={s={mark=square},u={mark=triangle*,blue}},] 
	table [meta=label] {scatter-kiki-vbs-sas.dat};
	\legend{Safe,Unsafe}
\addplot [domain=.1:1500] {x};
\addplot [red,sharp plot, domain=.1:1500] {900}
          node [below] at (axis cs:10,850) {timeout};
\addplot [red,sharp plot, domain=.1:1500] coordinates{(900,.1) (900,1500)}
          node [left,rotate=90] at (axis cs:700,150) {timeout}
 node [right,black] at (axis cs:10,3) {CPAchecker faster}
 node [right,black] at (axis cs:1,55) {kIkI faster};
\end{loglogaxis}
\end{tikzpicture} \\
(a) & (b)
\end{tabular}
\caption{\label{fig:results}
Runtime Comparison
}
\end{figure}

%%%%%%%%%%%%%%%%%%%%%%%%%%%%%%%%%%%%%%%%%%%%%%%%%%%%%%%%%%%%%%%%%%%%%%%%%%%%%%%%




%================  
