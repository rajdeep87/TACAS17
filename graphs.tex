\section{Experiments}
\label{section:experiments}
\Omit {
We performed a number of experiments to demonstrate the utility and
applicability of {\algorithmName}.  All experiments were performed
on an Intel Xeon X5667 at 3\,GHz running Fedora 20 with 64-bit binaries.
Each individual run was limited to 13\,GB 
of memory and 900 seconds of
CPU time, % (with one exception noted below), 
enforced by the operating
system kernel.  We took the \emph{loops} meta-category (143 benchmarks) from the SV-COMP'15 benchmark
set.%
\footnote{\url{http://sv-comp.sosy-lab.org/2015/benchmarks.php}}
%

\subsection{\algorithmName\ Verifies More Programs Than the Algorithms it Simulates}

Table~\ref{tab:results} gives a comparison between \systemName\ running
\algorithmName\ (column 6) and \emph{the same system} running as an incremental
bounded model checker (IBMC) (column 2), incremental $k$-induction (i.e. without invariant inference, column 3)
and as an abstract interpreter (AI) (column 4).
%
 \algorithmName\ is more complete than each of the restricted modes. 
This is not self-evident since it could be much less efficient and,
thus, fail to solve the problems within the given time or memory limits.
%
$k$-induction can solve 60.8\% of the benchmarks, 13 more than IBMC. 
%
32\% of the benchmarks can be solved by abstract interpretation (bugs
are only exposed if they are reachable with 0 loop unwindings).  
%
\algorithmName\ solves 62.9\% of the benchmarks, 
proving 3 more properties than $k$-induction.

%%%%%%%%%%%%%%%%%%%%%%%%%% results table %%%%%%%%%%%%%%%%%%%%%%%%%%%%%%%%%%%%%%%

\begin{table}
\centering
\begin{tabular}{|l|ccc|c|c|c|c|}
\hline
& IBMC & $k$-induction & AI & portfolio & algorithmName &
                                                               CPAchecker & ESBMC \\
\hline
counterexamples & \bf 38 & \bf 38 &  17 & \bf 38 & \bf 38 & 36 &  35 \\
proofs          &    36 &  49 &  30 &  51 &  52 & 59 & \bf 91 \\
false proofs    &     0 &   0 &   0 &   0 &   0 &  2 &  12\\
false alarms    &     2 &   2 &   0 &   2 &   2 &  2 &   0\\
inconclusive    &     0 &   0 &  93 &   0 &   0 &  4 &   2\\
timeout         &    65 &  53 &   3 &  50 &  51 & 38 &   2 \\
memory out      &     2 &   1 &   0 &   2 &   0 &  2 &   1 \\
total runtime   & 17.1h & 13.8h & 0.89h & 13.3h & 13.2h & 10.9h & 0.54h \\
\hline
\end{tabular}~\\[1ex]
\caption{Comparison between algorithmName, the algorithms it subsumes,
the portfolio, and CPAchecker. The rows false alarms and false proofs 
indicate soundness bugs of the tool implementations.
\label{tab:results}}
\end{table}

\subsection{\algorithmName\ is at Least as Good as Their Na\"ive Portfolio}

To show
that \algorithmName\ is more than a mixture of three techniques and
that they strengthen each other, consider column 5 of
Table~\ref{tab:results}.  This gives the results of an ideal portfolio
in which the three restricted techniques are run in parallel on and
the portfolio terminates when the first returns a conclusive result.
Thus the CPU time taken is three times the time taken by the fastest
technique for each benchmark (in practice these could be run in
parallel, giving a lower \emph{wall clock} time).
%
In our setup, \algorithmName\ had a disadvantage as each component of
virtual portfolio had the same %time and 
memory restriction as
{\algorithmName}, thus effectively giving the portfolio three times as much memory.

Still, {\algorithmName} is slightly faster and more
accurate than the portfolio as can be seen in Table \ref{tab:results}.
%
The scatter plot in Figure~\ref{fig:results}a shows the results for
each benchmark:
%
one can observe that {\algorithmName} is up to one order of magnitude
slower on many unsafe benchmarks, which is obviously due to the
additional work of invariant inference that
{\algorithmName} has to perform in contrast to IBMC.
%
However, note that {\algorithmName} is faster than the 
portfolio on some safe and even one unsafe benchmarks.
This suggests that {\algorithmName} is more than the sum of its parts.

\subsection{\algorithmName\ is Comparable with State-of-the-Art Approaches}

We compared our implementation of \algorithmName\ with
CPAchecker%
\footnote{SVCOMP'15 version, http://cpachecker.sosy-lab.org/}%
%
and ESBMC%
\footnote{SVCOMP'15 version, http://www.esbmc.org/}%
, which uses $k$-induction.
%
The results are shown in the last three columns in Table \ref{tab:results}
and in the scatter plot in Figure~\ref{fig:results}b. Additional
results are given in \pponly{the extended version
  \cite{extended-version}}\rronly{Appendix \ref{sec:further}}.
%
In comparison to CPAchecker, the winner of SVCOMP'15,
our prototype of {\algorithmName} is overall a bit slower and proves 
fewer properties (due to more timeouts), but as
Figure~\ref{fig:results}b shows, it significantly outperforms
CPAchecker on most benchmarks.
%
ESBMC exposes fewer bugs, but proves many more properties and is 
significantly faster. However, it has 6 times
more soundness bugs than our implementation.%
\footnote{The two false alarms in our current implementation are due to 
limited support for dynamic memory allocation.}
%
These results show that our prototype implementation of
\algorithmName\ can keep up with state-of-the-art verification tools.

}
%%%%%%%%%%%%%%%%%%%%%%%%%% kiki vs vbs vs CPAchecker %%%%%%%%%%%%%%%%%%%%%%%%%%%
\begin{figure}[t]
\begin{tabular}{@{\hspace{-1.5em}}c@{\hspace{1em}}c}
\begin{tikzpicture}[scale=0.75]
	\begin{loglogaxis} [xmin=.1,xmax=1500, ymin=.1, ymax=1500, xlabel=kIkI (time in seconds),
			ylabel=portfolio (time in seconds), 
			legend style={at={(0.8,0.15)},
			anchor=north,legend columns=-1 },
			]
\addplot [mark size=1pt,only marks,scatter,point meta=explicit symbolic,
	scatter/classes={s={mark=square},u={mark=triangle*,blue}},] 
	table [meta=label] {scatter-kiki-vbs-sas.dat};
	\legend{Safe,Unsafe}
\addplot [domain=.1:1500] {x};
\addplot [red,sharp plot, domain=.1:1500] {900}
          node [below] at (axis cs:10,850) {timeout};
\addplot [red,sharp plot, domain=.1:1500] coordinates{(900,.1) (900,1500)}
          node [left,rotate=90] at (axis cs:700,10) {timeout}
 node [right,black] at (axis cs:10,3) {portfolio faster}
 node [right,black] at (axis cs:1,55) {kIkI faster};
%\addplot [red,sharp plot, update limits=false] coordinates{(900,.1) (900,1500)}
%	node [left] at {axis cs:700,200} {timeout};
\end{loglogaxis}
\end{tikzpicture}
 &
\begin{tikzpicture}[scale=0.75]
	\begin{loglogaxis} [xmin=.1,xmax=1500, ymin=.1, ymax=1500, xlabel=kIkI (time in seconds),
			ylabel=CPAchecker (time in seconds),
			legend style={at={(0.8,0.15)},
			anchor=north,legend columns=-1 },
			]
\addplot [mark size=1pt,only marks,scatter,point meta=explicit symbolic,
	scatter/classes={s={mark=square},u={mark=triangle*,blue}},] 
	table [meta=label] {scatter-kiki-vbs-sas.dat};
	\legend{Safe,Unsafe}
\addplot [domain=.1:1500] {x};
\addplot [red,sharp plot, domain=.1:1500] {900}
          node [below] at (axis cs:10,850) {timeout};
\addplot [red,sharp plot, domain=.1:1500] coordinates{(900,.1) (900,1500)}
          node [left,rotate=90] at (axis cs:700,150) {timeout}
 node [right,black] at (axis cs:10,3) {CPAchecker faster}
 node [right,black] at (axis cs:1,55) {kIkI faster};
\end{loglogaxis}
\end{tikzpicture} \\
(a) & (b)
\end{tabular}
\caption{\label{fig:results}
Runtime Comparison
}
\end{figure}

%%%%%%%%%%%%%%%%%%%%%%%%%%%%%%%%%%%%%%%%%%%%%%%%%%%%%%%%%%%%%%%%%%%%%%%%%%%%%%%%



