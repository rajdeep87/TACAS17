\section{Introduction}

%%%%%%%%%%%%%%%%%%%%%%% AI %%%%%%%%%%%%%%%%%%%%%%%%
%
Static program analysis based on abstract interpretation~\cite{CC77} has
been widely used to verify of properties of safety-critical systems. 
Standard static analyses~\cite{se2011} aim to compute program invariants as
fixed-points of abstract transformers.  Abstract states are chosen from a
lattice that has meet and join operations.  The meet precisely models set
intersection (or conjunction, taking a logical view), and the join
over-approximates set union (or disjunction).  The over-approximation in the
join operation is a key cause of loss of precision, which yields false
alarms.  Standard means to address false alarms in static analysis are
richer abstract domains or equipping the analysis with more precise set
unions -- which is often very expensive since the analysis exhibits case
enumeration behaviour.

%%%%%%%%%%%%%%%%%%%%%%% ACDCL %%%%%%%%%%%%%%%%%%%%%
%
On the other hand, propositional SAT solvers are known to be highly
effective for solving complex propositional formulae, and are frequently
able to avoid enumerating the exponential number of cases.  The impressive
performance of modern solvers is credited to well-tuned decision heuristics
and clause learning algorithms in modern solvers.  Collectively, these
algorithms are referred to as \emph{Conflict Driven Clause Learning}
(CDCL)~\cite{cdcl}. The idea to lift these algorithms from the Boolean
domain to richer, arithmetic domains suggests itself.

\emph{Abstract Conflict Driven Clause Learning} (ACDCL)~\cite{dhk2013-popl}
is one such generalisation of CDCL.  The partial satisfying assignments in
CDCL algorithm are generalized to be chosen from a given abstract domain. 
The unit rule is the best abstract transformer, Boolean Constraint
Propagation (BCP) computes a greatest fixed-point over the abstract domain,
decision is dual widening and clause learning can be viewed as synthesising
an abstract transformer for negation.  This enables an ACDCL-based analyser
to refine an imprecise analysis and thus prevents enumeration behaviour. 
ACDCL has so far been instantiated as a decision procedure for
floating-point arithmetic~\cite{DBLP:journals/fmsd/BrainDGHK14} and a
program analyser that reasons using interval~\cite{tacas12}.  Hence, the
existing ACDCL instances are limited to bounds checking in numerical
programs.

Evidently, the verification of many programs requires more expressive
domains like octagons or equalities or polyhedra.  These domains capture
interesting properties of programs that intervals can not express.  In this
paper, we identify specific modifications to the CDCL algorithm that are
necessary to lift propositional CDCL to relational abstract domains such as
octagons, zones and equalities over numerical variables.
\Omit { To this end, we
present a characterisation of an abstract model search procedure and a
conflict analysis procedure that lift the abstract deduction transformer and
the conflict analysis procedure to relational lattice structures.
}
