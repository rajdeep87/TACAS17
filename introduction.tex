\section{Introduction}
%
Static program analysis based on abstract 
interpretation~\cite{DBLP:conf/emsoft/Cousot07} has been widely used to
verify certain classes of properties for safety-critical systems.  A 
standard static analyses~\cite{se2011} for program safety aims to compute 
invariants as fixed-points of abstract transformers.  Most lattice structure 
used in static analysis have meet and join operations.  The meet operation 
precisely model conjunction, and the join operation over-approximate 
disjunction. Thus, the precision loss in static analyzers is often 
due to imprecise join operations. The loss in precision is overcome 
by using richer abstract domains or equipping the analysis with disjunction --
which is often very expensive since the analysis exhibit case enumeration 
behavior. 

On the other hand, the combination of precise conjunction in partial assignments 
domain with learning allow Conflict Driven Clause Learning (CDCL)~\cite{cdcl} 
solvers to reason about disjunction without enumerating cases.  
Silva et. al.~\cite{tacas12, sas12, dhk2013-popl} presented 
a formal recipe for applying abstract interpretation~\cite{se2011} to solve the 
satisfiability problem.  ~\cite{sas12} provide a mathematical generalization 
of CDCL algorithm called, {\em Abstract Conflict Driven Clause Learning} (ACDCL), 
in terms of deduction and abduction transformers, fixed-point, interpolation 
and extrapolation operators on lattices.  The partial assignments in the 
CDCL algorithm are an abstract domain, unit rule is the best abstract 
transformer,  Boolean Constraint Propagation (BCP) computes a greatest fixed-point 
over the abstract domain, decision is dual widening and clause learning can be 
viewed as synthesizing an abstract transformer for negation.  Thus, ACDCL 
lifts learning techniques in SAT solver to operate on non-distributive 
lattice -- thus equipping the analysis with negation. This enables the 
analyzer to refine the analysis and prevent enumeration behavior.

\rmcmt{relational domains}

The instantiation of ACDCL has so far been limited to interval domains only.  
In this paper, we identify specific properties of the CDCL algorithm that 
are necessary to lift the propositional skeleton of CDCL to relational 
abstract domains such as template polyhedra domains~\cite{sriram} 
(octagons, zones and equalities).  To this end, we present a characterization 
of abstract model search procedure and conflict analysis procedure that lift 
the abstract deduction transformer and the first-UIP algorithm to 
relational lattice structures. 
