\documentclass[a4paper,conference]{llncs}

\usepackage{colortbl}

\newcommand{\dkcmt}[1]{{\color{red} [{#1}]}}
\newcommand{\rmcmt}[1]{{\color{magenta} [{#1}]}}
\newcommand{\sjcmt}[1]{{\color{blue} [{#1}]}}
\newcommand{\tmcmt}[1]{{\color{orange} [{#1}]}}
\usepackage{xspace}
\usepackage{tikz}
\usetikzlibrary{shapes,arrows,positioning,decorations.markings,chains,fit,shapes.multipart}
\usepackage{import}
\usepackage{times}
\usepackage{graphicx}
\usepackage{cite}
\usepackage{amsfonts}
\usepackage{amssymb}
\usepackage{amsmath}
\usepackage{mathptm}
\usepackage{rotating}
\usepackage{color}
\usepackage{listings}
\usepackage{verbatim}
\usepackage{comment}
\usepackage{alltt}
\usepackage{mdwlist}
\usepackage{booktabs}
\usepackage[algo2e,linesnumbered,ruled,lined]{algorithm2e}
\usepackage{hyperref}
\makeatother

\newcommand{\tool}[1]{\textsc{#1}\xspace}
\newcommand{\hector}{\tool{hector}}
\newcommand{\slec}{\tool{slec}}
\newcommand{\hwcbmcv}{\tool{hw-cbmc}}
\newcommand{\hwcbmcver}{\tool{hw-cbmc 5.4}}
\newcommand{\verifoxver}{\tool{VerifOx 0.1}}
\newcommand{\verifox}{\tool{VerifOx}}
\newcommand{\klee}{\tool{klee}}
\newcommand{\kleever}{\tool{klee 1.0.0}}
\newcommand{\symbiotic}{\tool{symbiotic}}

\newcommand{\Omit}[1]{}
\newcommand{\scs}[1]{\scriptsize{#1}}

%%% Standard text acronyms

\newcommand{\para}[1]{
  \subsubsection*{#1}}
 
%\noindent
%\textbf{#1~}}

\newcommand{\cnf}{\textsc{cnf}\xspace}
\newcommand{\sat}{\textsc{sat}\xspace}
\newcommand{\smt}{\textsc{smt}\xspace}
\newcommand{\dpll}{\textsc{dpll}\xspace}
\newcommand{\fpdpll}{\textsc{fdpll}\xspace}
\newcommand{\cfg}{\textsc{cfg}\xspace}
\newcommand{\cfgs}{\textsc{cfg}s\xspace}
\newcommand{\acfg}{\textsc{acfg}\xspace}
\newcommand{\smpp}{\textsc{smpp}\xspace}
\newcommand{\cegar}{\textsc{cegar}\xspace}
\newcommand{\cdfl}{\textsc{cdfl}\xspace}
\newcommand{\cdflitv}{\textsc{cdfl}$(\itvdom)$\xspace}
\newcommand{\cdcl}{\textsc{cdcl}\xspace}
\newcommand{\cbmc}{\textsc{cbmc}\xspace}
\newcommand{\bmc}{\textsc{bmc}\xspace}
\newcommand{\cdflalgo}{\textsc{\textsf{cdfl}}\xspace}
\newcommand{\ieeefp}{\textsc{\textsf{IEEE 754}}\xspace}


%%%%%%%%%%%%%%%%%%%%%%%%%%%%%%% 
%%% Mathematical Symbols %%%
%%%%%%%%%%%%%%%%%%%%%%%%%%%%%%% 

\renewcommand{\vec}[1]{{\boldsymbol #1}}
\newcommand{\vecv}[2]{{\left(\begin{array}{@{}c@{}} #1 \\ #2\end{array}\right)}}
\newcommand{\qmat}[4]{{\left(\begin{array}{@{}cc@{}} #1 & #2 \\  #3 & #4\end{array}\right)}}
\newcommand{\mat}[1]{{\boldsymbol #1}}

%%% Sets and functions

\newcommand{\powerset}{\ensuremath{\wp}}
\newcommand{\set}[1]{\ensuremath{\left\{#1\right\}}}
\newcommand{\setneg}[1]{\overline{#1}}
\newcommand{\setsep}{\ensuremath{~|~}}
\newcommand{\tuple}[1]{\ensuremath{(#1)}}

\newcommand{\bnf}{\ensuremath{\mathrel{\mathop{::}}=}}
\newcommand{\bnfsep}{~\mid~}

\newcommand{\true}{\mathsf{true}}
\newcommand{\false}{\mathsf{false}}

%%% Lattices
\newcommand{\sle}{\ensuremath{\sqsubseteq}}
\newcommand{\sles}{\ensuremath{\sqsubset}}
\newcommand{\sge}{\ensuremath{\sqsupseteq}}
\newcommand{\sges}{\ensuremath{\sqsupset}}
\newcommand{\join}{\ensuremath{\sqcup}}
\newcommand{\bigjoin}{\ensuremath{\bigsqcup}}
\newcommand{\meet}{\ensuremath{\sqcap}}
\newcommand{\bigmeet}{\ensuremath{\bigsqcap}}


%%% Program model

\newcommand{\program}{\mathit{Prog}}
\newcommand{\assertions}{\mathit{Asserts}}
\newcommand{\assertion}{\mathit{assert}}
\newcommand{\constraints}{\Sigma}
\newcommand{\constraint}{\sigma}
\newcommand{\formula}{\varphi}

\newcommand{\vars}{\mathit{Vars}}
\newcommand{\subvars}{\mathit{V}}
\newcommand{\boolvars}{\mathit{BVars}}
\newcommand{\numvars}{\mathit{NVars}}

\newcommand{\booldomain}{\mathcal{B}}
\newcommand{\reldomain}{\mathcal{RD}}

\newcommand{\numvar}{x}
\newcommand{\numconcval}{\mathit{X}}
\newcommand{\numabsval}{d}
\newcommand{\decisionvar}{\mathit{q}}
\newcommand{\conflictset}{\mathit{C}}
\newcommand{\exclude}{\mathit{exclude}}

\newcommand{\abstrans}[2]{\llbracket #2\rrbracket_{#1}}
\newcommand{\abstransset}{\mathcal{A}}
\newcommand{\abstransel}[1]{\mathit{ded}^{#1}}
\newcommand{\domain}{\mathit{D}}
\newcommand{\subdomain}{\mathit{L}}
\newcommand{\var}{\mathit{s}}
\newcommand{\val}{\mathit{u}}
\newcommand{\concval}{\mathit{concVal}}
\newcommand{\absval}{\mathit{a}}
\newcommand{\newdeductions}{\mathit{v}}
\newcommand{\intervals}{\mathit{Itvs}}
\newcommand{\octagons}{\mathit{Octs}}
\newcommand{\onlynew}{\mathit{onlyNew}}
\newcommand{\aunit}{\mathit{AUnit}}

\newcommand{\abs}{\mathit{abs}}
\newcommand{\trail}{\mathcal{T}}
\newcommand{\reasons}{\mathcal{R}}
\newcommand{\worklist}{\mathit{worklist}}
\newcommand{\initworklist}{\mathit{initWorklist}}
\newcommand{\updateworklist}{\mathit{updateWorklist}}
\newcommand{\makesubdomain}{\mathit{MakeL}}

\newcommand{\decide}{\mathit{decide}}
\newcommand{\deduce}{\mathit{deduce}}
\newcommand{\analyzeconflict}{\mathit{analyzeConflict}}

\newcommand{\propheur}{\mathit{H_P}}
\newcommand{\decheur}{\mathit{H_D}}
\newcommand{\confheur}{\mathit{H_C}}


\newcommand{\cvars}{\mathit{CVar}}
\newcommand{\vals}{\mathit{Val}}
\newcommand{\exps}{\mathit{Exp}}
\newcommand{\expr}{\mathit{exp}}
\newcommand{\bexps}{\mathit{B\!Exp}}
\newcommand{\envs}{\mathit{Env}}
\newcommand{\env}{\varepsilon}
\newcommand{\aenv}{\hat{\varepsilon}}
\newcommand{\assg}{\ensuremath{\mathrel{\mathop:}=}}
\newcommand{\cond}[1]{\ensuremath{[#1]}}
\newcommand{\choice}[1]{\ensuremath{\mathit{choose}\{#1\}}}
\newcommand{\loopstmt}[1]{\ensuremath{\mathit{loop}\{#1\}}}
\newcommand{\nondetstmt}[1]{\ensuremath{\mathit{nondet}\{#1\}}}
\newcommand{\lfp}{\mathsf{lfp}}
\newcommand{\gfp}{\mathsf{gfp}}

\newcommand{\locs}{\mathit{Loc}}
\newcommand{\stmt}{\ensuremath{\mathit{stmt}}}
\newcommand{\err}{\ensuremath{\lightning}}
\newcommand{\init}{\ensuremath{\mathit{init}}}
\newcommand{\reach}{\mathit{Reach}}
\newcommand{\loopfree}{\mathit{Loop-Free}}

\newcommand{\badprog}{\ensuremath{\mathit{Err}}}
\newcommand{\progexec}{\ensuremath{\mathit{Exec}}}
\newcommand{\absbadprog}{\ensuremath{\hat{\mathit{Err}}}}
% \newcommand{\inv}{\ensuremath{\mathit{Inv}}}
\newcommand{\absinv}{\ensuremath{\hat{\mathit{Inv}}}}
\newcommand{\safe}{\ensuremath{\mathit{Safe}}}
\newcommand{\abssafe}{\ensuremath{\hat{\mathit{Safe}}}}

%%% Standard Abstract interpretation 
%%%
\newcommand{\aenvs}{\mathit{A\!Env}}
\newcommand{\post}{\mathit{post}}
\newcommand{\lpost}[1]{\mathit{post}_{#1}}
\newcommand{\abspost}{\mathit{\hat{post}}}
\newcommand{\abslpost}[1]{\mathit{\hat{post}}_{#1}}

\newcommand{\absuawp}{\mathit{\breve{wp}}}


%%% Fixedpoint-DPLL
%%%
\newcommand{\cvals}{\mathit{CVals}}
\newcommand{\avals}{\mathit{AVals}}

\newcommand{\aval}{\hat{\mathit{v}}}
\newcommand{\sem}[2]{\ensuremath{\llbracket #1 \rrbracket_{#2}}}
\newcommand{\rsem}[3]{\ensuremath{\llbracket #1 \rrbracket_{#2}^{#3}}}
\newcommand{\asem}[2]{\ensuremath{\|#1 \|_{#2}}}

\newcommand{\comps}{\mathit{Comp}}
%%\newcommand{\ccomp}[1]{\tilde{#1}}
\newcommand{\ccomp}[1]{\mathord{\sim}{#1}}
\newcommand{\rest}{R}
\newcommand{\makerest}[1]{\mathit{res}(#1)}
\newcommand{\restrictions}{\mathit{Res}}
\newcommand{\lrestrictions}{\mathit{LRes}}
\newcommand{\restrict}[2]{\ensuremath{#1\mathord{\downharpoonright_{#2}}}}
\newcommand{\implabel}{\mathit{st}}
\newcommand{\gimplabel}{\mathit{stg}}

\newcommand{\decs}{\mathit{Dec}}
\newcommand{\dec}{\mathsf{d}}
\newcommand{\imp}{\mathsf{i}}
\newcommand{\emptystack}{\epsilon}

\newcommand{\lit}{\ell}
\newcommand{\lits}{\mathit{Lit}}
\newcommand{\litseq}{L}
\newcommand{\stackres}[1]{\lfloor #1\rfloor}
\newcommand{\litrest}[1]{\langle#1\rangle}
\newcommand{\litres}{\mathit{lit}}

\newcommand{\solver}[2]{#1 ~\|~ #2}

%%% Pseudocode
\newcommand{\pdeduce}{\mathsf{deduce}}
\newcommand{\pdecide}{\mathsf{decide}}
\newcommand{\pmaximal}{\mathsf{maximal}}
\newcommand{\plearn}{\mathsf{learn}}
\newcommand{\pbacktrack}{\mathsf{backtrack}}
\newcommand{\pfail}{\mathsf{fail}}
\newcommand{\psafe}{\mathsf{safe}}

 \newcommand{\fpfont}[1]{\textbf{\sffamily{#1}}}
 \newcommand{\fpfn}[1]{\sffamily{#1}}
 \SetKwSty{fpfont}
 \SetFuncSty{fpfn}

 \SetKwFunction{Decide}{decide}
 \SetKwFunction{TransProp}{tprop}
 \SetKwFunction{UnitProp}{uprop}
 \SetKwFunction{Refines}{refines}
 \SetKwFunction{Deduce}{deduce}
 \SetKwFunction{IsUnit}{is\_unit}
 \SetKwFunction{Propagate}{propagate}
 \SetKwFunction{Invariant}{invariant}
 \SetKwFunction{Atomic}{atomic}
 \SetKwFunction{Learn}{learn}
 \SetKwFunction{Backtrack}{backtrack}
 \SetKwFunction{Clausegen}{clauseGen}
 \SetKwFunction{Invariantgen}{invariantGen}

 \SetKw{Let}{let}

%%% Generalization
\SetKwFunction{Asserting}{asserting}
\SetKwFunction{Cutheur}{cutheur}
\SetKwFunction{Generalise}{generalise}
\SetKwFunction{Analyse}{analyse}
\SetKwFunction{ClauseGeneralise}{clauseGeneralise}
\SetKwFunction{WpGeneralise}{wpGeneralise}
\SetKw{MyCase}{case}
\SetKwFunction{SetReason}{setReason}
\SetKwFunction{PickAssertingCut}{assertingCut}
\newcommand{\implicationgraph}{\mathit{implicationGraph}}

%%% Operators
\newcommand{\wrt}{w.r.t.\ }
% \newcommand{\inp}{\mathit{input}}
\newcommand{\M}{{\cal M}}
\newcommand{\Tr}{{\mathit{tr}}}
\newcommand{\clip}{{\mathit{clip}}}
\newcommand{\decomp}{{\mathit{decomp}}}
\newcommand{\form}{{\mathit{Form}}}
\newcommand{\clausecompl}{\mathit{clcomp}}
\newcommand{\covers}{{\mathit{covers}}}
\newcommand{\cfgdecomps}{{\cal D}_{\mathit{CFG}}}
\newcommand{\cfgrefine}{{\cal R}_{\mathit{CFG}}}
\newcommand{\literals}{{\cal L}}
\newcommand{\clauses}{{\cal C}}
\newcommand{\proofstate}{\mathit{proof}}
\newcommand{\literaldecomp}[1]{\mathit{lits(}#1\mathit{)}}
\newcommand{\transded}{\mathit{trans}}
\newcommand{\unitded}{\mathit{unit}}

\newcommand{\prog}{P}

\renewcommand{\min}{\mathit{min}}
\renewcommand{\max}{\mathit{max}}
\newcommand{\itvdom}{{\mathit{IEnv}}}
\newcommand{\atoms}{\mathit{Atoms}}

\newcommand{\irred}{\mathit{Irred}_\meet}

\newcommand{\implsu}{I_\mathsf{u}}
\newcommand{\implst}{I_\mathsf{t}}

\newcommand{\predec}{\mathit{predec}}
\newcommand{\conflictnode}{\mathit{conflict}}

\newcommand{\dred}[1]{\textcolor{red!60!black}{#1}}
\newcommand{\dgreen}[1]{\textcolor{green!40!black}{#1}}


\input{refs}

\begin{document}

\title{Invariant Generation using ACDL}
\author{}
\institute{}

\maketitle 

\section{Algorithm for Invariant Generation}
\Omit{
We propose a two-solver approach for proving programs with loops using ACDL. 
Let us call these solvers: \em{proof solver} (PS) and \em{invariant solver} (IS).
The purpose of proof solver is to determine the safety of the program in the
presence of the negation of assertion. Whereas, the purpose of invariant solver is 
to determine the inductivenss of the candidate invariant produced by the proof 
solver in the presence of the actual assertion.
}
We present a program verification algorithm, ACDCL-P, inspired 
from the propositional-level CDCL algorithm. ACDCL-P generates a candidate 
invariant following decisions and deductions. The inductiveness of the 
candidate invariants are then checked using an Abstract CDCL solver, ACDCL. ACDCL 
solver returns SAT if the input formula is satisfiable and UNSAT otherwise. 

\section{Communication between ACDCL and ACDCL-P}
The deduction information in the trail corresponding to 
decisions that participate in the invariant can be 
communicated from ACDCL-P to ACDCL.  


\subsection{Highlights of ACDCL-P}
\begin{enumerate}
\item decision heuristics on cond variables enable us to prove many 
programs without actually deciding on the actual program variables. This 
leads to significantly less number of decisions and less number of conflicts 
compared to the SAT solver. Further, decisions on conditional program 
variables can be seen as simulating path-based symbolic execution. But when 
a decision on conditional varaible leads to BOTTOM, ACDCL-P tried to learn 
the reason for failure, and hence can also be seen similar to IMPACT like
algorithm. 
\item Last UIP allows us to learn clauses that has more information in it. 
\item Compared to pure AI based tools, ACDCL-P performs program and property 
driven partitioning with simpler domains to prove programs with relational 
properties (branch7). AI based tool would require richer domains like 
relational (equality) domain to prove such programs. 
\item Compared to cdfpl, our tool proves more programs correctly and works on 
non-numeric programs as well.
\item Chaotic propagations help ACDCL-P to generate stronger deductions and reach 
fixpoint early in propagation stage.
\item Compared to cdfpl, acdl solver is agnostic to abstract domain. cdfpl
stores the value of variables (from interval domain) in every decision 
and deduction stage in a container "value". So, cdfpl keeps track of literals 
and their intervals at every stage which makes it difficult to extend it to other 
richer domains. However, ACDCL-P treat literals as meet irreducibles. Hence, 
ACDCL-P can be easily extended to other richer domains.    
\item ACDCL-P handles loops by automatic invariant generation.
\end{enumerate}

\subsection{Difference between cdfpl and ACDCL-P}
\begin{enumerate}
\item cdfpl stores deductions from conflict clause into the trail. It does not 
store the deductions from the AI phase into the propagation trail. Hence, each 
entry in the trail has a corresponding reason (conflict clause responsible for 
deducing the literal in the entry). Thus, it is easier to compute the UIP in
cdfpl since it retains only the clause based deduction information in the trail. 
Thus, it can perform resolution proof easily to compute the UIP. The resolution 
proof in cdfpl is implemented in a way that it iterates through the marked
portion of the trail (contradicted portion in the current decision segment of
the trail with respect to the conflict clause) to find the corresponding clause 
(predecessor clause) for each marked entry in the DL segment of the trail 
(Note that the predecessor clause is the one which inferred this literal in the 
trail). The predecessor clause is then iterated to find if a literal in the 
clause is contradicted in the current decision level or previous decision level 
and accordingly the learnt clause and the backtrack level is updated. 
  
\item Whereas, ACDCL-P stores the deduction information from the AI phase into the 
trail. In such case, the entries in the trail does not have a corresponding
reason clause. Thus, it is not possible to perform resolution proof to find the 
UIP since the propagation trail does not store such information.   
\end{enumerate}

\textbf{Lemma 1:}Normalization is an equivalent transformation. \\ 
\textbf{Lemma 2:}Normalizing a learnt clause is UNIT preserving. \\
Proof: Prove that the above lemma holds both for monotonic ($x \leq 2 \land x \leq 1 
\land x \leq 0)$ and non-monotonic decisions $(x \leq 20 \land x \geq 10 \land
x \leq 15 \land x \leq 13)$. \\
\textbf{Lemma 3:}Normalizing a learnt clause does not affect the backtrack
level. \\
\begin{enumerate}
\item Step 1: Construct the learnt clause, L=${\neg{d1} \lor \neg{d2} \lor \neg{UIP}}$
\item Step 2: Normalize the learnt clause without UIP, normalize L'=${\neg{d1} \lor \neg{d2}}$
\item Step 3: Add the UIP to the normalized learnt clause, L\_final=L' $\lor \neg{UIP}$
\item Step 4: Backtrack to blevel=$max\{d1, d2\}$
\item Step 5: Let $val$ be the current partial assignment after backtracking
\item Step 6: Normalize the current partial assignment, $val$
\item Step 7: Check unit rule for (L\_final, val) -- The learnt clause should now be UNIT !
\end{enumerate}

\subsection{Some ideas}
\begin{enumerate}
\item Can we infer the smallest UNSAT core when we reach BOTTOM 
as the potential candidate for Invariant. If the UNSAT core for 
a proof is indeed an invariant, then it will be UNSAT core for other proofs 
as well. 
\end{enumerate}

\begin{algorithm2e}[t]
\DontPrintSemicolon
\SetKwFunction{decide}{Decide}
\SetKwFunction{deduce}{deduce}
\SetKwFunction{backtrack}{backtrack}
\SetKwFunction{print}{print}
\SetKwFunction{return}{return}
\SetKwFunction{nondet}{nondet}
\SetKwFunction{run}{run}
\SetKwFunction{learn}{learn}
\SetKwFunction{continue}{continue}
\SetKwFunction{assign}{assign}
\SetKwRepeat{Do}{do}{while}
%\SetKwFunction{assume}{assume}
%\SetKwFunction{isf}{isFeasible}
\SetKwData{safe}{safe}
\SetKwData{unsafe}{unsafe}
\SetKwData{sat}{sat}
\SetKwData{unsat}{unsat}
\SetKwData{unknown}{unknown}
\SetKwData{true}{true}
\SetKwInOut{Input}{input}
\SetKwInOut{Output}{output}
\SetKw{KwNot}{not}
\begin{small}
\Input{Formula $\mathcal{F}$}
\Output{The status (\sat or \unsat or \unknown)}
%\tcc{The initial state}
%$ACDL \leftarrow \tuple{\neg(assert(c)),loopselect=nondet}$ \nllabel{algline:initPS}\;
\assign result=\deduce(F)\;
\uIf{$result == \perp$} {
  \return \unsat \;
}
\uElse
{
  \uIf{is\_complete()} {
    \return \sat \;
  }
}
\While{$true$} 
{
  \assign decision\_variable = decision\_heuristics($F$)\;
  \assign decision\_container = decision\_container $\cup$ decision\_variable\; 
  \uIf{decision\_variable == NULL}
  {
    return \unknown\; 
  }
  \uElse
  {
    \assign result=\deduce(F)\;
    \uIf{result == UNKNOWN}
    {
      \uIf{is\_complete()} {
        \return \sat \;
      }
      \uElse 
      {
        \continue\;
      }
    }
    \uElse
    {
      \Do{result == PASS} {
        \uIf{$\neg$ conflict\_analysis()} {
          return \unsat \;  
        }
        \assign result=\deduce(F)\;
      }
    }
  }
}
\end{small}
\caption{ACDCL Solver \label{Alg:acdl}}
\end{algorithm2e}


\begin{algorithm2e}[t]
\DontPrintSemicolon
\SetKwFunction{decide}{Decide}
\SetKwFunction{deduce}{deduce}
\SetKwFunction{backtrack}{backtrack}
\SetKwFunction{print}{print}
\SetKwFunction{return}{return}
\SetKwFunction{nondet}{nondet}
\SetKwFunction{run}{run}
\SetKwFunction{learn}{learn}
\SetKwFunction{continue}{continue}
\SetKwFunction{assign}{assign}
\SetKwRepeat{Do}{do}{while}
%\SetKwFunction{assume}{assume}
%\SetKwFunction{isf}{isFeasible}
\SetKwData{safe}{safe}
\SetKwData{unsafe}{unsafe}
\SetKwData{unknown}{unknown}
\SetKwData{true}{true}
\SetKwInOut{Input}{input}
\SetKwInOut{Output}{output}
\SetKw{KwNot}{not}
\begin{small}
\Input{Program $\mathcal{P}$ with properties specified with $assert(c)$}
\Output{The status (\safe or \unsafe) and a counterexample if \unsafe}
%\tcc{The initial state}
$ACDCL-P \leftarrow \tuple{\neg(assert(c)),loopselect=nondet}$ \nllabel{algline:initPS}\;
\assign result=\deduce(P)\;
\uIf{$result == \perp$} {
  \return \safe \;
}
\uElse
{
  \uIf{is\_complete()} {
    \return \unsafe \;
  }
}
\While{$true$} 
{
  \assign decision\_variable = decision\_heuristics($P$)\;
  \assign decision\_container = decision\_container $\cup$ decision\_variable\; 
  \uIf{decision\_variable == NULL}
  {
    return \unknown\; 
  }
  \uElse
  {
    \assign result=\deduce(P)\;
    \uIf{result == UNKNOWN}
    {
      \uIf{is\_complete()} {
        \return \unsafe \;
      }
      \uElse 
      {
        \continue\;
      }
    }
    \uElse
    {
      \Do{result == PASS} {
        \uIf {loop\_variable $\in$ decision\_container} {
          \assign $Inv=(\bigcap_{i=1}^{n} d_{i} \rightarrow
              \bigcap_{j=1}^{m} l_{j})$ where $d_i$ $\in$ \{non\_loop\_variable 
            $\cup$ loop\_variable\}, $l_j$ $\in$ \{loop\_variable\} \nllabel{algline:inv}\;
            \assign Inv\_formula = $Inv$ $\wedge$ $P$ $\wedge$ $Inv'$, where $Inv'$
            is the invariant after the application of transition system\;
        $inv\_result \leftarrow$ run($ACDCL$, $Inv\_formula$) \; 
        \uIf{inv\_result == UNSAT \nllabel{algline:statusIS}} {
          \learn $Inv$ \;
        }
        }
        \uIf{$\neg$ conflict\_analysis()} {
          return \safe \;  
        }
        \assign result=\deduce(P)\;
      }
    }
  }
}
\end{small}
\caption{ACDCL-P Verifier with Invariant Checking\label{Alg:verifox}}
\end{algorithm2e}

\begin{algorithm2e}[t]
\DontPrintSemicolon
\SetKwFunction{decide}{Decide}
\SetKwFunction{deduce}{deduce}
\SetKwFunction{backtrack}{backtrack}
\SetKwFunction{print}{print}
\SetKwFunction{return}{return}
\SetKwFunction{nondet}{nondet}
\SetKwFunction{run}{run}
\SetKwFunction{learn}{learn}
\SetKwFunction{continue}{continue}
\SetKwFunction{assign}{assign}
\SetKwRepeat{Do}{do}{while}
%\SetKwFunction{assume}{assume}
%\SetKwFunction{isf}{isFeasible}
\SetKwData{safe}{safe}
\SetKwData{unsafe}{unsafe}
\SetKwData{true}{true}
\SetKwData{false}{false}
\SetKwInOut{Input}{input}
\SetKwInOut{Output}{output}
\SetKw{KwNot}{not}
\begin{small}
\Input{A counterexample trace $t$ of Program $P$}
\Output{The status (\true) if $t$ is not spurious and $t$ is complete 
or (\false) if $t$ is spurious or not complete}
\uIf{loopselect $\in$ t} {
   \uIf{loopselect==false} {
      \uIf{gamma\_complete(t)} {
        return \true \;
      }
      \uElse 
      {
        return \false \;
      }
   }
   \uElse {
     return \false \;
   }
}
\end{small}
\caption{Spurious Counterexample Checking \label{Alg:spurious-check}}
\end{algorithm2e}

\begin{algorithm2e}[t]
\DontPrintSemicolon
\SetKwFunction{decide}{Decide}
\SetKwFunction{deduce}{deduce}
\SetKwFunction{backtrack}{backtrack}
\SetKwFunction{print}{print}
\SetKwFunction{return}{return}
\SetKwFunction{nondet}{nondet}
\SetKwFunction{run}{run}
\SetKwFunction{learn}{learn}
\SetKwFunction{continue}{continue}
\SetKwFunction{assign}{assign}
\SetKwRepeat{Do}{do}{while}
\SetKwData{safe}{safe}
\SetKwData{unsafe}{unsafe}
\SetKwData{true}{true}
\SetKwData{false}{false}
\SetKwInOut{Input}{input}
\SetKwInOut{Output}{output}
\SetKw{KwNot}{not}
\begin{small}
\Input{A counterexample trace $t$ of Program $P$}
\Output{The status (\true) if $t$ is complete, else \false if $t$ is not complete} 
\For{elements $e \in t$} {
  \uIf{$e$ is non-singleton} {
    return \false \;  
  }
}
return \true \;  
\end{small}
\caption{Gamma Completeness Check \label{Alg:gamma-complete}}
\end{algorithm2e}

\end{document}
