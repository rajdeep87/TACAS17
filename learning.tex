\section{Abstract Conflict Analysis}
A conflict analysis in a SAT solver can be seen as abductive 
reasoning which underapproximate a set of models that do not satisfy a 
formula~\cite{sas12,dhk2013-popl}.  
\Omit {
where a singleton assignment $c$ is replaced by a partial 
assignment that is sufficient to infer $c$.  Thus, a conflict 
analysis is a generalization of an under-approximation of a set of 
countermodels, where a countermodel is a 
set of structures that do not satisfy a formula 
}
Since relational abstract domains have more complex structure than a partial 
assignments domain or interval domain, so finding a generalized reason for 
conflict through UIP computation~\cite{uip} is non-trivial. 

\paragraph {\textbf{Abstract Conflict Graph}}
The first-UIP algorithm in SAT solver works on a data structure called 
{\em implication graph} that computes a cut that suffices to produce a 
conflict.  The trail in ACDCL implicitly encodes a graph structure called 
{\em abstract conflict graph}, which records the dependencies between 
deductions made during abstract model search phase.  Nodes in the graph 
represent elements of the trail $\mathcal{T}$.  Nodes can be either 
{\em decision} node or {\em deduction} node. Edges can be extracted from
the transformers in the reason trail.  Incoming arrows to node $n$ 
indicate that the predecessors of $n$ are sufficient to deduce $n$.  
A decision node has no incoming edges.  
For example, consider a formula 
$\varphi:= (x=5 \; \wedge \; y=x \; \wedge \; y=y+x \; \wedge \; y \leq 0)$ with current 
abstract valuation $v:= (x=5)$.  Figure~\ref{conflict} shows a 
snapshot of an abstract conflict graph that 
stores the deductions obtained from abstract model search using octagon domain.  
The last deduction conflict with the constraint $(y \leq 0)$.  

\paragraph {\textbf{Lifting UIP to relational domains}}
An UIP is a special node in the abstract conflict graph such that 
any path from the last decision node to the conflict node must pass 
through it.  Unlike a SAT solver, a UIP for relational domain may 
involve meet irreducibles that contains the same variables.  The 
first-UIP is a unique node closest to the conflict, and the last UIP 
is the decision node itself.  Computing UIPs ensures asserting
cuts~\cite{cdcl,DBLP:journals/fmsd/BrainDGHK14}, that is, it 
yields clauses that generates new deduction after backtracking.  
Every cut in the graph is a reason for conflict that can be 
used in learning.  An abstract first-UIP algorithm~\cite{DBLP:journals/fmsd/BrainDGHK14} 
traverses the trail $\mathcal{T}$ starting from the conflict node and 
computes a cut that suffices to produce a conflict. 
\Omit {
contradict with the transformer stored in $\mathcal{R}[\bot]$.  
}
For our example, there exists multiple incomaparable reasons for conflict,
marked as {\em cut0, cut1}, in figure~\ref{conflict}.  Here, cut0 is the first UIP.  
Choosing cut0 yields a learnt clause 
$(x+y<15 \vee -x-y<-15 \vee -x+y<5 \vee x-y<-5 \vee y-10<0 \vee -y+10<0)$, 
which is obtained by negating the reason for conflict.  
%
\begin{figure}[t]
%\caption{An example of Octagon}\label{octagon}
\scalebox{.65}{\import{figures/}{conflict_graph.pspdftex}}
\caption{\label{conflict} Abstract Conflict Graph \pscmt{What is the
    semantics of ``;''?}}
\end{figure} 
%    
\paragraph{\textbf{Learning in Relational Lattice}}
Learning in propositional solver yields a asserting clause~\cite{cdcl} 
that express the negation of the conflict reasons.  However, we model 
learning in relational domains as learning a transformer which infers 
new meet irreducibles using the abstract unit rule called 
{\em abstract unit transformer (AUnit)}.  We add $AUnit$ to the set of 
transformers \rmcmt{give symbol} used during model search. $AUnit$ is 
a generalization of propositional unit rule for numerical domains.  For 
an abstract lattice $A$ with complementable meet irreducibles and a set 
of meet irreducibles $C \subseteq A$ of $\varphi$ such that $\bigsqcap C$ does not 
satisfy $\varphi$, $AUnit_C: A \rightarrow A$ is formally defined as follows. 
\[ AUnit(a) =
 \begin{cases}
  \bot       & \quad \text{if } a \sqsubseteq \meet{C} \qquad \qquad \qquad
  \qquad \qquad \qquad \quad (1)\\
  \bar{t}    & \quad \text{if } t \in C \; \text{and} \; \forall t' \in C
  \setminus \{t\}. a  \sqsubseteq t' \qquad \quad \; \; (2) \\
  \bar{t}    & \quad \text{if } t \in C \; \text{and} \; \forall t' \in C \setminus \{t\}. a
  \not\sqsubseteq t', \; \text{then} \\ 
             & \quad \forall a' \subseteq decomp(a) \;  \text{and} \; \forall a'' \in a \setminus \{a'\}, \\ 
             & \quad \meet{a''} \sqsubseteq r' \; \text{and} \; a'
             \not\sqsubseteq t   \qquad \qquad \qquad \qquad \; \; (3) \\
  \top & \quad \text{otherwise} \qquad \qquad \qquad \qquad \qquad \qquad \quad (4) \\
 \end{cases}
\]
$AUnit$ returns $\bot$ if $C$ is conflicting following rule (1).  Rule (2) and
(3) of $AUnit$ infers a valid meet irreducible, which implies that $C$ is unit. 
However, rule (3) is domain-specific and only applies to numerical abstract
domains.  For example, let $C = \langle x+z \geq 10,  w \geq 0 \rangle$ 
be a conflicting element that does not satisfy $\varphi$ and let $a= \langle x+y \geq 2, y+z \geq 5,
x+z \leq 15, w \geq 1, w \leq 1 \rangle$.  Applying rule (3) of $AUnit_C$, we get a 
decomposition of $a$ into $a'$ and $a''$ where 
$a'= \langle x+y \geq 2, y+z \geq 5, x+z \leq 15 \rangle \not\sqsubseteq (x+z \geq 10)$, and 
$a''= \langle w \geq 1, w \leq 1 \rangle \sqsubseteq (w \geq 0)$.  Then, $AUnit_C(a) = (x+z < 10)$, 
which constrains the bound of the octagonal constraint.  
Let us consider another example, where $C = \langle x \geq 2, x \leq 5, y \leq 7 \rangle$
and let $a = \langle x \geq 3, x \leq 4, y \geq 5, y \leq 6 \rangle$, then $AUnit_C(a) = \bot$ using
the rule (1), since $a \sqsubseteq C$.  Now, let $a' = \langle x \geq 3, x \leq 4 \rangle$, 
then $AUnit_C(a') = (y > 7)$ using rule (2), since $a' \sqsubseteq (x \geq 2)$
and $a' \sqsubseteq (x \leq 5)$.  However, if none of the above rule holds true, $AUnit$ 
returns $\top$. 
 
\paragraph {\textbf{Backjumping}}
A backjumping procedure undo all the assignments in the trail up to 
a decision level that restores the solver to a consistent state 
(non-conflicting).  The backjumping level is defined by the literal 
of the conflict clause that is closest to the root (decision level 0) 
where the conflict clause is still unit. If a conflict clause is 
globally unit, then the backjumping level is the root of the search tree.

The abstract clause learning and backjumping procedures in the abstract 
conflict graph is stated in terms of the state of ACDCL solver as follows. 
\[AbsLearn: \quad  (\mathcal{E},S) \rightarrow (\mathcal{E},S \wedge
\mathcal{L}) \quad \text{if} \; \mathcal{L} \notin S \; \textrm{and}
\; (S \wedge \mathcal{L}) \; \text{is not UNSAT} \]
\[AbsBackjump: \quad (\mathcal{E}_1(m,s)\mathcal{E}_2,S) \rightarrow
(\mathcal{E}_1,S) \quad \text{if} \; (\mathcal{E}_1,S) \; \text{is not in conflict} \]   

\Omit {
\subsection{Clause Learning in Abstract Lattice}
Conflict graph for Intervals
Conflict graph for Octagons

\textit{Characteristics of Conflict clause}
\begin{enumerate}
\item A conflict clause must include asserting cuts. An asserting cut is a cut
that contain exactly one node at the current decision level. Assertion cuts yields 
clauses that can be used to derive new information after backtracking.

\item A conflict clause must be UNIT after backtracking. 

\item There can be multiple cuts and hence multiple UIPs. In other words, there
can be multiple incomparable reasons for a conflict. But conflict analysis
procedure choses one that is asserting. 

\item The conflict clause should be made false by the current partial assignment
and thus exclude an assignment leading to conflict. 
\end{enumerate}

1.DPLL style -- chronological backtracking \\
2. CDCL style -- non-chronological backtracking \\
  a. first-uip \\
  b. last-uip \\

**********************************************
\subsection{Lifting First UIP to Octagon domain}
**********************************************
unit-ness guarantee in octagon domain:
  Popped stmt: y23=1+y21
   Abstract value:
   D1: y23-y21 < 2 &&
   D2: y23+y21 > 0 &&
   D3: y21 < 1
   After backtracking, apply unit rule 
   y23 > 1 -- deduction from unit rule
   Value inconsistent !!
 
 Note: 
 1> cannot make reasoning at literal level for relational domain because literals are dependant on each other. As soon as literals denote relation between first-order variables, the pure reasoning on boolean skeleton is not sufficient. 
 
2> Intervals are orthogonal half spaces similar to booleans. 

3> After backtracking, the application of unit rule is done as follows for relational domains:
  Pass the learnt clause (as statement) and the abstrat value to the domain to make deductions.
}
