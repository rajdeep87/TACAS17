\subsection{Literals and Clauses for Abstract CDCL}
%
\textit{Literal:} A Literal is a meet irreducible which specify that 
for certain program variable, there is a certain bound. An example of 
literal in interval domain is $x \in [0,5]$. An example of octagonal 
literal is given by $((x-y \leq 0) \wedge (y-x \leq 0))$.

Let I and J be two interval literals of the form $x \in [l, u]$ with $l$ 
as lower bound and $u$ as upper bound. We say $I \leq J$ or {\em (I leq J)} 
iff $(I.u \leq J.u \wedge I.l \geq J.l)$.  Whereas, {\em I disjoint J} iff 
$(I.u < J.l \vee  I.l > J.u)$. 

Similarly, let $O_1$ and $O_2$ be two octagonal literals of the form 
$x-y \leq c$.  Then, $O_1 \leq O_2$ {\em ($O_1$ leq $O_2$)} iff 
$O_1.c \leq O_2.c$.  Two octagons are disjoint ({\em $O_1$ disjoint $O_2$}) 
iff $O_1:= (x-y < c)$ and $O_2:= (x-y > c)$.  

\textit{Clause:} A clause is a disjunction of one or more meet irreducibles. 
An example clause is given by $(x \geq 0 \vee y \geq 5 \vee y+z \leq 10)$.

Let $x$, $x'$ be the literals in clause $C$ and current partial assignment 
$A$ respectively.  We call a literal $x \in C$ to be {\em satisfiable} with 
respect to the literal $x' \in A$ iff $(x \leq x')$.  We call $x$ to be  
{\em unsatisfiable} with respect to $x'$ iff $!(x \leq x') \wedge$ $!$(x disjoint $x'$).
A literal is said to be {\em contradicting} iff $!(x \leq x') \wedge$ (x disjoint $x'$).

Similarly, a clause may be catagorized into four classes -- satisfiable,
unsatisfiable, conflicting, unit. \\ 
\textit{Satisfiable clause:}
A clause $C$ is said to be {\em satisfiable} with respect to a current partial 
assignment $A$ if at least one literal in $C$ is satisfiable. For example, 
consider a clause $C=(x<4 \vee y>10)$ and let the current partial assignment 
be $A: x \in [0,3]$. Then $C$ is a satisfiable clause, where $x<4$ is a
satisfiable literal. 

\textit{Conflicting clause:}
If all literals in a clause are contradicting, then the clause is said to
be {\em conflicting}. For example, consider a clause $C=(x<4 \vee y>10 \vee z<15)$. 
Let the current partial assignment be $A: x \in [5,13]$ and $y \in [-2,9]$ and $z \in
[17,32]$, then $C$ is a conflicting clause where all literals in $C$ are
contradicting with respect to $A$. 

\textit{Unsatisfiable clause:}
A clause $C$ is said to be {\em unsatisfiable} with respect to a current partial 
assignment $A$ if there exists no satisfiable literal in $C$ or some literals in 
$C$ are unsatisfiable and the rest are contradicting. For example, consider a 
clause $C=(x<4 \vee y>10 \vee z<15)$.  Let the current partial assignment be 
$A: x \in [3,10]$, $y \in [8,10]$ and $z \in [12,20]$, then $C$ is a unsatisfiable 
clause, where $x<4$ and $z<15$ are unsatisfiable literal and $y>10$ is
contradicting literal. 

\textit{Unit clause:}
Clause $C$ is unit if all literals but one is contradicting in $C$. 
For example, consider a clause $C=(x<4 \vee y>10 \vee z<15)$. Let the 
current partial assignment be $A: x \in [5,13]$ and $y \in [-2,9]$ and $z \in
[10,12]$, then $C$ is a unit clause where the unit literal is $z$. 

\Omit {
\textit{Boolean constraint propagation (BCP):} BCP is the repeated 
application of unit rule. This corresponds to computing the greatest fixed point.

\textit{Characteristics of Conflict clause}
\begin{enumerate}
\item A conflict clause must include asserting cuts. An asserting cut is a cut
that contain exactly one node at the current decision level. Assertion cuts yields 
clauses that can be used to derive new information after backtracking.

\item A conflict clause must be UNIT after backtracking. 

%\item There can be multiple cuts and hence multiple UIPs. In other words, there
%can be multiple incomparable reasons for a conflict. But conflict analysis
%procedure choses one that is asserting. 

%\item The conflict clause should be made false by the current partial assignment
%and thus exclude an assignment leading to conflict. 
\end{enumerate}

\textit{Backjumping:}
The backjumping level is defined by the literal of the conflict clause assigned
at the level that is the closest to the conflict one. In other words, the
backjumping level is the level closest to the root (decision level 0)  where the
conflict clause is still unit. If a conflict clause is "globally"  unit, then
the backjumping level is the root of the search tree.
}
