\documentclass[a4paper]{llncs}

\usepackage{import}
\usepackage{times}
\usepackage{microtype}
\usepackage{array}
\usepackage{graphicx,wrapfig}
\usepackage{cite}
\usepackage{tikz}
\usetikzlibrary{plotmarks}
\usepackage{pgfplotstable}
\usepackage{filecontents}
\usepackage{pgfplots}
\usepackage{amsfonts}
\usepackage{amssymb}
\usepackage{amsmath}
\usepackage{stmaryrd}
%\usepackage{mathptm}
\usepackage{color}
\usepackage{listings}
\usepackage{verbatim}
\usepackage{comment}
\usepackage{alltt}
\usepackage{psfrag}
\usepackage{epsfig}
\usepackage{wasysym} 
%\usepackage{subfigure}
\usepackage{paralist}
\usepackage{dingbat}
\usepackage[algo2e,linesnumbered,ruled,lined]{algorithm2e}
\usepackage{hyperref}
\usepackage[subnum]{cases}
\makeatother

\newcommand{\extendedonly}[1]{}
\newcommand{\paperonly}[1]{#1}

\newcommand{\tool}[1]{\textsc{#1}\xspace}
\newcommand{\cbmcv}{\tool{cbmc 5.0}}
\newcommand{\Omit}[1]{}
\newcommand{\mydef}[1]{\begin{definition}#1\end{definition}}
\newcommand{\rmcmt}[1]{{\color{magenta}{#1}}}%#1
\newcommand{\pscmt}[1]{{\color{blue}{#1}}}%#1
\newcommand{\dkcmt}[1]{{\color{red!70!black}{#1}}}%#1

%%% Standard text acronyms

\newcommand{\para}[1]{
  \subsubsection*{#1}}
 
%\noindent
%\textbf{#1~}}

\newcommand{\cnf}{\textsc{cnf}\xspace}
\newcommand{\sat}{\textsc{sat}\xspace}
\newcommand{\smt}{\textsc{smt}\xspace}
\newcommand{\dpll}{\textsc{dpll}\xspace}
\newcommand{\fpdpll}{\textsc{fdpll}\xspace}
\newcommand{\cfg}{\textsc{cfg}\xspace}
\newcommand{\cfgs}{\textsc{cfg}s\xspace}
\newcommand{\acfg}{\textsc{acfg}\xspace}
\newcommand{\smpp}{\textsc{smpp}\xspace}
\newcommand{\cegar}{\textsc{cegar}\xspace}
\newcommand{\cdfl}{\textsc{cdfl}\xspace}
\newcommand{\cdflitv}{\textsc{cdfl}$(\itvdom)$\xspace}
\newcommand{\cdcl}{\textsc{cdcl}\xspace}
\newcommand{\cbmc}{\textsc{cbmc}\xspace}
\newcommand{\bmc}{\textsc{bmc}\xspace}
\newcommand{\cdflalgo}{\textsc{\textsf{cdfl}}\xspace}
\newcommand{\ieeefp}{\textsc{\textsf{IEEE 754}}\xspace}


%%%%%%%%%%%%%%%%%%%%%%%%%%%%%%% 
%%% Mathematical Symbols %%%
%%%%%%%%%%%%%%%%%%%%%%%%%%%%%%% 

\renewcommand{\vec}[1]{{\boldsymbol #1}}
\newcommand{\vecv}[2]{{\left(\begin{array}{@{}c@{}} #1 \\ #2\end{array}\right)}}
\newcommand{\qmat}[4]{{\left(\begin{array}{@{}cc@{}} #1 & #2 \\  #3 & #4\end{array}\right)}}
\newcommand{\mat}[1]{{\boldsymbol #1}}

%%% Sets and functions

\newcommand{\powerset}{\ensuremath{\wp}}
\newcommand{\set}[1]{\ensuremath{\left\{#1\right\}}}
\newcommand{\setneg}[1]{\overline{#1}}
\newcommand{\setsep}{\ensuremath{~|~}}
\newcommand{\tuple}[1]{\ensuremath{(#1)}}

\newcommand{\bnf}{\ensuremath{\mathrel{\mathop{::}}=}}
\newcommand{\bnfsep}{~\mid~}

\newcommand{\true}{\mathsf{true}}
\newcommand{\false}{\mathsf{false}}

%%% Lattices
\newcommand{\sle}{\ensuremath{\sqsubseteq}}
\newcommand{\sles}{\ensuremath{\sqsubset}}
\newcommand{\sge}{\ensuremath{\sqsupseteq}}
\newcommand{\sges}{\ensuremath{\sqsupset}}
\newcommand{\join}{\ensuremath{\sqcup}}
\newcommand{\bigjoin}{\ensuremath{\bigsqcup}}
\newcommand{\meet}{\ensuremath{\sqcap}}
\newcommand{\bigmeet}{\ensuremath{\bigsqcap}}


%%% Program model

\newcommand{\program}{\mathit{Prog}}
\newcommand{\assertions}{\mathit{Asserts}}
\newcommand{\assertion}{\mathit{assert}}
\newcommand{\constraints}{\Sigma}
\newcommand{\constraint}{\sigma}
\newcommand{\formula}{\varphi}

\newcommand{\vars}{\mathit{Vars}}
\newcommand{\subvars}{\mathit{V}}
\newcommand{\boolvars}{\mathit{BVars}}
\newcommand{\numvars}{\mathit{NVars}}

\newcommand{\booldomain}{\mathcal{B}}
\newcommand{\reldomain}{\mathcal{RD}}

\newcommand{\numvar}{x}
\newcommand{\numconcval}{\mathit{X}}
\newcommand{\numabsval}{d}
\newcommand{\decisionvar}{\mathit{q}}
\newcommand{\conflictset}{\mathit{C}}
\newcommand{\exclude}{\mathit{exclude}}

\newcommand{\abstrans}[2]{\llbracket #2\rrbracket_{#1}}
\newcommand{\abstransset}{\mathcal{A}}
\newcommand{\abstransel}[1]{\mathit{ded}^{#1}}
\newcommand{\domain}{\mathit{D}}
\newcommand{\subdomain}{\mathit{L}}
\newcommand{\var}{\mathit{s}}
\newcommand{\val}{\mathit{u}}
\newcommand{\concval}{\mathit{concVal}}
\newcommand{\absval}{\mathit{a}}
\newcommand{\newdeductions}{\mathit{v}}
\newcommand{\intervals}{\mathit{Itvs}}
\newcommand{\octagons}{\mathit{Octs}}
\newcommand{\onlynew}{\mathit{onlyNew}}
\newcommand{\aunit}{\mathit{AUnit}}

\newcommand{\abs}{\mathit{abs}}
\newcommand{\trail}{\mathcal{T}}
\newcommand{\reasons}{\mathcal{R}}
\newcommand{\worklist}{\mathit{worklist}}
\newcommand{\initworklist}{\mathit{initWorklist}}
\newcommand{\updateworklist}{\mathit{updateWorklist}}
\newcommand{\makesubdomain}{\mathit{MakeL}}

\newcommand{\decide}{\mathit{decide}}
\newcommand{\deduce}{\mathit{deduce}}
\newcommand{\analyzeconflict}{\mathit{analyzeConflict}}

\newcommand{\propheur}{\mathit{H_P}}
\newcommand{\decheur}{\mathit{H_D}}
\newcommand{\confheur}{\mathit{H_C}}


\newcommand{\cvars}{\mathit{CVar}}
\newcommand{\vals}{\mathit{Val}}
\newcommand{\exps}{\mathit{Exp}}
\newcommand{\expr}{\mathit{exp}}
\newcommand{\bexps}{\mathit{B\!Exp}}
\newcommand{\envs}{\mathit{Env}}
\newcommand{\env}{\varepsilon}
\newcommand{\aenv}{\hat{\varepsilon}}
\newcommand{\assg}{\ensuremath{\mathrel{\mathop:}=}}
\newcommand{\cond}[1]{\ensuremath{[#1]}}
\newcommand{\choice}[1]{\ensuremath{\mathit{choose}\{#1\}}}
\newcommand{\loopstmt}[1]{\ensuremath{\mathit{loop}\{#1\}}}
\newcommand{\nondetstmt}[1]{\ensuremath{\mathit{nondet}\{#1\}}}
\newcommand{\lfp}{\mathsf{lfp}}
\newcommand{\gfp}{\mathsf{gfp}}

\newcommand{\locs}{\mathit{Loc}}
\newcommand{\stmt}{\ensuremath{\mathit{stmt}}}
\newcommand{\err}{\ensuremath{\lightning}}
\newcommand{\init}{\ensuremath{\mathit{init}}}
\newcommand{\reach}{\mathit{Reach}}
\newcommand{\loopfree}{\mathit{Loop-Free}}

\newcommand{\badprog}{\ensuremath{\mathit{Err}}}
\newcommand{\progexec}{\ensuremath{\mathit{Exec}}}
\newcommand{\absbadprog}{\ensuremath{\hat{\mathit{Err}}}}
% \newcommand{\inv}{\ensuremath{\mathit{Inv}}}
\newcommand{\absinv}{\ensuremath{\hat{\mathit{Inv}}}}
\newcommand{\safe}{\ensuremath{\mathit{Safe}}}
\newcommand{\abssafe}{\ensuremath{\hat{\mathit{Safe}}}}

%%% Standard Abstract interpretation 
%%%
\newcommand{\aenvs}{\mathit{A\!Env}}
\newcommand{\post}{\mathit{post}}
\newcommand{\lpost}[1]{\mathit{post}_{#1}}
\newcommand{\abspost}{\mathit{\hat{post}}}
\newcommand{\abslpost}[1]{\mathit{\hat{post}}_{#1}}

\newcommand{\absuawp}{\mathit{\breve{wp}}}


%%% Fixedpoint-DPLL
%%%
\newcommand{\cvals}{\mathit{CVals}}
\newcommand{\avals}{\mathit{AVals}}

\newcommand{\aval}{\hat{\mathit{v}}}
\newcommand{\sem}[2]{\ensuremath{\llbracket #1 \rrbracket_{#2}}}
\newcommand{\rsem}[3]{\ensuremath{\llbracket #1 \rrbracket_{#2}^{#3}}}
\newcommand{\asem}[2]{\ensuremath{\|#1 \|_{#2}}}

\newcommand{\comps}{\mathit{Comp}}
%%\newcommand{\ccomp}[1]{\tilde{#1}}
\newcommand{\ccomp}[1]{\mathord{\sim}{#1}}
\newcommand{\rest}{R}
\newcommand{\makerest}[1]{\mathit{res}(#1)}
\newcommand{\restrictions}{\mathit{Res}}
\newcommand{\lrestrictions}{\mathit{LRes}}
\newcommand{\restrict}[2]{\ensuremath{#1\mathord{\downharpoonright_{#2}}}}
\newcommand{\implabel}{\mathit{st}}
\newcommand{\gimplabel}{\mathit{stg}}

\newcommand{\decs}{\mathit{Dec}}
\newcommand{\dec}{\mathsf{d}}
\newcommand{\imp}{\mathsf{i}}
\newcommand{\emptystack}{\epsilon}

\newcommand{\lit}{\ell}
\newcommand{\lits}{\mathit{Lit}}
\newcommand{\litseq}{L}
\newcommand{\stackres}[1]{\lfloor #1\rfloor}
\newcommand{\litrest}[1]{\langle#1\rangle}
\newcommand{\litres}{\mathit{lit}}

\newcommand{\solver}[2]{#1 ~\|~ #2}

%%% Pseudocode
\newcommand{\pdeduce}{\mathsf{deduce}}
\newcommand{\pdecide}{\mathsf{decide}}
\newcommand{\pmaximal}{\mathsf{maximal}}
\newcommand{\plearn}{\mathsf{learn}}
\newcommand{\pbacktrack}{\mathsf{backtrack}}
\newcommand{\pfail}{\mathsf{fail}}
\newcommand{\psafe}{\mathsf{safe}}

 \newcommand{\fpfont}[1]{\textbf{\sffamily{#1}}}
 \newcommand{\fpfn}[1]{\sffamily{#1}}
 \SetKwSty{fpfont}
 \SetFuncSty{fpfn}

 \SetKwFunction{Decide}{decide}
 \SetKwFunction{TransProp}{tprop}
 \SetKwFunction{UnitProp}{uprop}
 \SetKwFunction{Refines}{refines}
 \SetKwFunction{Deduce}{deduce}
 \SetKwFunction{IsUnit}{is\_unit}
 \SetKwFunction{Propagate}{propagate}
 \SetKwFunction{Invariant}{invariant}
 \SetKwFunction{Atomic}{atomic}
 \SetKwFunction{Learn}{learn}
 \SetKwFunction{Backtrack}{backtrack}
 \SetKwFunction{Clausegen}{clauseGen}
 \SetKwFunction{Invariantgen}{invariantGen}

 \SetKw{Let}{let}

%%% Generalization
\SetKwFunction{Asserting}{asserting}
\SetKwFunction{Cutheur}{cutheur}
\SetKwFunction{Generalise}{generalise}
\SetKwFunction{Analyse}{analyse}
\SetKwFunction{ClauseGeneralise}{clauseGeneralise}
\SetKwFunction{WpGeneralise}{wpGeneralise}
\SetKw{MyCase}{case}
\SetKwFunction{SetReason}{setReason}
\SetKwFunction{PickAssertingCut}{assertingCut}
\newcommand{\implicationgraph}{\mathit{implicationGraph}}

%%% Operators
\newcommand{\wrt}{w.r.t.\ }
% \newcommand{\inp}{\mathit{input}}
\newcommand{\M}{{\cal M}}
\newcommand{\Tr}{{\mathit{tr}}}
\newcommand{\clip}{{\mathit{clip}}}
\newcommand{\decomp}{{\mathit{decomp}}}
\newcommand{\form}{{\mathit{Form}}}
\newcommand{\clausecompl}{\mathit{clcomp}}
\newcommand{\covers}{{\mathit{covers}}}
\newcommand{\cfgdecomps}{{\cal D}_{\mathit{CFG}}}
\newcommand{\cfgrefine}{{\cal R}_{\mathit{CFG}}}
\newcommand{\literals}{{\cal L}}
\newcommand{\clauses}{{\cal C}}
\newcommand{\proofstate}{\mathit{proof}}
\newcommand{\literaldecomp}[1]{\mathit{lits(}#1\mathit{)}}
\newcommand{\transded}{\mathit{trans}}
\newcommand{\unitded}{\mathit{unit}}

\newcommand{\prog}{P}

\renewcommand{\min}{\mathit{min}}
\renewcommand{\max}{\mathit{max}}
\newcommand{\itvdom}{{\mathit{IEnv}}}
\newcommand{\atoms}{\mathit{Atoms}}

\newcommand{\irred}{\mathit{Irred}_\meet}

\newcommand{\implsu}{I_\mathsf{u}}
\newcommand{\implst}{I_\mathsf{t}}

\newcommand{\predec}{\mathit{predec}}
\newcommand{\conflictnode}{\mathit{conflict}}

\newcommand{\dred}[1]{\textcolor{red!60!black}{#1}}
\newcommand{\dgreen}[1]{\textcolor{green!40!black}{#1}}



\begin{document}

\title{Model Search in Relational Domains}

\author{Rajdeep Mukherjee\inst{1} \and Peter Schrammel\inst{2} \and
Daniel Kroening\inst{1} \and Tom Melham\inst{1}}

\institute{University of Oxford, UK \and University of Sussex, UK \\
%\email{\{rajdeep.mukherjee,kroening,tom.melham\}@cs.ox.ac.uk},
%\email{p.schrammel@sussex.ac.uk}
}

\maketitle

%===============================================================================
\begin{abstract}
%===============================================================================
%
\emph{Abstract Conflict-Driven Clause Learning} (ACDCL) is a generalisation
of the well-known CDCL algorithm that is used in SAT solvers.
%
\Omit {  ACDCL alternates between a model search, which
performs over-approximate deduction with constraint propagation, and a
conflict analysis, which performs under-approximate abduction with heuristic
choice.
}
%
ACDCL has been successfully used with non-relational abstract domains such
as intervals to improve the performance of a floating-point decision
procedure.  Whereas the propositional reasoning in SAT 
solvers can be lifted easily from the Boolean lattice to other 
non-relational domains, application to relational domains is 
challenging because of the relational properties of deductions.
%
In this paper, we present the non-trivial liftings of the model search and
conflict analysis algorithms in SAT solver for relational abstract domains.  We have
implemented these algorithms inside the 2LS verification tool and
instantiated our analyser over a template polyhedra domain to check safety
properties of C programs.  We~evaluate the performance of our analyser in
comparison with program verifiers that use propositional-level solvers and
a static analyser based on abstract interpretation.
%
\Omit {
on a set of benchmarks drawn from bit-vector regression 
in SV-COMP'16, bit-precise software models of hardware 
circuits auto-generated from v2c tool, and several 
bounds checking benchmarks 
}
We observe a reduction of two orders of magnitude in the number of 
decisions and conflict analyses compared to a SAT 
solver and higher precision compared to an abstract interpreter. 
\Omit{
along with stronger deductions and learnt clauses aided by the richer
abstract domains.
}
%
\end{abstract}

%===============================================================================
\section{Introduction}
%
In this paper, we identify specific properties of the CDCL algorithm that 
are necessary to lift the propositional skeleton of CDCL to other richer 
lattice structures.

Most lattice structure used in static analysis have meet and join operations 
but lack negations. The meet operation precisely model conjunction, and
the join operation over-approximate disjunction. Thus, the precision loss in 
static analyzers is often due to imprecise join operations. The loss in precision is
overcome by using richer abstract domains or equipping the analysis with disjunction --
which is often very expensive since the analysis exhibit case enumeration behavior. 
 
On the other hand, the combination of precise conjunction in partial assignments domain 
with learning allow CDCL solvers to reason about disjunction without 
enumerating cases. Learning can be viewed as synthesizing an abstract transformer for 
negation. ACDCL lifts learning techniques in SAT solver to operate on non-distributive 
lattice -- thus equipping the analysis with negation. This enables the analyzer to refine 
the analysis and prevent enumeration behavior.

\rmcmt{Our lifting of CDCL is generic for any abstract domains.  We used
octagons, equalities and intervals for our experiments.}

%===============================================================================

%===============================================================================
\paragraph{Contributions}
In this paper, we make the following contributions.
\begin{compactenum}
\item We present a novel program analysis technique that 
lifts the CDCL algorithm to relational numerical abstract domains.   
We have implemented our technique in 2LS verification tool.  To 
the best of our knowledge, this is the first instantiation of CDCL 
algorithm with relational abstract domains. 

\item We present an abstract model search algorithm that strictly 
generalizes the Boolean Constraint Propagation (BCP) procedure to
compute a fixed point over a relational domain.  Our abstract 
conflict analysis algorithm present a non-trivial lifting of the 
Unique Implication Point (UIP) procedure to relational abstract 
domains in order to learn a generalized reason for conflicts.  
 
\item  Experiments on a set of benchmarks drawn from the bitvector 
category in SV-COMP as well as numerical programs and a few 
control-intensive benchmarks from hardware domain show that 
the strict generalization of CDCL algorithm effectively combines 
the efficiency of abstract domains while being path-sensitive and 
bit-precise at the same time.  
\end{compactenum}

%===============================================================================

%===============================================================================
\section{Motivating Example}

\begin{figure}[t]
\centering
\begin{tabular}{c|c|c}
\hline
C program & Control-Flow Graph & Abstract Conflict Graph \\
\hline
\scriptsize
\begin{lstlisting}[mathescape=true,language=C]
void foo(int v, 
         _Bool c) 
{
  assume(v>=0 &&
         v<=N);
  int x; 
  if(c)
    x = v;
  else 
    x = -v;
  int z = x * x;
  assert(z>=0);
}
\end{lstlisting}
&
\begin{minipage}{3.7cm}
\centering
%\vspace*{0.3cm}
\scalebox{.52}{\import{figures/}{example.pspdftex}}
%\caption{A control flow graph for P, P' and abstract conflict graph (ACFG) \label{fig:filter}}
\end{minipage}
&
\begin{minipage}{5.6cm}
\centering
\vspace*{0.3cm}
\scalebox{.5}{\import{figures/}{acdl_run.pspdftex}}
%\caption{A control flow graph for P, P' and abstract conflict graph (ACFG) \label{fig:filter}}
\end{minipage}
\\
\hline
\end{tabular}
\caption{\label{fig:example}
C Program, Control-flow Graph (CFG) and Abstract Conflict Graph}
\end{figure}

Let us consider the simple program in Figure~\ref{fig:example}, which
squares a machine integer and checks that the result is always positive.  To
avoid overflow, we assume that the input $v$ has an upper bound~$N$.  The
control-flow graph is given in the middle column of
Figure~\ref{fig:example}.  We apply three different analysis techniques:
1)~{\em abstract interpretation},
1)~{\em bounded model checking (BMC)} and
3)~{\em ACDCL}.

A standard forward interval analysis is too imprecise to verify the safety
of the above program.  The imprecision is owed to the control-flow join at
node~$n4$.  A state-of-the-art abstract interpretation tool,
Astr{\'e}e~\cite{se2011}, requires external hints, provided by manually
annotating the code with partition directives at $n1$, to prove safety.  The
partition directive tells the tool to analyse each program execution path
separately.  To determine safety, Astr{\'e}e requires a product of a trace
partitioning domain and an interval domain or a relational domain such as
octagons.
%
\Omit { In general, the imprecision is either intended by the tool because such
high precision analysis is normally not required for runtime error analysis
or the imprecision is unavoidable due to the complexity of the application
under analysis.
}

\begin{table}[!b]
\begin{center}
{
\begin{tabular}{l|r|r|r|r|r}
\hline
Solver & Decisions & propagations & conflict & conflict literals & restarts \\ \hline
MiniSat & 233 & 36436 & 162 & 2604 & 2 \\ \hline
ACDCL (Interval) & 1 & 17 & 1 & 1 & 0 \\ \hline
ACDCL (Octagons) & 0 & 7 & 0 & 0 & 0 \\ 
\hline
\end{tabular}
}
\end{center}
\caption{Solver statistics for Example given in Fig.~\ref{fig:example}}
\label{solver}
\end{table}

BMC converts the program into a bit-vector equation and passes that to a
CDCL-based SAT solver.  We will contast the analysis performed by CDCL with
that of ACDCL.  The right-hand side of Figure~\ref{fig:example} illustrates
an analysis using ACDCL with the octagon domain (top) and the interval
domain (bottom). The analysis associates an abstract element with each
control location and variable.  The octagons deduced by ACDCL are sufficient
to prove safety without any decision or clause learning, as in the case of
Ast{\'e}e.

The benefit of ACDCL on this example becomes apparent using the interval
domain.  The intervals generated by forward analysis in the initial
deduction phase are $\langle x:[-5,5], z:[-25,25] \rangle$.  Clearly, these
do not prove safety.  Hence, ACDCL makes a heuristic decision to refine the
analysis.  Assume we decide that $c$ is $[1,1]$.  Using deduction, ACDCL
determines that the interval of $x$ is then constrained to $[0,5]$.  The
deductions made during fixed point iteration are represented by an abstract
conflict graph, given in right-hand side of Figure~\ref{fig:example}.  Nodes
in the abstract conflict graph denote updates to the program variables in
the CFG.  The interval analysis concludes that $(\text{Error}:\bot)$, that
is, the decision $c:=[1,1]$ leads to a {\em conflict}.  Thus, the program is
{\em safe} for the case $c:=[1,1]$.

At this point, a clause learning propositional SAT solver learns a reason
for the conflict and then backtracks to a level such that the learnt clause
is \emph{unit}.  Similarly, ACDCL learns that \(n0: c:=[0,0]\), that is, all
error traces must satisfy $(c \neq 1)$ at node~$n0$.  The analysis discards
all interval constraints that lead to the conflict and backtracks to
decision level~0.  ACDCL then performs interval analysis with the learnt
constraint $(c \neq 1)$, which also leads to a conflict, as shown in
right-hand side of Figure~\ref{ssa}.  The analysis cannot backtrack further
and therefore it terminates, proving that the program is safe.  Thus,
decision and clause learning are used to avoid case-based reasoning and
prevent enumeration behaviour.
   
Table~\ref{solver} compares statistics for the MiniSAT~\cite{minisat}
solver for $N=46000$ for the analysis using BMC, and the analysis using
ACDCL with the interval and octagon domains.  Compared to the propositional
SAT solver, there is a significant reduction in the number of decisions,
propagations, learnt clauses and restarts.  Compared to abstract
interpretation, ACDCL does not require external hints to prove the program,
thus automatically performing program and property-driven trace partitioning
to generate proofs using heuristic decisions and clause learning.

%===============================================================================

%===============================================================================
\section{Program and Abstract Domains}
In this section, we define a notion of program analysis constraints, concrete
domain and abstract domain. \\
\Omit {
\subsection{Programs:}
%
A {\em program} $\mathcal{P}$ is formally defined as follows.
\[
\begin{array}[t]{@{}lll}
\mathcal{P} & {:}{:}{=} & \mathit{Procedure} \\
\mathit{Procedure} & {:}{:}{=} & \mathit{Statement} \mid \mathit{Procedure} \\
\mathit{Statement} & {:}{:}{=} & CStatement \mid \mathit{function(Var_1,\dots,Var_n)} \\
\mathit{CStatement} & {:}{:}{=} & x {:}{=} exp \mid \mathit{ITE}(b, s_1,s_2) \mid \mathit{s_1;s_2} \mid loop\{s\} \\
\end{array}
\]
Consider sets of expressions $Exp$ and Boolean expressions $BExp$
over variables $Var$ of $\mathcal{P}$.  The variables in $Var$ 
can take numeric values in $Val$.  A procedure is denoted as 
$\mathit{function(Var_1,\dots,Var_n)}$.  A $\mathit{CStatement}$ 
is an assignment, conditional, sequential concatenation or a loop. \\

\textbf{Control-flow Graph:} A CFG is a triple $(Loc, E, lbl)$, 
where $Loc$ is the set of locations with an unique start 
location $(init)$ and error location $(err)$, $E$ is the set 
of control-flow edges that are labelled with the set 
$lbl \in Statement$.  For the purpose of illustration, we 
assume that all the procedures are inlined.  
}
\textbf{Program Analysis Constraints:}  A program $\mathcal{P}$ is represented 
as set of equations called, {\em program analysis constraints}.  The 
program analysis constraints, as shown below, is an assignment, 
or boolean expressions $BExp$ over program variables $Var$ or
a non-deterministic choice (ternary experssion) between statements $s_1$ and
$s_2$.  All the bounded loops are completely unrolled and represented using 
following constraints. Similarly, an assertion is of type $BExp$.  
\[c::= x:= exp | BExp | (exp) ? s_1, s_2\]
For example, figure~\ref{swssa} presents the program analysis constraints for 
a C program in Static Single Assignment (SSA) form.  

\pscmt{we should give the (simplified) SSA for an example here}

\subsection{Concrete Semantics}
\pscmt{not sure that adds anything for TACAS}
The {\em concrete} domain is a lattice 
of concrete environments, $Env = Var \rightarrow Val$, and is 
defined by $CDom = (Env, \sle_{C}, \join_{C}, \meet_{C})$.
A transformer, $post_{stmt}$, of concrete domain defines 
the effect of a statement $stmt$ on the concrete domain, 
$post_{stmt} : \powerset(Env) \rightarrow \powerset(Env)$.  

A state in concrete domain is a tuple $\langle l, \sigma \rangle$, 
where $l$ is a location and $\sigma \in Env$.  A trace is a sequence 
of states $(l_0, \sigma_0), \ldots (l_n, \sigma_n)$ such that for all 
$0 \leq i \leq n$, there exists a cfg egde $(l_{i}, l_{i+1})$ 
if $\sigma_{i+1} \in post_{i}(\sigma_{i})$. 

\subsection{Template Polyhedra Abstract Domain}

\pscmt{TODO} 

\paragraph{Interval Abstract Domain} \pscmt{TODO: simplify}
The {\em Interval} abstract domain is a lattice 
$ItvDom = (ItvElm, \sle_{I}, \join_{I}, \meet_{I})$, where
$ItvElm: (Var \rightarrow Itv) \cup \{\bot\}$, and $Itv$ is 
the set of intervals of type $[l,u]$ over numeric data 
type with $l \leq u$. The least element is $\bot$ and the 
greatest element is $\top$ which maps all variables to their
minimum ($min$) and maximum ($max$) values.  An interval 
$\langle x, [min, v] \rangle$ is written as $x \leq v$.  The 
partial order $\sle_{I}$ over elements in the set $Itv$ is 
given by $I_1 \sle_{I} I_2$ if $I_2$ contains $I_1$.
A join $(\join_{I})$ of two intervals $\langle x_1 \rightarrow [l_1, u_1], 
x_1 \rightarrow [l_2, u_2] \rangle$ is an interval 
$\langle x_1 \rightarrow [min(l_1, l_2), max(u_1, u_2)]$.
A meet $(\meet_{I})$ of two intervals $\langle x_1 \rightarrow [l_1, u_1], 
x_1 \rightarrow [l_2, u_2] \rangle$ is an interval 
$\langle x_1 \rightarrow [max(l_1, l_2), min(u_1, u_2)] \rangle$.
The galois connection between the concrete environment and intervals is 
given as follows.
\[\alpha(x) = \{[inf(x), sup(x)] | x \in Var\} \qquad   \alpha(\emptyset) = \bot \]
\[\gamma[p,q] = \{c \in \mathbb{Z} | p \leq c \leq q\} \qquad \gamma(\bot) = \emptyset \]
For a program with $N$ variables, there are a total of 
$2*N$ intervals. The intervals restrict the set of values 
taken by program variables at each program point, starting 
from $[-\infty, \infty]$.   

\paragraph{Octagon Abstract Domain} \pscmt{TODO: simplify}
The {\em Octagon} abstract domain is a lattice 
$OctDom = (OctElm, \sle_{O}, \join_{O}, \meet_{O})$, where
$OctElm: (Var \times Var \rightarrow (\mathbb{R} \cup \{\infty\})) \cup \bot$. 
The least element is $\bot$ that contains all unsatisfiable 
set of inequalities and the greatest element is $\top$ which 
maps the bounds of all octagonal inequalities to $\infty$. 
The partial order $\sle_{O}$ over elements in the set $OctElm$ is 
given by $O_1 \sle_{O} O_2$ iff the bounds of each inequalities in $0_1$ 
is included (by $\leq$ order) in the bounds of corresponding inequalities 
in $O_2$, that is, the octagons are ordered by the inclusion relations.

The join $(\join_{O})$ of two octagons, $\langle (v_i \times v_j \rightarrow N_1),
(v_i \times v_j \rightarrow N_2) \rangle$, is not necessarily an octagon 
and is computed by taking piece-wise maximum of bounds of corresponding 
octagonal inequalities, $\langle (v_i \times v_j \rightarrow max(N_1, N_2) \rangle$.
However, the meet $(\meet_{O})$ of two octagons, $\langle (v_i \times v_j \rightarrow N_1),
(v_i \times v_j \rightarrow N_2) \rangle$, is always an octagon and is 
computed by taking piece-wise minimum of bounds of corresponding 
octagonal inequalities, $\langle (v_i \times v_j \rightarrow min(N_1, N_2)$.  A closed 
octagon is the smallest octagon following the partial order $\sle_{O}$, among all 
the octagons that abstract the same concrete values.

The octagon domain is a relational abstract domain that permits $2n^2$ 
linear inequalities between $n$ program variables.  The octagonal 
inequalities are of two types: binary or unary inequalities as shown below.
\[Binary: \pm v_i \pm v_j \leq a, v_i \neq v_j \qquad Unary: v_i \leq b, \{a, b\} \in \mathbb{R} \cup \infty \]  
An element in octagon domain, $OctElm$, is a conjunction of such 
inequalities and is called an octagon.  Figure~\ref{octagon} shows an 
example octagon and its associated inequalities.   
%
\begin{wrapfigure}{r}{5.5cm}
\caption{An example of Octagon}\label{octagon}
\scalebox{.65}{\import{figures/}{octagons.pspdftex}}
\end{wrapfigure} 
%
For building program analyzers using octagon domains, the domain also 
provides few operators like {\em widening ($\nabla$)} and {\em closure ($*$)}.  
The widening operator is used to accelerate convergence for loops in the program
and has a quadratic complexity in the number of variables.  If the bound of an 
octagonal inequality increases every iteration, then $\nabla$ sets the bounds 
to $\infty$.  However, the closure operator is often used to reduce the degree 
of over-approximation resulting from the join operation. Closure always produces 
unique octagonal inequalities, but it is very expensive to compute because of 
cubic complexity in the number of program variables. An example of closure
operation is shown below.
\[((x-y \leq 4) \wedge (y-z \leq 5))^* \implies ((x-y \leq 4) \wedge (y-z \leq 5) \wedge (x-z \leq 9)) \]  
 
The abstract transformer, $apost_{stmt}$, captures the effect of different program 
statements in the abstract domain. The transformer is precise for octagonal 
assignments $(x:=y+1)$ but imprecise for non-octagonal assignments $(x:=y+z)$, 
as shown below.
\[apost_{x:=y+1}(a) = b = \langle x-y \leq 1, y-x \leq 1 \rangle \qquad apost_{x:=y+z}(b) = \langle \top \rangle \]  

\begin{definition}{(Abstract Valuation)} An {\em abstract valuation} is a
mapping of variables to an element of abstract domain, for example 
a mapping of variable $x$ to an interval environment is given by 
$\langle x \mapsto [2,5] \rangle$ or a mapping of $\{x,y\}$ to octagon 
environment is given by $\langle x-y \mapsto 0, y-x \mapsto 0 \rangle$.  
An abstract valuation is {\em atomic} if each variable is mapped to a singleton 
value or to $\bot$.  
\end{definition}

\begin{definition}{(Meet Irreducibles)} A {\em meet irreducible} $(M)$ 
in a complete lattice structure $A$ is a minimum complementable element 
$M \in A$ that has the following property.
\[\forall M_1, M_2 \in A, M_1 \meet M2 = M \implies (M = M_1 \lor M = M_2), M \neq \top \]  
\end{definition}

A meet irreducible in partial assignments domain are the singleton 
assignments, for example $\{x \mapsto true \}$, which represents the set 
of all propositional assignments where the literal $x$ is mapped to true.  
A meet irreducible in interval domain is $\langle x \leq n \rangle$ or 
$\langle x \geq n \rangle$.  

%
\begin{definition}{(Complementable Meet Irreducibles)} A complementable meet
irreducible $\bar{M}$ of an abstract lattice $A$ is the complement of a meet 
irreducible $M \in A$ such that $\bar{M} \in A$ and the concretisation of $M$ 
is the set complement of $\bar{M}$.  If every meet irreducible in $A$ is
complementable, then $A$ is said to be have complementable meet irreducibles.  
\end{definition}

\Omit {
An important property of meet irreducibles in case of partial assignments 
domain is that they have precise complements.  
For example, the complement of $\{x \mapsto true \}$ is a 
singleton element,$\{x \mapsto false \}$, in the partial assignments domain. 
}

%
%========================
\subsection{Literals and Clauses for Abstract CDCL}
%
\textit{Literal:} A Literal is a meet irreducible which specify that 
for certain program variable, there is a certain bound. An example of 
literal in interval domain is $x \in [0,5]$. An example of octagonal 
literal is given by $((x-y \leq 0) \wedge (y-x \leq 0))$.

Let I and J be two interval literals of the form $x \in [l, u]$ with $l$ 
as lower bound and $u$ as upper bound. We say $I \leq J$ or {\em (I leq J)} 
iff $(I.u \leq J.u \wedge I.l \geq J.l)$.  Whereas, {\em I disjoint J} iff 
$(I.u < J.l \vee  I.l > J.u)$. 

Similarly, let $O_1$ and $O_2$ be two octagonal literals of the form 
$x-y \leq c$.  Then, $O_1 \leq O_2$ {\em ($O_1$ leq $O_2$)} iff 
$O_1.c \leq O_2.c$.  Two octagons are disjoint ({\em $O_1$ disjoint $O_2$}) 
iff $O_1:= (x-y < c)$ and $O_2:= (x-y > c)$.  

\textit{Clause:} A clause is a disjunction of one or more meet irreducibles. 
An example clause is given by $(x \geq 0 \vee y \geq 5 \vee y+z \leq 10)$.

Let $x$, $x'$ be the literals in clause $C$ and current partial assignment 
$A$ respectively.  We call a literal $x \in C$ to be {\em satisfiable} with 
respect to the literal $x' \in A$ iff $(x \leq x')$.  We call $x$ to be  
{\em unsatisfiable} with respect to $x'$ iff $!(x \leq x') \wedge$ $!$(x disjoint $x'$).
A literal is said to be {\em contradicting} iff $!(x \leq x') \wedge$ (x disjoint $x'$).

Similarly, a clause may be catagorized into four classes -- satisfiable,
unsatisfiable, conflicting, unit. \\ 
\textit{Satisfiable clause:}
A clause $C$ is said to be {\em satisfiable} with respect to a current partial 
assignment $A$ if at least one literal in $C$ is satisfiable. For example, 
consider a clause $C=(x<4 \vee y>10)$ and let the current partial assignment 
be $A: x \in [0,3]$. Then $C$ is a satisfiable clause, where $x<4$ is a
satisfiable literal. 

\textit{Conflicting clause:}
If all literals in a clause are contradicting, then the clause is said to
be {\em conflicting}. For example, consider a clause $C=(x<4 \vee y>10 \vee z<15)$. 
Let the current partial assignment be $A: x \in [5,13]$ and $y \in [-2,9]$ and $z \in
[17,32]$, then $C$ is a conflicting clause where all literals in $C$ are
contradicting with respect to $A$. 

\textit{Unsatisfiable clause:}
A clause $C$ is said to be {\em unsatisfiable} with respect to a current partial 
assignment $A$ if there exists no satisfiable literal in $C$ or some literals in 
$C$ are unsatisfiable and the rest are contradicting. For example, consider a 
clause $C=(x<4 \vee y>10 \vee z<15)$.  Let the current partial assignment be 
$A: x \in [3,10]$, $y \in [8,10]$ and $z \in [12,20]$, then $C$ is a unsatisfiable 
clause, where $x<4$ and $z<15$ are unsatisfiable literal and $y>10$ is
contradicting literal. 

\textit{Unit clause:}
Clause $C$ is unit if all literals but one is contradicting in $C$. 
For example, consider a clause $C=(x<4 \vee y>10 \vee z<15)$. Let the 
current partial assignment be $A: x \in [5,13]$ and $y \in [-2,9]$ and $z \in
[10,12]$, then $C$ is a unit clause where the unit literal is $z$. 

\Omit {
\textit{Boolean constraint propagation (BCP):} BCP is the repeated 
application of unit rule. This corresponds to computing the greatest fixed point.

\textit{Characteristics of Conflict clause}
\begin{enumerate}
\item A conflict clause must include asserting cuts. An asserting cut is a cut
that contain exactly one node at the current decision level. Assertion cuts yields 
clauses that can be used to derive new information after backtracking.

\item A conflict clause must be UNIT after backtracking. 

%\item There can be multiple cuts and hence multiple UIPs. In other words, there
%can be multiple incomparable reasons for a conflict. But conflict analysis
%procedure choses one that is asserting. 

%\item The conflict clause should be made false by the current partial assignment
%and thus exclude an assignment leading to conflict. 
\end{enumerate}

\textit{Backjumping:}
The backjumping level is defined by the literal of the conflict clause assigned
at the level that is the closest to the conflict one. In other words, the
backjumping level is the level closest to the root (decision level 0)  where the
conflict clause is still unit. If a conflict clause is "globally"  unit, then
the backjumping level is the root of the search tree.
}

%========================

\subsection{Properties of Domain Elements for Learning}
%
\pscmt{TODO: Review by Peter}
An important property of a clause learning SAT solver 
is that each element of the partial assignments domain 
are decomposable into precisely complementable 
elements.  This property of the domain helps to guide 
the model search away from the conflicting region of the 
search space.  

Most numerical abstract domains, such as intervals, octagons, 
polyhedra lacks precise complements, but they can be 
represented as intersections of complementable half-spaces, 
each of which have precise complements.  For example, intervals 
lack complements, but the decomposition of intervals into 
meet-irreducibles have complements.  
\[x \mapsto [2,5] \mathrel{\mathop{\longrightarrow}^{\mathrm{decompose}}} \{\langle x
\geq 2 \rangle \meet \langle x \leq 5 \rangle \} \]
Similarly, the complementation of the octagon in Figure~\ref{octagon} 
can be written as disjunction of:
\[(x<-2) \lor (x>1) \lor (y<-1) \lor (y>2) \lor (x+y>2) \lor (x-y>1) \lor (y-x>3) \lor (-x-y>2)\]

Standard abstract interpretation does not require complementation property.  
Abstract Domain library, such as APRON C library~\cite{apron} is specialized 
for abstract interpretation.  We implement a template-based polyhedra domain 
that returns precise complements by decomposing it into half-spaces.    

\Omit {
The commonly used library for numerical abstract domains  
is the APRON C library~\cite{apron}.  This library is 
used for the static analysis of the numerical variables 
of a program by abstract interpretation. APRON provides a 
C API interface to various abstract domains and libraries 
such as {\em BOX}, {\em OCTAGON}, {\em Convex Polyhedra} and
{\em Linear Equalities} library.  The aim of such analysis is 
to compute invariants over numerical variables in the 
program~\cite{se2011}. 
}
\Omit {
To this end, we implement our own template-based polyhedra domain and interval 
domain which supports complementation operator.  
%For example, the octagon in Figure~\ref{octagon} can be written as a conjunction of:
%\[(x>=-2) \land (x<=1) \land (y>=-1) \land (y<=2) \land (x+y<=2)
%\land (x-y<=1) \land (y-x<=3) \land (-x-y<=2)\] 
\pscmt{That's no valid motivation. We never complement a whole octagon, but just a
  meet-irreducible. It's trivial to do that with APRON. The reason was
a different one: APRON does not support all C operators, e.g. the bitwise
operators.} 
} 

%===============================================================================

%===============================================================================
\section{Abstract Conflict Driven Clause Learning}
%
\begin{algorithm2e}[t]
\DontPrintSemicolon
\SetKw{return}{return}
\SetKwRepeat{Do}{do}{while}
%\SetKwFunction{assume}{assume}
%\SetKwFunction{isf}{isFeasible}
\SetKwData{conflict}{conflict}
\SetKwData{safe}{safe}
\SetKwData{sat}{sat}
\SetKwData{unsafe}{unsafe}
\SetKwData{unknown}{unknown}
\SetKwData{true}{true}
\SetKwInOut{Input}{input}
\SetKwInOut{Output}{output}
\SetKwFor{Loop}{Loop}{}{}
\SetKw{KwNot}{not}
\begin{small}
\Input{A program in the form of a set of abstract transformers $\abstransset$.}
\Output{The status \safe or \unsafe, \rmcmt{and a counterexample if \unsafe.}}
$\trail \leftarrow \langle\rangle$ \;
$\reasons \leftarrow []$ \;
$\mathit{result} \leftarrow \deduce_{\propheur}(\abstransset,\trail,\reasons)$ \;
\lIf{$\mathit{result}$ = \conflict} {
  \return \safe}
\While{$true$} 
{
\lIf{$\mathit{result}$ = \sat} {
  \return (\unsafe,$\abs(\trail)$)}
  $\decisionvar \leftarrow \decide_{\decheur}(\abs(\trail))$ \;
  $\trail \leftarrow \trail \cdot \decisionvar$ \; 
  $\reasons[|\trail|] \leftarrow \top$ \;
  $\mathit{result} \leftarrow \deduce_{\propheur}(\abstransset,\trail,\reasons)$\;
  \Do{$\mathit{result} = \conflict$} {
    \lIf{$\neg \analyzeconflict_{\confheur}(\abstransset,\trail,\reasons)$} {
      \return \safe
    }
    $\mathit{result} \leftarrow \deduce_{\propheur}(\abstransset,\trail,\reasons)$ \;
  }
}
\end{small}
\caption{Abstract Conflict Driven Clause Learning $ACDCL_{\propheur,\decheur,\confheur}(\abstransset)$ \label{Alg:acdcl}}
\end{algorithm2e}
%
%\subsection{Algorithm for Abstract Conflict Driven Clause Learning}
Algorithm~\ref{Alg:acdcl} presents an overview of the ACDCL algorithm.
The input to the algorithm is a program in the form of the set of its 
abstract transformers $\abstransset$ w.r.t.\ an abstract domain~$\domain$. \pscmt{define} 
The algorithm is parametrised by propagation \rmcmt{$(\propheur)$}, decision
$(\decheur)$, and conflict analysis heuristics $(\confheur)$.  
%\rmcmt{Approximation of the concrete transformers in 
%$\abstransset$ are typically available in abstract domain in the 
%form of strongest-post condition or weakest pre-condition. } 
The algorithm begins with an empty propagation trail $\trail$ and 
a reason trail $\reasons$.
%The propagation trail is initialized with an empty sequence.  
The propagation trail stores all meet irreducibles inferred by 
the abstract model search phase (deductions and decisions).  
The reason trail maps the elements of the propagation trail to 
transformers $\abstransel{}\in\abstransset$ that were used to
derive them. 
%
\begin{definition} 
The \emph{abstract value} $\abs(\trail)$ corresponding to 
the propagation trail $\trail$ is the conjunction of the 
meet irreducibles on the trail:
$\abs(\trail)=\bigsqcap_{m \in \trail}m$ with
$\abs(\trail)=\top$ if $\trail$ is the empty sequence $\langle\rangle$.
\end{definition}
%
\rmcmt{
\begin{definition} 
An abstract value $v$ is $\gamma$-complete if all concrete values in 
$\gamma(v)$ satisfy $\formula$.  
\end{definition}
}
%
The procedure $\deduce$ computes a greatest fixed point over the
transformers in $\abstransset$ that refines the abstract value,
similar to the BCP step in SAT solvers.  If the result of $\deduce$
returns \textsf{conflict} ($\bot$), the algorithm terminates with
\textsf{safe}.  Else, the analysis enters into the while($\true$) loop
and makes a new decision by a call to $\decide$ which returns a new
meet irreducible $\decisionvar$.
%
%\pscmt{what happens if we cannot make any decision any more?
%  -- then we should either have deduced $\bot$ or $\abs(\trail)$ must
%  be $\gamma$-complete} 
%
We concatenate $\decisionvar$ to the
trail $\trail$.  The decision $\decisionvar$ therefore refines \rmcmt{not
defined} the current abstract value $\abs(\trail)$ represented by the trail. 
\rmcmt{a decision is added to the trail only if it refines the trail.}
% WE ExPLAIN THAT LATER
%However, $\decisionvar$ must be consistent with the state of the
%solver, i.e. $\abs(\trail\cdot \decisionvar)\neq \bot$.
%
For example, a decision in the interval domain restricts the range of 
intervals for variables.
%
We set the corresponding entry in the reason trail $\reasons$ to $\top$
\rmcmt{to mark it as a decision.}
% so the analysis jumps under a 
%greatest fixed-point \pscmt{???}. Note that the widening operation in abstract interpretation 
%jumps above a least fixed-point, so decisions can be viewed as 
%{\em dual widening} \pscmt{what do we learn?}.  

The procedure $\deduce$ is called next to infer new meet irreducibles
taking into account the current decision.  The model search phase
alternates between the decision and deduction until $\deduce$ returns
either \textsf{sat} or \textsf{conflict}.  In the former case, we have
found an abstract value that represents models of $\formula$, which
are counterexamples to the safety problem, and thus \rmcmt{we or ACDCL} return
\textsf{unsafe}.
%
% has found an abstract value that represents a set of models of $\formula$ or the
% deduction leads to a \textsf{conflict}.  Checking whether the abstract
% model concretises to a set of models of $\formula$; this
% corresponds to a $\gamma$-completeness~\cite{dhk2013-popl} check in
% abstract interpretation.  If $v$ is $\gamma$-complete, then it cannot
% be refined further.  Thus, the algorithm returns the abstract model
% $v$, which is a set of concrete models, and terminates with
% \textsf{unsafe} or the current abstraction is insufficient to
% determine the satisfiability of $\formula$.  

In the latter case the algorithm enters in the $\analyzeconflict$
phase to learn the reason for the conflict.   There can be multiple
incomparable reasons for conflict.
% \pscmt{do you have to explain this here?} -- based on the choice of Unique
%Implication Point (UIP)~\cite{cdcl}.  
ACDCL heuristically chooses one reason $\conflictset$ and learns it 
by adding it as an abstract transformer to $\abstransset$. The algorithm 
backtracks by removing the content of $\trail$ up to a point where it does not 
conflict with $\conflictset$.  ACDCL then performs deductions with the learnt 
transformer.  If $\analyzeconflict$ returns $\false$ then no further
backtracking is possible.  Thus, the $\formula$ is unsatisfiable
and \rmcmt{we or ACDCL} return \textsf{safe}.

% A learnt clause must include asserting cuts which guarantees
% derivation of new meet irreducibles after backtracking. The clause
% learning and backtracking continues as long as the result of deduction
% is conflicting ($\bot$), that is, the abstract value does not
% abstractly satisfy the formula \pscmt{define}.  If no further
% backtrack is possible, then the algorithm terminates and $\formula$ is
% \textsf{safe}. Else, the algorithm makes a new decision and the above
% process is repeated until $\deduce$ returns \textsf{sat} or the
% algorithm backtracks to decision level 0 after a conflict in which
% case it returns \textsf{safe}.

%\paragraph{Solver State.}  \pscmt{This paragraph seems redundant. Doesn't this redefine $\trail$?} The state of a ACDCL solver is a tuple 
%of the form $\langle \mathcal{E}, S \rangle$.  Here, $\mathcal{E}$ 
%is a sequence of labelled information of the form $(m,s)$ where 
%$m$ is a meet irreducible and $s = \mathsf{decision}$ if $m$ is a decision, 
%or $s = \mathsf{deduction}$ if $m$ is inferred by a deduction. And $S$ is 
%a set of abstract deduction transformer \pscmt{???}.  A trail in a SAT solver 
%stores variable assignments of the form $(p, t)$, where $p$ is 
%a propositional variable that appears at most once in the trail 
%and $t$ is a truth assignment (true or false).  Whereas, a trail 
%$\trail$ in ACDCL contains a sequence of meet irreducibles 
%inferred by deduction or decision phase where a variable in trail 
%can be assigned \pscmt{constrained?} multiple times, each time with increasingly precise 
%bounds \pscmt{that's specific for a particular domain}.  
%\rmcmt{define trail refinement}
%\Omit {
%A current valuation $v$ is a meet of all elements of trail
%$\trail$, such that $v = \top$ when $\trail$ is 
%empty or $v=\underset{i \geq 0 \wedge i \leq |\trail|}{\meet m_i}$, where 
%$m_i \in \trail$.  A solver is in conflict if some clauses in
%$\abstransset$ is not abstractly satisfied by $v$.
%}
%\pscmt{define ``consistent with trail''}

%===============================================================================

%===============================================================================
\section{Abstract Model Search}
%
\Omit {
\begin{figure}[t]
\scriptsize
\begin{tabular}{l|l|l}
\hline
C program & SSA & Octagon \\
\hline
\begin{lstlisting}[mathescape=true,language=C]
int main() {
 unsigned x, y;
 __CPROVER_assume(x==y);
 x++;
 assert(x==y+1);
}
\end{lstlisting}
&
\begin{minipage}{4.40cm}
$\begin{array}{l@{\,\,}c@{\,\,}l}
SSA &\iff& ((g0 == TRUE) \land \\
    &    & (cond == (x == y)) \land \\
    &    & (g1 == (cond \&\& guard0)) \land \\
    &    & (x' == 1u + x) \land \\
    &    & (x' == 1u + y || !1))
\end{array}$
\end{minipage}
&
\begin{minipage}{3.75cm}
$\begin{array}{l@{\,\,}c@{\,\,}l}
C &\iff& ((x' > 1) \land (-x'-y < -2) \land \\
  &    & (-x-x' < -2) \land (y-x' < 0) \land \\                                                                
  &    & (x-x' < 0) \land (y > 0) \land \\
  &    & (x > 0) \land (-x'-y < 0) \land \\
  &    & (x+y > 1) \land (y-x < 1) \land \\
  &    & (x'-y < 2) \land (x-y < 1) \land \\
  &    & (x+y > 0) \land (x+x' > 0) \land \\
  &    & (x'-x < 2))
\end{array}$
\end{minipage}
\\
\hline
\end{tabular}
\caption{C Program, corresponding SSA and Octagonal Inequalities}
\label{ssa}
\end{figure}
}
%
%    
\begin{algorithm2e}[t]
\DontPrintSemicolon
\SetKwFunction{deduce}{deduce}
\SetKwFunction{print}{print}
\SetKwFunction{return}{return}
\SetKwFunction{continue}{continue}
\SetKwFunction{assign}{assign}
\SetKwData{sat}{SAT}
\SetKwData{conflict}{CONFLICT}
\SetKwData{unsat}{UNSAT}
\SetKwData{unknown}{UNKNOWN}
\SetKwData{true}{true}
\SetKwInOut{Input}{input}
\SetKwInOut{Output}{output}
\SetKwFor{Loop}{Loop}{}{}
\SetKw{KwNot}{not}
\begin{small}
\Input{Set $S$ of $\widehat{ded_{\varphi}}$ for $\varphi$ over Abstract domain $A$, propagation trail $\mathcal{T}$, reason trail $\mathcal{R}$}
\Output{\sat or \conflict or \unknown}
worklist $\leftarrow$ initialize\_worklist($S$) \;
\While{$!worklist.empty()$} 
{
  $(s,\mathit{live\_vars}) \leftarrow$ worklist.pop() \;
  $v \leftarrow ded(s, \mathit{espan}(\mathit{live\_vars}), \mathit{abs}(\mathcal{T}))$\;
  \uIf{$v = \bot$} {
    $\mathcal{R}[\bot] \leftarrow s$ \pscmt{how is this used in the conflict analysis? wouldn't it be more uniform to add $\bot$ to the $\mathcal{T}$?}\;
    worklist.clear() \pscmt{why can't you clean up the worklist during conflict analysis? it seems inefficient to throw everything away...} \;
    \return \conflict \;
  }
  \uElse
  {
    $\mathcal{T} \leftarrow \mathcal{T} \cdot \mathit{decomp}(v)$ \; 
%    $v \leftarrow v \meet \mathit{meet\_irrd}$ \; 
    $\mathcal{R}[|\mathcal{T}|] \leftarrow s$ \;
    worklist.update($v$) \; 
  }
}
\lIf{$\mathit{abs}(\mathcal{T})$ is $\gamma$-complete} {
  \return \sat;
}
\lElse {
 \return \unknown \;
}
\end{small}
\caption{Abstract Model Search $\langle deduce(S,v,\mathcal{T},\mathcal{R}) \rangle$ \label{Alg:ms}}
\end{algorithm2e}
%  
A model search procedure in a SAT solver involves two steps -- {\em deductions} 
using the unit rule refines current partial assignments and 
{\em decisions} to heuristically guess a value for an unassigned 
literal.  The unit rule overapproximates a model transformer and deduction 
computes a greatest fixed point over the partial assignments
domain~\cite{dhk2013-popl}.  We present an abstract model search procedure 
that computes a greatest fixed point over meet irreducible deduction 
transformer in $S$ \pscmt{???}.  
%

%-------------------------------------------------------------------------------
\subsection{Abstract Transformers} \label{sec:abst}
%-------------------------------------------------------------------------------

To make our algorithm efficient, we have to focus abstract
transformers on performing only the minimally necessary
work. 
%
We thus define a specialised variant of the transformer to compute
deductions w.r.t.\ a given subdomain $L\subseteq A$; $\bot,\top \in L$.
\[\llbracket s \rrbracket^\sharp(a,L)=\bigsqcup \{a'\mid a'\in L, a' \sqsubseteq a, a'\models s\}\]
We have the property $\llbracket s \rrbracket^\sharp(a)\sqsubseteq \llbracket s \rrbracket^\sharp(a,L)$.
%
For example, for
$a=(0\leq y \leq 1 \wedge 5\leq z)$ we have
$\llbracket x=y+z\rrbracket^\sharp(a,\{x\})=(x\geq 6)$, whereas
$\llbracket x=y+z\rrbracket^\sharp(a,\{y,z\})=\top$.
%
If $L=A$ then intuitively $a$ is intersected with the constraint $s$.
Yet, with the help of $L$ we can also control whether to perform
forward (right-hand side to left-hand side, $L=\{x\}$ in the above
example) or backward (lhs to rhs, $L=\{y,z\}$) propagation.

%Moreover, we want to know what the reasons for a specific deduction are.
%\[ded(s,a,L)=\{R\rightarrow d\mid R=reasons(s,a,L,d), d \in decomp(\llbracket s \rrbracket^\sharp(a,L))\}\]
%where 
%$reasons(s,a,L,d)=\text{argmin}_{R\in\{R'\mid R'\subseteq decomp(a), \llbracket s\rrbracket^\sharp(R,L)=d\}} |R| $.

\pscmt{give an example for $ded$, link it to the reason trail}

\pscmt{define $ded$ for all statements, connect this to worklist}

\pscmt{link livevars to $L$ -- the subdomain spanned by livevars}

%-------------------------------------------------------------------------------
\subsection{Algorithm for Deduction Phase}
%-------------------------------------------------------------------------------
%
\rmcmt{Talk about lazy closure operation}

\pscmt{
Rough and sketchy:
%
$espan(V)=$all meet irreducibles $\in A$ that contain at most one variable that is not in $V$.
%
$select(S,PV)$ is the set of statements that contain variables in $PV$ where $PV$ is the set of variables in the previous deduction.
%
$LV(s)$ are the live variables of a statement $s$, i.e.\ the intersection of $PV$ with the variables in $s$.
\begin{theorem}
$$
\gfp\lambda a. \bigsqcap_{s\in S} ded(s,a,A) = 
\gfp\lambda a. \bigsqcap_{s\in select(S,PV)} ded(s,a,espan(LV(s)))
$$
\end{theorem}
Proof sketch: see Appendix
}

Algorithm~\ref{Alg:ms} presents the deduction phase in abstract model search procedure.
A {\em worklist} is initialized following a heuristics choice for iteration 
strategies. A {\em forward} iteration strategy initializes the worklist with 
transformers that has constants in the right-hand side. Whereas, a {\em backward} 
iteration strategy starts with a set of assertions. On the other hand, a 
{\em multi-way} iteration strategy initializes the worklist with set of all 
transformers.  For every meet irreducible deduction transformer \pscmt{rename} $s \in S$ 
passed to the abstract domain, if the domain infers a new meet irreducible 
$\mathit{meet\_irrd}$, it is added to $\mathcal{T}$ \pscmt{It usually infers several ones.}.  Additionally, the reason 
trail $\mathcal{R}$ \pscmt{define} is updated with the transformer $s$ as the reason for inferring
$\mathit{meet\_irrd}$. If a conflict is deduced, that is, $\mathit{meet\_irrd} = \bot$ 
is inferred by the domain, the algorithm terminates with CONFLICT.  However,
when a fixed-point is reached, the abstract valuation $v$ is checked for set of
models.  This is determined by checking whether the set of assignments in $v$ is 
$\gamma$-complete~\cite{dhk2013-popl}. If $v$ is $\gamma$-complete, that is, the 
abstract valuation $v$ is abstractly satisfying, the abstract model 
search terminates with SAT.  Otherwise, the algorithm returns UNKNOWN and the 
model search procedure makes a new decision.    




%===============================================================================

%===============================================================================
\subsection{Decision}
A decision $\mathcal{D}$ is a meet irreducible that refines the 
current abstract valuation.  For example, a decision in interval 
domain can be of the form $\langle x R c \rangle$ where 
$R \in \{\leq, \geq\}$, $c$ is the bound.  A decision in 
relational domain such as octagon is of the form $ax - by \leq c$, 
where $x$ and $y$ are variables, $a,b \in \{-1,0,1\}$ are co-efficients, 
and $c \in \mathbb{R}\cup\{\infty\}$ is the bound of the inequality.  
That is, a decision can specify relation between variables, for example 
$\{x \leq y)$.  A decision must always be consistent with respect to 
the trail $\mathcal{T}$, that is, it must not contradict with the elements 
in the trail.  A new decision always increases the decision level by one. 

A singleton meet irreducible $s$ in an abstract domain $A$ is an element 
whose lower bound and upper bound are the same.  For example, 
$\langle x:[1,1] \rangle$ is a singleton meet irreducible in interval domain.  
For an octagon domain, $\langle 1 \geq x-y \leq 1 \rangle$ is a singleton meet 
irreducible for the template $(x-y)$, though the concretisation of individual 
template variables ($x$, $y$) may not be singletons.   

Given a set of variables $\mathcal{V}$ and a current abstract value $v$, 
a decision phase heuristically returns a non-singleton meet irreducible 
over a set of branching variables $\{B\} \subseteq \mathcal{V}$, a 
bound $c$, and a polarity ($\leq$ or $\geq$).  For interval domain, $|B|=1$ 
since an interval meet irreducible is defined over a single variable.  For 
octagons, $|B|=2$.  A decision adds a new meet irreducible $m$ to the trail.  
Subsequently, the new state of the solver is defined over $m$ and corresponding 
label information $s=decision$, which is described below. 
\[decide: \quad (\mathcal{E},S) \rightarrow (\mathcal{E}(m,s),S) \]

%Let $\mathcal{V}$ be a set of all singletons and non-singletons.  
%Otherwise, the variable is non-singleton 

ACDCL supports several decision heuristics namely, {\em ordered}, 
{\em longest-range}, {\em random}, {\em relational} and 
{\em Berkmin}~\cite{eugoldberg07} decision heuristics.  
The {\em ordered} decision heuristics creates an ordering among non-singleton 
meet irreducibles, thereby making decisions on meet irreducibles that involve 
conditional variables (variables that appear in conditional branches) first 
before choosing meet irreducibles with numerical variables.  
The ordering heuristic is similar to trace partitioning~\cite{toplas07}.  

The {\em longest-range} heuristics simply keeps track of the
interval $[l,u]$ range ($r=u-l$) of a meet irreducible and chooses 
a meet irreducible with the longest range value.  This ensures a 
fairness policy in selecting a variable since it guarantees that 
the intervals of meet irreducibles are uniformly restricted.  
%
The {\em random} decision heuristics arbitrarily picks a meet irreducible  
for making decision. 
%
The {\em relational} decision heuristics is only relevant for relational 
abstract domains.  
%
Whereas, a {\em berkmin} decision heuristics is inspired 
from decision heuristic used in Berkmin~\cite{eugoldberg07} SAT solver.  
The berkmin heuristic is currently implemented for interval constraints only.  
The heuristic keeps track of the activity of an interval meet irreducible 
that participate in conflict clauses. 
\Omit {
as well as variables that actively contribute to conflicts but do not explicitly 
appear in conflict clauses.  The set of conflict clauses is 
organized chronologically with the top clause 
as the one deduced in the last.  A branching variable is chosen among the 
free variables whose literals are in the top unsatisfied conflict clause.  
A similar decision heuristics is also implemented in Chaff~\cite{chaff} SAT 
solver, that computes the activity of a variable as the number of occurrences 
of that variable in conflict clauses only. 
}
%
A bound of a meet irreducible is heuristically chosen to be an approximation of the 
arithmetic average of the current bounds.  However, the polarity ($\leq$ or $\geq$) 
of a meet irreducible is chosen randomly.  

%===============================================================================

%===============================================================================
\section{Abstract Conflict Analysis}
Conflict analysis in a SAT solver can be seen as abductive 
reasoning which underapproximates a set of models that do not satisfy a 
formula~\cite{sas12,dhk2013-popl}.  
\Omit {
where a singleton assignment $c$ is replaced by a partial 
assignment that is sufficient to infer $c$.  Thus, a conflict 
analysis is a generalization of an under-approximation of a set of 
countermodels, where a countermodel is a 
set of structures that do not satisfy a formula 
}
Since relational abstract domains have more complex structure than the partial 
assignments domain or interval domain, so finding a generalized reason for 
conflict through UIP computation~\cite{uip} is non-trivial.  We formally 
describe the adaptations of conflict analysis procedure for relational domains. %
\begin{figure}[t]
%\caption{An example of Octagon}\label{octagon}
\scalebox{.65}{\import{figures/}{conflict_graph.pspdftex}}
\caption{\label{conflict} Abstract Conflict Graph}
\end{figure}  

\paragraph {\textbf{Abstract Conflict Graph}}
The first-UIP algorithm in SAT solvers works on a data structure
called {\em implication graph} and computes a cut that suffices to
produce a conflict.  The propagation and reason trails in ACDCL
implicitly encode a graph structure called {\em abstract conflict
  graph}, which records the dependencies between deductions made
during the abstract model search phase.  Nodes in the graph represent
elements of the trail $\trail$.  A node can be either a {\em decision}
node or a {\em deduction} node. Edges can be extracted from the
transformers in the reason trail $\reasons$.  Incoming arrows to node
$n$ indicate that the predecessors of $n$ are sufficient to deduce
$n$.  A decision node has no incoming edges.  For example, consider a
formula $\formula:= (y{=}x \wedge y{=}y{+}x \wedge y {\leq} 0)$ with
current abstract valuation $\absval:= (x{-}5 {\geq} 0 \wedge {-}x{+}5 {\geq}
0)$.  Figure~\ref{conflict} shows a snapshot of an abstract conflict
graph that stores the deductions obtained from abstract model search
using the octagon domain.  The last deduction conflicts with the
meet irreducible $(y \leq 0)$.

\paragraph {\textbf{Lifting UIP to relational domains}}
An UIP is a special node in the abstract conflict graph such that 
any path from the last decision node to the conflict node must pass 
through it.  Unlike a SAT solver, a UIP for relational domain may 
involve meet irreducibles that contain the same variables.  The 
first-UIP is a unique node closest to the conflict, and the last UIP 
is the decision node itself.  Computing UIPs ensures asserting
cuts~\cite{cdcl,DBLP:journals/fmsd/BrainDGHK14}, that is, it 
yields clauses that generate new deductions after backtracking.  
Every cut in the graph is a reason for conflict that can be 
used in learning.  An abstract first-UIP algorithm~\cite{DBLP:journals/fmsd/BrainDGHK14} 
traverses the trail $\trail$ starting from the conflict node and 
computes a cut that suffices to produce a conflict. 
\Omit {
contradict with the transformer stored in $\reasons[\bot]$.  
}
For our example, there exist multiple incomaparable reasons for conflict,
marked as {\em cut0, cut1}, in Figure~\ref{conflict}.  Here, cut0 is the first UIP.  
Choosing cut0 yields a learnt clause 
$(x+y<15 \vee -x-y<-15 \vee -x+y<5 \vee x-y<-5 \vee y-10<0 \vee -y+10<0)$, 
which is obtained by negating the reason for conflict.  
%    
\paragraph{\textbf{Learning in relational domains}}
Learning in a propositional solver yields an asserting
clause~\cite{cdcl} that expresses the negation of the conflict
reasons.  However, we model learning in relational domains as learning
a transformer which infers new meet irreducibles using the abstract
unit rule called {\em abstract unit transformer}.  We add $\aunit$ to
the set of abstract transformers $\abstransset$ used during model
search. $\aunit$ is a generalization of propositional unit rule for
numerical domains.  For an abstract lattice $\domain$ with
complementable meet irreducibles and a set of meet irreducibles $\conflictset
\subseteq \domain$ such that $\bigsqcap
\conflictset$ does not satisfy $\formula$, $\aunit_\conflictset: \domain \rightarrow
\domain$ is formally defined as follows.
\[ \aunit(\absval) =
 \left\{\begin{array}{l@{\quad}l@{\qquad}l}
  \bot       & \text{if } \absval \sqsubseteq \bigsqcap \conflictset & (1)\\
  \bar{t}    & \text{if } t \in \conflictset \; \text{and} \; \forall t' \in \conflictset
  \setminus \{t\}. \absval  \sqsubseteq t' & (2) \\
  \bar{t}    & \text{if } t \in \conflictset \; \text{and} \; \forall t' \in \conflictset \setminus \{t\}. \absval
  \not\sqsubseteq t', \; \text{then} \\ 
             & \quad \forall \absval' \sqsubseteq \absval: \absval'
             \not\sqsubseteq t \text{ and} & (3)\\ 
             & \quad \absval''=\exclude(\absval,\absval') \text{ and
             }\absval''\sqsubseteq t'  \\
  \top & \text{otherwise} & (4) \\
 \end{array}\right.
\]
with $\exclude(\absval,\absval')=\bigsqcap(\decomp(\absval) \setminus
\decomp(\absval'\}))$.  $\aunit$ returns $\bot$ if $\conflictset$ is
conflicting following rule (1).  Rule (2) and (3) of $\aunit$ infer a
valid meet irreducible, which implies that $\conflictset$ is unit.

$\aunit$ strictly generalizes the unit rule in SAT solvers.  
However, rule (3) only applies to interval and octagons domains.  
For example, let $\conflictset = \{ x+z \geq 10, w \geq 0 \}$ be a conflicting element 
that does not satisfy $\formula$ and let $\absval= (x+y
\geq 2\wedge y+z \geq 5\wedge x+z \leq 15\wedge w \geq 1\wedge w \leq
1)$.  Applying rule (3) of $\aunit_\conflictset$, we get a
decomposition of $\absval$ into $\absval'$ and $\absval''$ where
$\wedge'= (x+y \geq 2\wedge y+z \geq 5\wedge x+z \leq 15)
\not\sqsubseteq (x+z \geq 10)$, and $\absval''= (w \geq 1\wedge w \leq
1) \sqsubseteq (w \geq 0)$.  Then, $\aunit_C(a) = (x+z < 10)$, which
constrains the bound of the octagonal constraint.

Let us consider another example, where $\conflictset = \{x \geq 2, x
\leq 5, y \leq 7 \}$ and let $\absval = (x \geq 3\wedge x \leq 4\wedge
y \geq 5\wedge y \leq 6)$, then $\aunit_\conflictset(\absval) = \bot$
using the rule (1), since $\absval \sqsubseteq \bigsqcap\conflictset$.  Now,
let $\absval' = (x \geq 3\wedge x \leq 4)$, then
$\aunit_\conflictset(\absval') = (y \geq 8)$ using rule (2), since
$\absval' \sqsubseteq (x \geq 2)$ and $\absval' \sqsubseteq (x \leq
5)$.  However, if none of the above rule holds true, $\aunit$ returns~$\top$.
 
\paragraph {\textbf{Backjumping}}
After adding $\aunit$ to $\abstransset$, we remove all the meet irreducibles from the trail back to a decision level that restores the solver to a
non-conflicting state.  The backjumping level is defined by the
meet irreducible of the conflict clause that is closest 
to the root (decision level~0) where the conflict
clause is still unit.  If a conflict clause is globally unit, then the
backjumping level is the root of the search tree and
$\analyzeconflict$ returns $\false$, otherwise $\true$.

%The abstract clause learning and backjumping procedures in the abstract 
%conflict graph is stated in terms of the state of ACDCL solver as follows.
%THIS is already explained above
%\[AbsLearn: \quad  (\mathcal{E},S) \rightarrow (\mathcal{E},S \wedge
%\mathcal{L}) \quad \text{if} \; \mathcal{L} \notin S \; \textrm{and}
%\; (S \wedge \mathcal{L}) \; \text{is not UNSAT} \]
%\[AbsBackjump: \quad (\mathcal{E}_1(m,s)\mathcal{E}_2,S) \rightarrow
%(\mathcal{E}_1,S) \quad \text{if} \; (\mathcal{E}_1,S) \; \text{is not in conflict} \]   

\Omit {
\subsection{Clause Learning in Abstract Lattice}
Conflict graph for Intervals
Conflict graph for Octagons

\textit{Characteristics of Conflict clause}
\begin{enumerate}
\item A conflict clause must include asserting cuts. An asserting cut is a cut
that contain exactly one node at the current decision level. Assertion cuts yields 
clauses that can be used to derive new information after backtracking.

\item A conflict clause must be UNIT after backtracking. 

\item There can be multiple cuts and hence multiple UIPs. In other words, there
can be multiple incomparable reasons for a conflict. But conflict analysis
procedure choses one that is asserting. 

\item The conflict clause should be made false by the current partial assignment
and thus exclude an assignment leading to conflict. 
\end{enumerate}

1.DPLL style -- chronological backtracking \\
2. CDCL style -- non-chronological backtracking \\
  a. first-uip \\
  b. last-uip \\

**********************************************
\subsection{Lifting First UIP to Octagon domain}
**********************************************
unit-ness guarantee in octagon domain:
  Popped stmt: y23=1+y21
   Abstract value:
   D1: y23-y21 < 2 &&
   D2: y23+y21 > 0 &&
   D3: y21 < 1
   After backtracking, apply unit rule 
   y23 > 1 -- deduction from unit rule
   Value inconsistent !!
 
 Note: 
 1> cannot make reasoning at literal level for relational domain because literals are dependant on each other. As soon as literals denote relation between first-order variables, the pure reasoning on boolean skeleton is not sufficient. 
 
2> Intervals are orthogonal half spaces similar to booleans. 

3> After backtracking, the application of unit rule is done as follows for relational domains:
  Pass the learnt clause (as statement) and the abstrat value to the domain to make deductions.
}

%===============================================================================

%===============================================================================
\section{Experimental Results}
We have implemented the instantiation of ACDCL on relational domains
in 2LS verification tool~\cite{2ls}.  We used the abstract domains
provided by 2LS (intervals, octagons and equalities) for our
experiments. Although the performance of these implementations is not
competitive with dedicated implementations as found in
APRON~\cite{apron}, for instance, they have the advantage that they
handle all C operators (including bitwise operations) out of the box.

We verified a total of 70 benchmarks in ANSI-C, which are derived from 
(a) bit-vector regression category in SV-COMP'16, (c) bit-precise and 
cycle accurate software models of hardware circuits automatically 
generated using v2c~\cite{mtk2016}, (c) controller code generated 
from Simulink (d) several hand-crafted benchmarks for equivalence 
checking and bounded loop analysis.  All benchmarks containing 
bounded loops are completely unrolled before analysis.  We compare 
our tool with a state-of-the-art bounded model checker CBMC and a 
commercial static analysis tool, Astr{\'e}e.  CBMC uses MiniSAT 
solver in the backend.  Astr{\'e}e was configured with interval 
domain, trace-partition domain, and relational domains.  All the 
bounded loops are completely unrolled in Astr{\'e}e.   

\Omit {
To enable precise analysis using Astr{\'e}e, all our benchmarks are 
manually instrumented with partition directives which provides external 
hint to the tool to guide the trace partitioning heuristics.  Usually, 
such high-precision is not needed for static analysis, since it makes 
the analysis very expensive.  Without trace partitioning, the 
analysis using Astr{\'e}e shows high degree of imprecision. 
}
\paragraph {\em Observation of Analysis}

\paragraph {\em \textbf{Decision Heuristics}} We compare the performance of 
different decision heuristic on our benchmarks.  We observe that 
the ordered heuristic outperforms other heuristics on control-intensive 
benchmarks due to its ability to prioritize decisions on variables that 
appears in conditional branches.  For straight-line code and most of 
the bit-vector regression suite, random heuristics performs the best.  
Whereas, the activity based heuristics such as Berkmin heuristic which 
works well in propositional cases performs best for benchmarks that 
encountered the maximum number of conflicts to prove safety, thus allowing 
the heuristics to choose the decison variable among the set of learnt clauses.   

\paragraph {\em \textbf{Propagation Strategy}}      

\paragraph {\em \textbf{Clause Learning}}      

  

%===============================================================================

%===============================================================================
\section{Related Work}
The work of~\cite{sas13,DBLP:journals/fmsd/BrainDGHK14} present an
instantiation of ACDCL as a decision procedure for floating-point
arithmetic that uses interval constraint propagation.  
%
The use of
ACDCL as program analysis is proposed in~\cite{tacas12}.  The analysis
operates over the abstract domain of floating-point numbers and
machine integers.  Thus, it is restricted to bounds checking for
numerical programs.  
%
A similar technique that lifts DPLL(T) to
programs is Satisfiability Modulo Path Programs (SMPP) \cite{SMPP}. SMPP
enumerates program paths using a SAT formula, which are then verified
using abstract interpretation.  
%
The lifting of CDCL to first-order theories is proposed
in~\cite{dpll,cp09,ndsmt}.
%
\Omit{ operates on a fixed first-order partial assignment lattice
  structure, where first-order variables are mapped to domain values,
  similar to constants lattice in program analysis.  } However, unlike
previous works that operate on a fixed first-order lattice structure,
ACDCL can be instantiated with different abstract domains.  This
involves model search and learning in abstract lattices.  \Omit { The
  abstract lattice in natural domain SMT does not have complementable
  meet irreducibles, and therefore does not support generalized clause
  learning~\cite{sas}.  On the other hand, most abstract lattice have
  complementation property, thus enabling ACDCL to perform generalized
  clause learning.  } 
%
\cite{DBLP:journals/fmsd/BrainDGHK14} presents an instantiation of ACDCL for
heap-manipulating programs, but uses a simplistic conflict analysis.
%
\cite{DBLP:conf/vmcai/PelleauMTB13} describes a constraint solving algorithm
using relational numerical domains, but performs model search only.
%
The work of \cite{DBLP:conf/esop/MineBR16} propose an algorithm inspired by constraint
solvers for inferring disjunctive invariants using intervals.
%
% is presented in . They
%use a relational domain, but focus on the model search, whereas the
%conflict analysis is simplistic and restarts the algorithm with the
%learnt clause instead of backtracking.
%}


%===============================================================================

%===============================================================================
\section{Conclusions}
In this paper, we instantiate ACDCL solver with relational domains.  We explained specific properties of model search and conflict analysis procedure necessary
for such lifting.  Experimental evaluation on different benchmarks shows that
ACDCL is able to suitably exploit the expressivity of richer
abstract domains as well as precision of CDCL architecture.
This is attributed to the intelligent decision heuristics
which exploits the high-level structure of the problem
combined with stronger deductions and clause learning
mechanism aided by the richer abstract domains.

Future work -- domain refinement, incremental solving, invariant inference, intelligent $\gamma$-completeness check. Implement dedicated abstract domains to exploit full performance.
generalization for relational domain, instantiate acdl with trace-based domains.

%===============================================================================

\bibliographystyle{splncs03}
\bibliography{biblio.bib}

\extendedonly{
\appendix
\section{Omitted proofs}

\pscmt{
Rough and sketchy:
%
$espan(V)=$all meet irreducibles $\in A$ that contain at most one variable that is not in $V$.
%
$select(S,PV)$ is the set of statements that contain variables in $PV$ where $PV$ is the set of variables in the previous deduction.
%
$LV(s)$ are the live variables of a statement $s$, i.e.\ the intersection of $PV$ with the variables in $s$.
\begin{theorem}
$$
\gfp\lambda a. \bigsqcap_{s\in S} ded(s,a,A) = 
\gfp\lambda a. \bigsqcap_{s\in select(S,PV)} ded(s,a,espan(LV(s)))
$$
\end{theorem}
Proof sketch:
When computing $ded$, we obtain a deduction that affects variables $V$.
This has two consequences: (1) we have to compute the closure; (2) we have to propagate through all statements $s$ affected by the changes to $V$. 
We do only (2): $espan$ augments the set of affected variables by one transitivity step that we would have to compute in the closure. $ded$ on this subdomain hence performs the deduction w.r.t. $a$ and $s$ plus one transitivity step of the closure. Thus, by induction, we eventually compute the entire closure just by doing (2).

What are the consequences regarding complexity?
Assuming we require $m$ iterations to reach the fixed point with eager closure; we have to perform $|A|^c|S|m$ (idealised) computation steps to do that (where $c=3$ for octagons). With the lazy closure we have to do $|L|^c|S|mk$ computations taking into account that we need $k$ times more iterations to converge. Hence, the approach pays off if $k<<\frac{|A|^c}{|L|^c}$, which seems likely for $c=3$.

}

\section{Detailed experimental results}

}

\end{document}
