\documentclass[a4paper,conference]{llncs}

\usepackage{import}
\usepackage{times}
\usepackage{graphicx}
\usepackage{cite}
\usepackage{amsfonts}
\usepackage{amssymb}
\usepackage{amsmath}
\usepackage{mathptm}
\usepackage{color}
\usepackage{listings}
\usepackage{verbatim}
\usepackage{comment}
\usepackage{alltt}
\usepackage{psfrag}
\usepackage{epsfig}
\usepackage{wasysym} 
\usepackage{subfigure}
\usepackage{paralist}
\usepackage{dingbat}
\usepackage[algo2e,linesnumbered,ruled,lined]{algorithm2e}
\usepackage{hyperref}
\makeatother

\newcommand{\tool}[1]{\textsc{#1}\xspace}
\newcommand{\cbmcv}{\tool{cbmc 5.0}}
\newcommand{\Omit}[1]{}

\input{symbols}

\begin{document}

\title{Lifting CDCL to Numerical Domains}

\author{Rajdeep Mukherjee Peter Schrammel   Daniel Kroening}

\institute{University of Oxford, UK \and University of Sussex, UK \\
\email{\{rajdeep.mukherjee,kroening\}@cs.ox.ac.uk},
\email{p.schrammel@sussex.ac.uk}}
\Omit {
\author{Rajdeep Mukherjee\inst{1} \and
Peter Schrammel\inst{2} \and
Daniel Kroening\inst{1}}

\institute{University of Oxford, UK \and University of Sussex, UK \\
\email{\{rajdeep.mukherjee,kroening\}@cs.ox.ac.uk},
\email{p.schrammel@sussex.ac.uk}}
}

\maketitle

%===============================================================================
\begin{abstract}
%===============================================================================
%
The conflict driven clause learning algorithm can be seen as an 
alternation between over-approximate deduction with boolean 
constraint propagation, and under-approximate abduction, with 
conflict analysis.  {\em Abstract Conflict Driven Clause Learning} 
(ACDCL) is a novel program analysis that embeds an abstract domain 
inside the Conflict Driven Clause Learning (CDCL) algorithm. From 
decision procedure perspective, ACDCL is a SAT solver for program 
analysis constraints and is a strict generalisation of propositional CDCL 
solvers. From Abstract Interpretation point of view, ACDCL can be 
viewed as an abstract interpreter that uses decision and learning to 
increase the precision of abstract transformers.  In this paper, we 
instantiate ACDCL with different relational and non-relational numerical 
abstract domains and evaluate the performance of our program analyzer
with program verifiers that uses propositional-level solvers.  Experimental 
evidence on a set of \rmcmt{N} benchmarks shows significant drop in number 
of decisions and backjumping iterations in addition to stronger 
deductions and learned clause aided by the richer abstract domains.
%
\end{abstract}
%===============================================================================
\section{Introduction}
%
In this paper, we identify specific properties of the CDCL algorithm that 
are necessary to lift the propositional skeleton of CDCL to other richer 
lattice structures.

Most lattice structure used in static analysis have meet and join operations 
but lack negations. The meet operation precisely model conjunction, and
the join operation over-approximate disjunction. Thus, the precision loss in 
static analyzers is often due to imprecise join operations. The loss in precision is
overcome by using richer abstract domains or equipping the analysis with disjunction --
which is often very expensive since the analysis exhibit case enumeration behavior. 
 
On the other hand, the combination of precise conjunction in partial assignments domain 
with learning allow CDCL solvers to reason about disjunction without 
enumerating cases. Learning can be viewed as synthesizing an abstract transformer for 
negation. ACDCL lifts learning techniques in SAT solver to operate on non-distributive 
lattice -- thus equipping the analysis with negation. This enables the analyzer to refine 
the analysis and prevent enumeration behavior.

\rmcmt{Our lifting of CDCL is generic for any abstract domains.  We used
octagons, equalities and intervals for our experiments.}

%===============================================================================

%===============================================================================
\section{Related Work}
The work of~\cite{sas13} present an instantiation of ACDCL as a decision 
procedure for floating-point arithmetic that uses interval constraint 
propagation.  The use of ACDCL as program analysis is proposed
in~\cite{sas12,tacas12}.  The analysis operate over the abstract domain of 
floating-point and machine-integer.  Thus, it is restricted to bounds 
checking for numerical programs.  A similar technique that lifts DPLL(T) 
to programs is Satisfiability Modulo Path Programs (SMPP). SMPP enumerates 
path programs using a SAT formula which are then verified using abstract 
interpretation.  The lifting of CDCL to first-order theory is proposed 
in~\cite{dpll,cp09} and natural domain SMT framework~\cite{ndsmt}.  
\Omit{
operates on a fixed first-order partial assignment lattice structure, 
where first-order variables are mapped to domain values, similar to constants lattice 
in program analysis.  
}
However, unlike previous works that operate on a fixed first-order 
lattice structure, ACDCL can be instantiated with different 
abstract domains.  This involves model search and learning in 
abstract lattices.     
\Omit {
The abstract lattice in natural domain SMT does 
not have complementable meet irreducibles, and therefore does not 
support generalized clause learning~\cite{sas}.  On the other hand, 
most abstract lattice have complementation property, thus enabling 
ACDCL to perform generalized clause learning.
}

%===============================================================================

%===============================================================================
\section{Contributions}
In this paper, we have the following contributions.
\begin{enumerate}
\item We present a procedure for lifting CDCL algorithm to 
relational and non-relational numerical abstract domain.  
To this end, we instatiate model search and conflict analysis 
phases of CDCL algorithm over intervals and octagon abstract 
domains.  We implement our own template-based numerical 
domains that supports complementation property, essential for 
such lifting.  To the best of our knowledge, this is the first 
instantiation of CDCL algorithm with relational abstract domains. 
 
\item We present a new program analysis tool, {\em Titlis}, that 
is a strict generalisation of propositional CDCL algorithm.  From 
abstract interpretation perspective, {\em Titlis} uses decision 
and learning to improve the precision of abstract transformers.  
Thus, {\em Titlis} combines the efficiency of abstract domains 
while being path-sensitive and bit-precise at the same time.   
   
\item  Experiments on several numerical and control-intensive benchmarks 
shows that the proposed generalization of CDCL algorithm is able to 
suitably exploit the expressivity of richer abstract domains as well 
as precision of CDCL architecture.  This is attributed to the 
intelligent decision heuristics which exploits the high-level 
structure of the problem combined with stronger deductions and 
clause learning mechanism aided by the richer abstract domains.
\end{enumerate}

%===============================================================================

%===============================================================================
\section{Definitions}
%
\textit{Literal:} A Literal is a meet irreducible which are minimum complementable
elements which specify that for certain program variable, there is a certain
bound. An example of literal is $x \leq 0$.

Let us define the notion of literals with respect to interval domain. 
An interval $([l:u])$ is defined by a lower bound $(l)$ and an upper bound $(u)$.
Let I and J be two intervals. The relation between I and J is defined as follows:\\
We say $I \leq J$ or {\em (I leq J)} if $(I.u \leq J.u \wedge I.l \geq J.l)$\\
We say {\em I disjoint J} if $(I.u < J.l \vee  I.l > J.u)$.

Let $x$, $x'$ be the intervals of a variable $p$ in current 
partial assignment and the current clause respectively. 
We define a literal and a clause to be satisfiable or unsatisfiable 
or contradicting with respect to an interval abstraction as follows. \\
\textit{Satisfiable literal:} 
We say $x'$ is satisfiable literal if $(x \leq x')$.

\textit{Unsatisfiable literal:}
We say $x'$ is satisfiable literal if $!(x \leq x') \wedge$ $!$(x disjoint $x'$).

\textit{Contradicting Literal:}
We say $x'$ is contradicting literal if $!(x \leq x') \wedge$ (x disjoint $x'$).

\textit{Clause:} A clause is a disjunction of one or more meet irreducibles. An 
example clause is $(x \geq 0 \vee y \geq 5 \vee y \leq 10 \vee y \leq 7)$.

\textit{Satisfiable clause:}
If at least one meet irreducible in a clause is satisfiable, then a clause is said to 
be satisfiable. For example, consider a clause $C=(x<4 \vee y>10)$. Let the 
current partial assignment be $x \in [0,3]$, then $C$ is a satisfiable clause. 

\textit{Conflicting clause:}
If all the meet irreducibles in a clause are contradicting, then the clause is said to
be conflicting. For example, consider a clause $C=(x<4 \vee y>10 \vee z<15)$. Let the 
current partial assignment be $x \in [5,13]$ and $y \in [-2,9]$ and $z \in
[17,32]$, then $C$ is a conflicting clause.

\textit{Unsatisfiable clause:}
If no meet irreducibles in a clause are satisfiable or some meet
irreducibles are not satisfiable and the rest are contradicting, we call a 
clause unsatisfiable. For example, consider a clause $C=(x<4 \vee y>10 \vee z<15)$. 
Let the current partial assignment be $x \in [3,10]$, $y \in [8,10]$ and $z \in [12,20]$, 
then $C$ is a unsatisfiable clause. 

\textit{Unit clause:}
If all meet irreducible but one is contradicting in a clause, we call the clause
to be unit. For example, consider a clause $C=(x<4 \vee y>10 \vee z<15)$. Let the 
current partial assignment be $x \in [5,13]$ and $y \in [-2,9]$ and $z \in
[10,12]$, then $C$ is a unit clause where the unit literal is $z$. 

\textit{Unit Rule:} A unit rule is the best abstract transformer 
(strongest post-condition or weakest pre-condition). 

\textit{Boolean constraint propagation (BCP):} BCP is the repeated 
application of unit rule. This corresponds to computing the greatest fixed point.

\textit{Characteristics of Conflict clause}
\begin{enumerate}
\item A conflict clause must include asserting cuts. An asserting cut is a cut
that contain exactly one node at the current decision level. Assertion cuts yields 
clauses that can be used to derive new information after backtracking.

\item A conflict clause must be UNIT after backtracking. 

%\item There can be multiple cuts and hence multiple UIPs. In other words, there
%can be multiple incomparable reasons for a conflict. But conflict analysis
%procedure choses one that is asserting. 

%\item The conflict clause should be made false by the current partial assignment
%and thus exclude an assignment leading to conflict. 
\end{enumerate}

\textit{Backjumping:}
The backjumping level is defined by the literal of the conflict clause assigned
at the level that is the closest to the conflict one. In other words, the
backjumping level is the level closest to the root (decision level 0)  where the
conflict clause is still unit. If a conflict clause is "globally"  unit, then
the backjumping level is the root of the search tree.

%===============================================================================

%===============================================================================
\section{Motivating Example}
%
\begin{figure}[t]
\scriptsize
\begin{tabular}{l|l|l}
\hline
C program & Control-Flow Graph & Abstract Conflict Graph \\
\hline
\begin{lstlisting}[mathescape=true,language=C]
void main() {
 int x,z;
 int val;
 assume(val>=0 && val<=N);
 _Bool c;
 if(c)
   x = val;
 else 
   x = -val;
 z = x * x;
 assert(z>=0);
}
\end{lstlisting}
&
\begin{minipage}{4.40cm}
\centering
\vspace*{0.3cm}
\scalebox{.45}{\import{figures/}{example.pspdftex}}
%\caption{A control flow graph for P, P' and abstract conflict graph (ACFG) \label{fig:filter}}
\end{minipage}
&
\begin{minipage}{4.40cm}
\centering
\vspace*{0.3cm}
\scalebox{.45}{\import{figures/}{acdl_run.pspdftex}}
%\caption{A control flow graph for P, P' and abstract conflict graph (ACFG) \label{fig:filter}}
\end{minipage}
\\
\hline
\end{tabular}
\caption{\label{example}
C Program, Control-flow Graph (CFG) and Abstract Conflict Graph
\pscmt{change \&\& in the CFG to $\wedge$; the constraint towards to error does not show up correctly; in the second oct analysis box put $\wedge$ between the constraints; add an arrow from the interval Error box with three dots to indicate that the analysis continues (even though you explain it in the text); shorten ``val'' to ``v''}
}
\end{figure}
Let us consider a simple program in Figure~\ref{example} that performs 
multiplication of two machine integers and checks that the result of the 
multiplication is always positive for an integer input $x$.  To avoid 
overflow error, we assume $x$ has an upper bound $N$.  The control 
flow graph is shown in the middle column of Figure~\ref{example}.  

We analyze the program using three different analyzers -- {\em bounded 
model checking (BMC)}, {\em abstract interpretation} and {\em ACDCL}.    
\Omit {
The SAT solver makes 36 decisions, 1365 boolean constraint propagations 
and performs 5 conflict analysis with 65 conflict literals.  Moreover, 
the solver performs a restart, presumably to recover from bad decisions.  
}
Standard BMC tools convert the program to a bit-vector equation and pass 
them to a SAT solver.  However, standard forward interval analysis is 
too imprecise to verify the safety of the above program.  The 
imprecision is due to control-flow join at node $n4$.  State-of-the-art 
abstract interpretation tool, Astr{\'e}e, requires external hints, provided 
by manually annotating the code with partition directives at $n1$, 
to prove safety.  The partition directive helps the tool to analyze 
each program execution paths separately.  To do so, Astr{\'e}e 
requires a product of trace partitioning domain and an interval domain 
to determine safety.  However, a relational domain such as octagons is 
able to prove the program without any partition.  In general, the 
imprecision is either intended by the tool because such high 
precision analysis is normally not required for runtime error 
analysis or the imprecision is unavoidable due to the complexity 
of the application under analysis.  
%
\begin{table}
\begin{center}
{
\begin{tabular}{l|l|l|l|l|l}
\hline
Solver & Decisions & propagations & conflict & conflict literals & restarts \\ \hline
SAT & 233 & 36436 & 162 & 2604 & 2 \\ \hline
ACDCL (Interval) & 1 & 17 & 1 & 1 & 0 \\ \hline
ACDCL (Octagons) & 0 & 7 & 0 & 0 & 0 \\ 
\hline
\end{tabular}
}
\end{center}
\caption{Solver statistics}
\label{solver}
\end{table}
%
The right-side of Figure~\ref{example} shows an analysis using ACDCL with 
octagon domain (top) and an interval domain (bottom).  The analysis 
associates an abstract element for each control location and variable, 
as shown in the abstract conflict graph in Firgure~\ref{example}.  
Octagonal constraints deduced by ACDCL solver are sufficient to prove 
safety without any decision or clause learning.  However, unlike Ast{\'e}e, 
an automatic interval analysis is also sufficient to prove safety with 
only a single decision.  Interval constraints generated from forward 
analysis in the initial deduction phase infers that 
$\langle x:[-5,5], z:[-25,25] \rangle$.  Clearly, ACDCL can not prove safety.  

Hence, ACDCL makes a decision $c:=[1,1]$ to refine the analysis.  
ACDCL implements several decision heuritics.  In this case, the 
branching variable $c$ is chosen by an {\em ordered} decision 
heuristics that select variables appearing in conditional branches 
first before choosing other program variables. 
The decision constrains the interval of $x:=[0,5]$.  The deductions made 
during fixed point iteration are represented by abstract conflict 
graph in right-side of figure~\ref{example}.  Nodes in the abstract 
conflict graph denotes the corresponding updates to the program 
variables in the CFG.  Interval analysis concludes that the $(ERROR:\bot)$, that 
is the decision $c:=[1,1]$ leads to a {\em conflict}.  Thus, 
the program is {\em safe} for $c:=[1,1]$.  

A clause learning SAT solver would learn the reason for conflict at this 
point and then backtrack to a level such that the learnt clause is UNIT.  
Similar to conflict analysis phase in SAT solvers, ACDCL learns that 
(\(I: c:=[0,0]\)), that is all error traces must satisfy $(c \neq 1)$ 
at node $I$.  The analysis discards all interval constraints that
lead to the conflict and backtrack to decision level 0.  ACDCL then 
performs interval analysis with the learnt constraint $(c \neq 1)$ which also 
leads to a conflict.  The analysis cannot backtrack further and therefore 
it terminates proving that the program is safe.  A DPLL-style solver would 
perform proof by cases.  However, decision and clause learning are used 
to avoid case based reasoning and prevents enumeration behavior.     
   
For $N=46000$, table~\ref{solver} shows the statistics from 
MiniSAT~\cite{minisat} solver for analysis using BMC, and 
analysis using ACDCL with interval and octagon domains.  
Compared to a SAT solver, there is a significant reduction 
in the number of decisions, propagations, learnt clauses and 
restrarts.  Compared to abstract interpretation, ACDCL does 
not require external hints to proof the program, thus 
automatically performing program and property driven trace 
partitioning to generate proofs using decision and clause learning.  

%===============================================================================

%===============================================================================
\section{Abstract Conflict Driven Clause Learning}
%
\begin{algorithm2e}[t]
\DontPrintSemicolon
\SetKwFunction{decide}{Decide}
\SetKwFunction{deduce}{deduce}
\SetKwFunction{backtrack}{backtrack}
\SetKwFunction{print}{print}
\SetKw{return}{return}
\SetKwFunction{nondet}{nondet}
\SetKwFunction{run}{run}
\SetKwFunction{learn}{learn}
\SetKwFunction{continue}{continue}
\SetKwFunction{assign}{assign}
\SetKwRepeat{Do}{do}{while}
%\SetKwFunction{assume}{assume}
%\SetKwFunction{isf}{isFeasible}
\SetKwData{conflict}{conflict}
\SetKwData{safe}{safe}
\SetKwData{sat}{sat}
\SetKwData{unsafe}{unsafe}
\SetKwData{unknown}{unknown}
\SetKwData{true}{true}
\SetKwInOut{Input}{input}
\SetKwInOut{Output}{output}
\SetKwFor{Loop}{Loop}{}{}
\SetKw{KwNot}{not}
\begin{small}
\Input{%An abstract domain $A$, 
A program $\mathcal{S}$, a
  propagation heuristics $H$, a decision heuristics $\mathit{decide}$,
and a conflict analysis heuristics $\mathit{conflict\_analysis}$.\rmcmt{calling
them heuristics is not a good idea since these are key phases of the cdcl
algorithm. Each phase may have a heuristics for some sub phase but the algorithm
of each phase still remians a procedure and not a heuristic}}
\Output{The status \safe or \unsafe, and a counterexample if \unsafe.}
$\mathcal{T} \leftarrow \langle\rangle$ \;
$\mathcal{R} \leftarrow \emptyset$ \;
$\mathit{result} \leftarrow \mathit{deduce}(S,\mathcal{T},\mathcal{R}, H)$ \;
\lIf{$\mathit{result}$ = \conflict} {
  \return \safe}
\While{$true$} 
{
\lIf{$\mathit{result}$ = \sat} {
  \return \unsafe}
  $d \leftarrow \mathit{decide}(\mathit{abs}(\mathcal{T}))$ \;
%  $v \leftarrow v \meet d$ \; 
  $\mathcal{T} \leftarrow \mathcal{T} \cdot d$ \; 
  $\mathcal{R}[|\mathcal{T}|] \leftarrow $\textsf{decision} \;
  $\mathit{result} \leftarrow \mathit{deduce}(\mathcal{S},\mathcal{T},\mathcal{R})$\;
  \Do{$\mathit{result} = \conflict$} {
%    learn() \;
    \lIf{$\neg \mathit{conflict\_analysis}(\mathcal{S},\mathcal{T},\mathcal{R})$} {
      \return \safe
    }
    $\mathit{result} \leftarrow \mathit{deduce}(\mathcal{S},\mathcal{T},\mathcal{R})$ \;
  }
}
\end{small}
\caption{Abstract Conflict Driven Clause Learning $ACDCL(A)$ \label{Alg:acdcl}}
\end{algorithm2e}
%
%\subsection{Algorithm for Abstract Conflict Driven Clause Learning}
Algorithm~\ref{Alg:acdcl} presents an overview of the ACDCL algorithm.
The algorithm takes a program \pscmt{make this a set of transformers
  (parametrised by the propagation heuristics)}
$\mathcal{S}$ and propagation, decision, and conflict analysis
heuristics as inputs.  
\rmcmt{Approximation of the concrete transformers in 
$\mathcal{S}$ are typically available in abstract domain in the 
form of strongest-post condition or weakest pre-condition. } 
As internal state, it maintains a propagation
trail $\mathcal{T}$ and a reason trail \pscmt{map?} $\mathcal{R}$.
The propagation trail is an initially empty sequence of meet
irreducibles inferred by the abstract model search phase (deductions
and decisions). The reason trail maps \pscmt{certain?} elements of the
propagation trail to transformers $s\in\mathcal{S}$ that were used to
derive them. The \emph{abstract value} $\mathit{abs}(\mathcal{T})$
corresponding to the propagation trail $\mathcal{T}$ is the
conjunction of the meet irreducibles on the trail:
$\mathit{abs}(\mathcal{T})=\bigsqcap_{m \in \mathcal{T}}m$ with
$\mathit{abs}(\mathcal{T})=\top$ if $\mathcal{T}$ is the empty
sequence $\langle\rangle$.

The procedure $deduce$ computes a greatest fixed point over the
transformers in $\mathcal{S}$ that refines the abstract value,
similar to the BCP step in SAT solvers.  If the result of $deduce$
returns \textsf{conflict} ($\bot$), the algorithm terminates with
\textsf{safe}.  Else, the analysis enters into the while($true$) loop
and makes a new decision by a call to $\mathit{decide}$ which returns
a new meet irreducible $d$ that is consistent with the state of the
solver. \pscmt{what happens if we cannot make any decision any more?
  -- then we should either have deduced $\bot$ or
  $\mathit{abs}(\mathcal{T})$ must be $\gamma$-complete} If $d$
refines the current abstract value, the corresponding propagation
and reason trail, $\mathcal{T}$ and $\mathcal{R}$ are updated.
%
For example, a decision in the interval domain restricts the range of 
intervals for variables.
% so the analysis jumps under a 
%greatest fixed-point \pscmt{???}. Note that the widening operation in abstract interpretation 
%jumps above a least fixed-point, so decisions can be viewed as 
%{\em dual widening} \pscmt{what do we learn?}.  

The procedure $\mathit{deduce}$ is called next to infer new meet
irreducibles taking into account the current decision.  The model search phase
alternates between the decision and deduction until $\mathit{deduce}$
returns either \textsf{sat} or \textsf{conflict}.
In the former case, we have found an abstract value that represents 
models of $\varphi$ and thus we return \textsf{unsafe}.
%
% has found an abstract value that represents a set of models of $\varphi$ or the
% deduction leads to a \textsf{conflict}.  Checking whether the abstract
% model concretises to a set of models of $\varphi$; this
% corresponds to a $\gamma$-completeness~\cite{dhk2013-popl} check in
% abstract interpretation.  If $v$ is $\gamma$-complete, then it cannot
% be refined further.  Thus, the algorithm returns the abstract model
% $v$, which is a set of concrete models, and terminates with
% \textsf{unsafe} or the current abstraction is insufficient to
% determine the satisfiability of $\varphi$.  

In the latter case the algorithm enters in the
$\mathit{conflict\_analysis}$ phase to learn the reason for the
conflict.  There can be multiple incomparable reasons for conflict --
based on the choice of Unique Implication Point (UIP)~\cite{cdcl}.
ACDCL heuristically chooses one reason.  A learnt clause must include
asserting cuts which guarantees derivation of new meet irreducibles
after backtracking. The clause learning and backtracking continues as
long as the result of deduction is conflicting ($\bot$), that is, the
abstract valuation $v$ does not abstractly satisfy the formula
\pscmt{define}.  If no further backtrack is possible, then the
algorithm terminates and $\varphi$ is \textsf{safe}. Else, the
algorithm makes a new decision and the above process is repeated until
a real counterexample \pscmt{is this a model?} is obtained or the
algorithm backtracks to decision level 0 after a conflict in which
case it returns \textsf{safe}.

\paragraph{Solver State.}  \pscmt{This paragraph seems redundant. Doesn't this redefine $\mathcal{T}$?} The state of a ACDCL solver is a tuple 
of the form $\langle \mathcal{E}, S \rangle$.  Here, $\mathcal{E}$ 
is a sequence of labelled information of the form $(m,s)$ where 
$m$ is a meet irreducible and $s = \mathsf{decision}$ if $m$ is a decision, 
or $s = \mathsf{deduction}$ if $m$ is inferred by a deduction. And $S$ is 
a set of abstract deduction transformer \pscmt{???}.  A trail in a SAT solver 
stores variable assignments of the form $(p, t)$, where $p$ is 
a propositional variable that appears at most once in the trail 
and $t$ is a truth assignment (true or false).  Whereas, a trail 
$\mathcal{T}$ in ACDCL contains a sequence of meet irreducibles 
inferred by deduction or decision phase where a variable in trail 
can be assigned \pscmt{constrained?} multiple times, each time with increasingly precise 
bounds \pscmt{that's specific for a particular domain}.  
\rmcmt{define trail refinement}
\Omit {
A current valuation $v$ is a meet of all elements of trail
$\mathcal{T}$, such that $v = \top$ when $\mathcal{T}$ is 
empty or $v=\underset{i \geq 0 \wedge i \leq |\mathcal{T}|}{\meet m_i}$, where 
$m_i \in \mathcal{T}$.  A solver is in conflict if some clauses in
$\mathcal{S}$ is not abstractly satisfied by $v$.
}
%\pscmt{define ``consistent with trail''}

%===============================================================================

%===============================================================================
\section{Abstract Domains}
The commonly used library for numerical abstract domains  
is the APRON C library~\cite{apron}.  This library is 
used for the static analysis of the numerical variables 
of a program by abstract interpretation. APRON provides a 
C API interface to various abstract domains and libraries 
such as {\em BOX}, {\em OCTAGON}, {\em Convex Polyhedra} and
{\em Linear Equalities} library.  The aim of such analysis is 
to compute invariants over numerical variables in the 
program~\cite{se2011}. 

\subsection{Interval Abstract Domain}
Domain:
Operators:
Semantics of elements:

\subsection{Octagon Abstract Domain}
Domain:
Operators:
Semantics of elements:

\subsection{Conditions for Absract Domains}
One of the requirements for a clause learning SAT solver 
is that each element of the partial assignments domain 
needs to have a decomposition into precisely complementable 
elements.  This property of the underlying domain helps 
guiding the search away from the conflicting region of the search 
space.

Thus, clause learning SAT solvers requires a weak complementation property
of the abstraction.  However, numerical abstract domain,  such as 
intervals and octagons are intersections of complementable half-spaces. 
APRON library is specialized for the abstract interpretation 
framework which does not require complementation 
operator.  To this end, we implement our own template-based 
octagon domain which supports complementation operator.  

For example, the octagon in example can be written as a conjunction of:
\[(x>=-2) \land (x<=1) \land (y>=-1) \land (y<=2) \land (x+y<=2) \land (x-y<=1) \land (y-x<=3) \land (-x-y<=2)\] 

The complementation of the above octagon can be written as disjunction of:
\[(x<-2) \lor (x>1) \lor (y<-1) \lor (y>2) \lor (x+y>2) \lor (x-y>1) \lor (y-x>3) \lor (-x-y>2)\]


%===============================================================================

%===============================================================================
\section{Abstract Model Search}
%
\begin{figure}[t]
\scriptsize
\begin{tabular}{l|l|l}
\hline
C program & SSA & Octagon \\
\hline
\begin{lstlisting}[mathescape=true,language=C]
int main() {
 unsigned x, y;
 __CPROVER_assume(x==y);
 x++;
 assert(x==y+1);
}
\end{lstlisting}
&
\begin{minipage}{4.40cm}
$\begin{array}{l@{\,\,}c@{\,\,}l}
SSA &\iff& ((g0 == TRUE) \land \\
    &    & (cond == (x == y)) \land \\
    &    & (g1 == (cond \&\& guard0)) \land \\
    &    & (x' == 1u + x) \land \\
    &    & (x' == 1u + y || !1))
\end{array}$
\end{minipage}
&
\begin{minipage}{3.75cm}
$\begin{array}{l@{\,\,}c@{\,\,}l}
C &\iff& ((x' > 1) \land (-x'-y < -2) \land \\
  &    & (-x-x' < -2) \land (y-x' < 0) \land \\                                                                
  &    & (x-x' < 0) \land (y > 0) \land \\
  &    & (x > 0) \land (-x'-y < 0) \land \\
  &    & (x+y > 1) \land (y-x < 1) \land \\
  &    & (x'-y < 2) \land (x-y < 1) \land \\
  &    & (x+y > 0) \land (x+x' > 0) \land \\
  &    & (x'-x < 2))
\end{array}$
\end{minipage}
\\
\hline
\end{tabular}
\caption{C Program, corresponding SSA and Octagonal Inequalities}
\label{ssa}
\end{figure}
%
%    
\begin{algorithm2e}[t]
\DontPrintSemicolon
\SetKwFunction{deduce}{deduce}
\SetKwFunction{print}{print}
\SetKwFunction{return}{return}
\SetKwFunction{continue}{continue}
\SetKwFunction{assign}{assign}
\SetKwData{sat}{SAT}
\SetKwData{conflict}{CONFLICT}
\SetKwData{unsat}{UNSAT}
\SetKwData{unknown}{UNKNOWN}
\SetKwData{true}{true}
\SetKwInOut{Input}{input}
\SetKwInOut{Output}{output}
\SetKwFor{Loop}{Loop}{}{}
\SetKw{KwNot}{not}
\begin{small}
\Input{Set $S$ of $\widehat{ded_{\varphi}}$ for $\varphi$ over Abstract domain $A$,
abstract valuation $v$, propagation trail $\mathcal{T}$, reason trail $\mathcal{R}$}
\Output{The status (\sat or \conflict or \unknown)}
worklist $\leftarrow$ initialize\_worklist($S$) \;
\Loop{}
{
// Greatest fixed point \;
\While{$!worklist.empty()$} 
{
  $\langle live\_vars, s \rangle \leftarrow$ worklist.pop() \;
  $\mathit{meet\_irrd} \leftarrow$ domain(s, live\_vars, $v$)\;
  \uIf{$meet\_irrd == \bot$} {
    $\mathcal{R}[\bot] \leftarrow s$ \;
    worklist.delete() \;
    \return \conflict \;
  }
  \uElse
  {
    $\mathcal{T} \leftarrow \mathcal{T} . \mathit{meet\_irrd}$ \; 
    $v \leftarrow v \meet \mathit{meet\_irrd}$ \; 
    $\mathcal{R}[|\mathcal{T}|] \leftarrow s$ \;
    worklist.update(v, s) \; 
  }
}
\uIf{$v$ is $\gamma$-complete} {
  \return \sat;
}
\uElse {
 \return \unknown \;
}
}
\end{small}
\caption{Abstract Model Search $\langle deduce(S,v,\mathcal{T},\mathcal{R}) \rangle$ \label{Alg:ms}}
\end{algorithm2e}
%  
A model search procedure in a SAT solver involves two steps -- {\em deductions} 
using unit rule refines current partial assignments and 
{\em decisions} to heuristically guess a value for an unassigned 
literal.  A unit rule overapproximates a model transformer and deduction 
computes a greatest fixed point over the partial assignments
domain~\cite{dhk2013-popl}.  We present an abstract model search procedure 
that computes a greatest fixed point over meet irreducible deduction 
transformer in $S$.  
%

%-------------------------------------------------------------------------------
\subsection{Abstract Transformers} \label{sec:abst}
%-------------------------------------------------------------------------------

To make our algorithm efficient, we have to focus abstract
transformers on performing only the minimally necessary
work. 
%
We thus define a specialised variant of the transformer to compute
deductions w.r.t.\ a given subdomain $L\sqsubseteq A$; $\bot,\top \in L$.
\[\llbracket s \rrbracket^\sharp(a,L)=\min \{a'\mid a'\in L, a'\models s\}\]
We have the property $\llbracket s \rrbracket^\sharp(a)\sqsubseteq \llbracket s \rrbracket^\sharp(a,L)$.
%
For example, for
$a=(0\leq y \leq 1 \wedge 5\leq z)$ we have
$\llbracket x=y+z\rrbracket^\sharp(a,\{x\})=(x\geq 6)$, whereas
$\llbracket x=y+z\rrbracket^\sharp(a,\{y,z\})=\top$.
%
If $L=A$ then intuitively $a$ is intersected with the constraint $s$.
Yet, with the help of $L$ we can also control whether to perform
forward (right-hand side to left-hand side, $L=\{x\}$ in the above
example) or backward (lhs to rhs, $L=\{y,z\}$) propagation.

Moreover, we want to know what the reasons for a specific deduction are.
\[ded(s,a,L)=\{R\rightarrow d\mid R=reasons(s,a,L,d), d \in decomp(\llbracket s \rrbracket^\sharp(a,L))\}\]
where 
$reasons(s,a,L,d)=\min \{R'\mid R'\subseteq decomp(a), \llbracket s\rrbracket^\sharp(R,L)=d\}$.

\pscmt{give an example for $ded$, link it to the reason trail}

\pscmt{define $ded$ for all statements, connect this to worklist}

\pscmt{link livevars to $L$ -- the subdomain spanned by livevars}

%-------------------------------------------------------------------------------
\subsection{Algorithm for Deduction Phase}
%-------------------------------------------------------------------------------
%
\rmcmt{Talk about lazy closure operation}
Algorithm~\ref{Alg:ms} presents the deduction phase in abstract model search procedure.
A {\em worklist} is initialized following a heuristics choice for iteration 
strategies. A {\em forward} iteration strategy initializes the worklist with 
transformers that has constants in the right-hand side. Whereas, a {\em backward} 
iteration strategy starts with a set of assertions. On the other hand, a 
{\em chaotic} iteration strategy initializes the worklist with set of all 
transformers.  For every meet irreducible deduction transformer $s \in S$ 
passed to the abstract domain, if the domain infers a new meet irreducible 
$\mathit{meet\_irrd}$, it is added to $\mathcal{T}$.  Additionally, the reason 
trail $\mathcal{R}$ is updated with the transformer $s$ as the reason for inferring
$\mathit{meet\_irrd}$. If a conflict is deduced, that is, $\mathit{meet\_irrd} = \bot$ 
is inferred by the domain, the algorithm terminates with CONFLICT.  However,
when a fixed-point is reached, the abstract valuation $v$ is checked for set of
models.  This is determined by checking whether the set of assignments in $v$ is 
$\gamma$-complete~\cite{gamma}.  This can be achieved by checking if there is a 
concrete solution $c$ such that the inequality $c(x) \subseteq \gamma \circ
v(x)$ holds for every meet irreducible $x$.  If $v$ is $\gamma$-complete, that is, the 
abstract valuation $v$ is abstractly satisfying, the abstract model 
search terminates with SAT.  Otherwise, the algorithm returns UNKNOWN and the 
model search procedure makes a new decision.    




%===============================================================================

%===============================================================================
\subsection{Decision}
A decision $\mathcal{D}$ is a meet irreducible that refines the 
current abstract valuation.  A decision can be of the form \pscmt{that's too restrictive?}
$\langle x R c \rangle$ or $\langle x R y \rangle$, where 
$R \in \{\preceq, \succeq\}$, $c$ is the bound and 
$\{x, y\}$ are program variables.  A decision must always 
be consistent with respect to the trail $\mathcal{T}$, that 
is, it must not contradict with the elements in the trail.  

Let $\mathcal{V}$ be a set of all singletons and non-singletons 
variables.  A singleton variable is one that has been assigned a 
singleton value, for example, $\langle x:[1,1] \rangle$.  Otherwise, 
the variable is non-singleton \pscmt{this definition only works for non-relational domains}.  Given a set of variables $\mathcal{V}$, 
a decision phase heuristically chooses a non-singleton branching 
variable $v \in \mathcal{V}$, a bound $c$, and the polarity ($\succeq$ or 
$\preceq$).  A decision changes the sate of the solver by adding new element 
$m$ and labelling information $s=decision$ to the trail, which is described below. 
\[decide: \quad (\mathcal{E},S) \rightarrow (\mathcal{E}(m,s),S) \]

ACDCL supports several decision heuristics namely, {\em ordered}, 
{\em longest-range}, {\em random}, {\em relational} and 
{\em Berkmin}~\cite{eugoldberg07} decision heuristics.  
The {\em ordered} decision heuristics creates an ordering among non-singleton 
variables, thereby making decisions on conditional variables (variables that 
appear in conditional branches) first before choosing numerical variables.  
This variable ordering is path-sensitive as it gives an effect of trace
partitioning.  

The {\em longest-range} heuristics simply keeps track of the
interval $[l,u]$ range ($r=u-l$) of a variable and chooses a variable with 
the longest range value \pscmt{why is this restricted to variables?}.  This ensures a fairness policy in selecting a 
variable since it guarantees that the intervals of variables are uniformly 
restricted.  

The {\em random} decision heuristics arbitrarily picks a variable 
for making decision. \pscmt{uniformly distributed?}

The {\em relational} decision heuristics is of the form 
$x R y$ for $R \in \{preceq, succceq\}$ and is only relevant for relational 
abstract domains.  

Whereas, a {\em berkmin} decision heuristics is inspired 
from decision heuristics used in Berkmin~\cite{eugoldberg07} SAT solver, which 
keeps track of the activity of a variable that participate in conflict clauses 
as well as variables that actively contribute to conflicts but do not explicitly 
appear in conflict clauses.  The set of conflict clauses is organized 
chronologically with the top clause as the one deduced in the last.  A 
branching variable is chosen among the free variables whose literals are 
in the top unsatisfied conflict clause.  A similar decision heuristics is 
also implemented in Chaff~\cite{chaff} SAT solver, that computes the activity 
of a variable as the number of occurrences of that variable in conflict 
clauses only. \pscmt{how do you adapt this to relational domains?}

A bound of a variable \pscmt{why restricted to variables?} is heuristically chosen to be an approximation of the 
arithmetic average of the current bounds.  However, the polarity ($\preceq$ or
$\succeq$) of a variable is chosen randomly.  

%===============================================================================

%===============================================================================
\section{Abstract Conflict Analysis}
A conflict analysis in a SAT solver can be seen as abductive 
reasoning where a singleton assignment $c$ is replaced by a partial 
assignment that is sufficient to infer $c$.  Thus, a conflict 
analysis is a generalization of an underapproximation of a set of 
countermodels~\cite{sas12,dhk2013-popl}, where a countermodel is a 
set of structures that do not satisfy a formula.  Since relational 
abstract domains have more complex structure than a partial 
assignments domain, so the lifting of {\em first-UIP} algorithm~\cite{uip} 
and clause learning to any arbitraty lattice structure is a non-trivial task. 

\paragraph {\textbf{Abstract Conflict Graph}}
The first-UIP algorithm in SAT solver works on a data structure called 
{\em implication graph} that computes a cut that suffices to produce a 
conflict.  The trail in ACDCL implicitly encodes a graph structure called 
{\em abstract conflict graph}, which records the dependencies between 
deductions made during abstract model search phase.  Nodes in the graph 
represent elements of the trail $\mathcal{T}$.  Nodes can be either 
{\em decision} node or {\em deduction} node. Edges can be extracted from
the transformers in the reason trail.  Incoming arrows to node $n$ 
indicate that the predecessors of $n$ are sufficient to deduce $n$.  
A decision node has no incoming edges.  
For example, consider a formula 
$\varphi:= (x=5 \wedge y=x \wedge y=y+x \wedge y \leq 0)$ with current 
abstract valuation $v:= (x=5)$.  Figure~\ref{conflict} shows a 
snapshot of an abstract conflict graph that 
stores the deductions obtained from abstract model search using octagon domain.  
The last deduction conflict with the constraint $(y \leq 0)$.  

\paragraph {\textbf{Lifting UIP to relational domains}}
An UIP is a special node in the abstract conflict graph such that 
any path from the last decision node to the conflict node must pass 
through it.  Unlike a SAT solver, a UIP in ACDCL may involve meet irreducibles 
that involves the same variables.  The first-UIP is a node closest to the 
conflict, and the last UIP is the decision node itself.  Computing UIPs ensures 
asserting cuts, that is, it yields clauses that generates new deduction after 
backtracking.  Every cut in the graph reason for conflict that can be used in learning.  
A first-UIP algorithm traverses the trail $\mathcal{T}$ starting from the 
conflict node and iteratively finds new nodes that contradict with the 
transformer stored in $reasons[\bot]$.  

For our example, there exists multiple incomaparable reasons for conflict,
marked as {\em cut0, cut1}, in figure~\ref{conflict}.  Here, cut0 is the first UIP.  
Choosing cut0 yields a learnt clause 
$L_0: (x+y<15 \vee -x-y<-15 \vee -x+y<5 \vee x-y<-5 \vee y-10<0 \vee -y+10<0)$, 
which is obtained by negating the reason for conflict.  
%
\begin{figure}
%\caption{An example of Octagon}\label{octagon}
\scalebox{.65}{\import{figures/}{conflict_graph.pspdftex}}
\label{conflict}
\end{figure} 
%    
\paragraph{\textbf{Learning in Relational Lattice}}
We model learning as a meet irreducible deduction transformer, called 
{\em abstract unit transformer (AUnit)}.  For an 
abstract lattice $A$ with complementable meet irreducibles 
and a set of conflicting meet irreducibles (countermodels) 
$C \subseteq A$ of $\varphi$ such that \meet{C} does not 
satisfy $\varphi$, $AUnit_C: A \rightarrow A$ is defined as follows. 
\[ AUnit(a) =
 \begin{cases}
  \bot       & \quad \text{if } a \sqsubseteq \meet{C} \\
  \bar{t}    & \quad \text{if } t \in C \; \text{and} \; \forall t' \in C
  \setminus \{t\}. a  \sqsubseteq t' \\
  \bar{t}    & \quad \text{if } t \in C \; \text{and} \; \forall t' \in C \setminus \{t\}. a
  \not\sqsubseteq t', \; \text{then} \\ 
             & \quad \forall a' \subseteq decomp(a) \;  \text{and} \; \forall a'' \in a \setminus \{a'\}, \\ 
             & \quad \meet{a''} \sqsubseteq r' \; \text{and} \; a' \not\sqsubseteq t  \\
  \top & \quad \text{otherwise} \\
 \end{cases}
\]
For example, let $C = \langle x+z \geq 10,  k \geq 0 \rangle$ be the conflicting set 
of meet irreducibles of $\varphi$ and let $a= \langle x+y \geq 2, y+z \geq 5,
x+z \leq 15, k:[1,1] \rangle$.  Applying the third rule of $AUnit_C$, we get a 
decomposition of $a$ into $a'$ and $a''$ where 
$a'= \langle x+y \geq 2, y+z \geq 5, x+z \leq 15 \rangle \not\sqsubseteq (x+z \geq 10)$, and 
$a''= \langle k:[1,1] \rangle \sqsubseteq (k \geq 0)$.  Then, $AUnit_C(a) = (x+z < 10)$, 
which constrains the bound of the octagonal constraint.  
Let us consider another example, where $C = \langle x \geq 2, x \leq 5, y \leq 7 \rangle$
and let $a = \langle x:[3,4], y:[5,6] \rangle$, then $AUnit_C(a) = \bot$ using
the first rule, since $a \sqsubseteq C$.  Now, let $a' = \langle x:[3,4] \rangle$, 
then $AUnit_C(a') = (y > 7)$ using second rule, since $a' \sqsubseteq (x \geq 2)$
and $a' \sqsubseteq (x \leq 5)$.
 
\paragraph {\textbf{Backjumping}}
A backjumping procedure undo all the assignments in the trail up to 
a decision level that restores the solver to a consistent state 
(non-conflicting).  The backjumping level is defined by the literal 
of the conflict clause that is closest to the root (decision level 0) 
where the conflict clause is still unit. If a conflict clause is 
globally unit, then the backjumping level is the root of the search tree.

The abstract clause learning and backjumping procedures in the abstract 
conflict graph is stated in terms of the state of ACDCL solver as follows. 
\[AbsLearn: \quad  (\mathcal{E},S) \rightarrow (\mathcal{E},S \wedge
\mathcal{L}) \quad \text{if} \; \mathcal{L} \notin S \; \textrm{and}
\; (S \wedge \mathcal{L}) \; \text{is not UNSAT} \]
\[AbsBackjump: \quad (\mathcal{E}_1(m,s)\mathcal{E}_2,S) \rightarrow
(\mathcal{E}_1,S) \quad \text{if} \; (\mathcal{E}_1,S) \; \text{is
consistent} \]   


\Omit {
\subsection{Clause Learning in Abstract Lattice}
Conflict graph for Intervals
Conflict graph for Octagons

\textit{Characteristics of Conflict clause}
\begin{enumerate}
\item A conflict clause must include asserting cuts. An asserting cut is a cut
that contain exactly one node at the current decision level. Assertion cuts yields 
clauses that can be used to derive new information after backtracking.

\item A conflict clause must be UNIT after backtracking. 

\item There can be multiple cuts and hence multiple UIPs. In other words, there
can be multiple incomparable reasons for a conflict. But conflict analysis
procedure choses one that is asserting. 

\item The conflict clause should be made false by the current partial assignment
and thus exclude an assignment leading to conflict. 
\end{enumerate}

1.DPLL style -- chronological backtracking \\
2. CDCL style -- non-chronological backtracking \\
  a. first-uip \\
  b. last-uip \\

**********************************************
\subsection{Lifting First UIP to Octagon domain}
**********************************************
unit-ness guarantee in octagon domain:
  Popped stmt: y23=1+y21
   Abstract value:
   D1: y23-y21 < 2 &&
   D2: y23+y21 > 0 &&
   D3: y21 < 1
   After backtracking, apply unit rule 
   y23 > 1 -- deduction from unit rule
   Value inconsistent !!
 
 Note: 
 1> cannot make reasoning at literal level for relational domain because literals are dependant on each other. As soon as literals denote relation between first-order variables, the pure reasoning on boolean skeleton is not sufficient. 
 
2> Intervals are orthogonal half spaces similar to booleans. 

3> After backtracking, the application of unit rule is done as follows for relational domains:
  Pass the learnt clause (as statement) and the abstrat value to the domain to make deductions.
}

%===============================================================================

%===============================================================================
\section{Experimental Results}
We have implemented the instantiation of ACDCL on relational domains 
in 2LS verification tool~\cite{2ls}.  The abstract domains used for 
our experiments are intervals, template polyhedra, octagons and equalities.  
We verified a total of 70 benchmarks in ANSI-C, which are derived from 
(a) bit-vector regression category in SV-COMP'16, (c) bit-precise and 
cycle accurate software models of hardware circuits automatically 
generated using v2c~\cite{mtk2016}, (c) controller code generated 
from Simulink (d) several hand-crafted benchmarks for equivalence 
checking and bounded loop analysis.  All benchmarks containing 
bounded loops are completely unrolled before analysis.  We compare 
our tool with a state-of-the-art bounded model checker CBMC and a 
commercial static analysis tool, Astr{\'e}e.  CBMC uses MiniSAT 
solver in the backend.  Astr{\'e}e was configured with interval 
domain, trace-partition domain, and relational domains.  To enable precise 
analysis using Astr{\'e}e, all our benchmarks are manually instrumented with 
partition directives which provides external hint to the tool to guide 
the trace partitioning heuristics.  Usually, such high-precision is not 
needed for static analysis, since it makes the analysis very expensive.  
Without trace partitioning, the analysis using Astr{\'e}e shows high 
degree of imprecision. 

\paragraph {\em Observation of Analysis}

\paragraph {\em Decision Heuristics} We compare the performance of 
different decision heuristic on our benchmarks.  We observe that 
the ordered heuristic outperforms other heuristics on control-intensive 
benchmarks due to its ability to prioritize decisions on variables that 
appears in conditional branches.  For straight-line code and most of 
the bit-vector regression suite, random heuristics performs the best.  
Whereas, the activity based heuristics such as Berkmin heuristic which 
works well in propositional cases performs best for safe benchmarks 
that encountered maximum number of conflicts to prove safety.

\paragraph {\em Clause Learning}      

  

%===============================================================================

%===============================================================================
\section{Conclusions}
In this paper, we present a procedure for lifting the model search 
in satisfiability solvers to relational abstract domains.  We explain 
specific properties of abstract domain elements that are necessary to infer 
domain-specific deductions.  We present decision heuristics that takes 
into account the high-level structure of programs and the underlying 
expressivity of the abstract domains.  Our conflict analysis procedure 
learns abstract transformer following UIP computation over relational 
elements.  We instantiate our algorithms as a program analyzer to determine
safety of C programs.  Experimental evaluation shows that ACDCL is able 
to suitably exploit the expressivity of richer abstract domains as well 
as precision of CDCL architecture.

Generalizations of conflict reasons for relational domains 
and non-relational domain such as trace-based abstrac domains is a 
possible direction of future work.  We believe that domain refinement 
in ACDCL will also help to dynamically adjust the precision of the 
deduction in an effective manner. 


\Omit{
Experimental evaluation on different benchmark sets shows that
ACDCL is able to suitably exploit the expressivity of richer
abstract domains as well as precision of CDCL architecture.
This is attributed to the intelligent decision heuristics
which exploits the high-level structure of the problem
combined with stronger deductions and clause learning
mechanism aided by the richer abstract domains.
Future work -- domain refinement, incremental solving, 
invariant inference, intelligent $\gamma$-completeness check. 
Implement dedicated abstract domains to exploit full performance.
generalization for relational domain, instantiate acdl with trace-based domains.
}

%===============================================================================

\bibliographystyle{splncs03}
\bibliography{biblio.bib}

\end{document}
